\documentclass[%
  libro,          
  corpo=12pt,        % corpo normale 11pt
  tipotesi=custom, % laurea triennale
  draft
]{toptesi}

\usepackage[T1]{fontenc}
% \usepackage[utf8]{inputenc} %ignored by luatex?
\usepackage{toptesi}
\usepackage{xcolor}

% \usepackage[printwatermark]{xwatermark}
% \newwatermark[allpages,color=red!30,angle=45,scale=3,xpos=0,ypos=0]{BOZZA}
\usepackage{tikz}
\AddToHook{shipout/background}{
       \begin{tikzpicture}[overlay, remember picture]
            \node[fill=yellow!30, draw, text=red!30, rotate=45, scale=5] at (current page) {BOZZA};
        \end{tikzpicture}    
    }

\begin{pdfxmetadata}
  \Title{Titolo}
  \Author{Alessandro Cerati}
  \Publisher{Alessandro Cerati}
  \Keywords{Monografia di laurea \sep
            Geometria simplettica}
\end{pdfxmetadata}

% \usepackage[a-1b]{pdfx}% oppure [a-2b]

\usepackage{fontspec}
\setmainfont[Ligatures=TeX]{LibertinusSerif}[
  Extension = .otf,
  UprightFont = *-Regular,
  BoldFont = *-Bold, 
  ItalicFont = *-Italic,
  BoldItalicFont = *-BoldItalic
]
\setsansfont[Ligatures=TeX]{LibertinusSans}[
  Extension = .otf,
  UprightFont = *-Regular,
  BoldFont = *-Bold, 
  ItalicFont = *-Italic,
]
\setmonofont{NewCMMono10-Regular.otf}
\setmainlanguage[babelshorthands]{italian}

\usepackage{amsmath}
\usepackage{unicode-math}
\setmathfont{XITS Math}
\usepackage{physics}
\newcommand{\R}{\mathbb{R}}
\renewcommand{\vec}[1]{\boldsymbol{#1}}
\newcommand{\identity}{\mathcal{i}}
\newcommand{\e}{\mathrm{e}}
\newcommand{\defeq}{\equiv}

\usepackage{amsthm}
\usepackage{thmtools, thm-restate}
\declaretheorem[
  style = plain,
  parent = chapter,
  name = Teorema,
  refname={teorema,teoremi},
  Refname={Teorema,Teoremi}]{theorem}
\declaretheorem[
  style = definition, 
  sibling = theorem, 
  name = Definizione,
  refname={definizione,definizioni},
  Refname={Definizione,Definizioni}]{definition}
\newcommand{\dfn}[1]{\emph{#1}}
\declaretheorem[
  style = remark,
  parent = theorem,
  name = Osservazione,
  refname={osservazione,osservazioni},
  Refname={Osservazione,Osservazioni}]{remark}

\declaretheorem[
  style = theorem,
  parent = theorem,
  name = Corollario,
  refname={corollario,corollari},
  Refname={Corollario,Corollari}]{corollary}

\declaretheorem[style = plain, numbered = no, name = Legge di Newton]{newton}
\declaretheorem[style = plain, numbered = no, name = Principio di minima azione]{minaction}

\declaretheorem[
  style = plain,
  parent = chapter,
  name = Esempio,
  refname={esempio,esempi},
  Refname={Esempio,Esempi}]{example}



\usepackage{graphicx}
\graphicspath{./graphics}

\usepackage{hyperref} %autoloaded in xwatermark
\renewcommand{\equationautorefname}{equazione}
\renewcommand{\chapterautorefname}{capitolo}
\renewcommand{\sectionautorefname}{sezione}

\begin{document}
\renewcommand{\epsilon}{\varepsilon}
\renewcommand{\phi}{\varphi}

  % \begin{titlepage}
  %
  %
  % UNA VOLTA FATTE LE DOVUTE MODIFICHE SOSTITUIRE "RED" CON "BLACK" NEI COMANDI \textcolor
  %
  %
  \begin{center}
  {{\Large{\textsc{Alma Mater Studiorum $\cdot$ Università di Bologna}}}} 
  \rule[0.1cm]{\textwidth}{0.1mm}
  \rule[0.5cm]{\textwidth}{0.6mm}
  \\\vspace{3mm}
  
  {\small{\bf Scuola di Scienze \\ 
  Dipartimento di Fisica e Astronomia\\
  Corso di Laurea in Fisica}}
  
  \end{center}
  
  \vspace{23mm}
  
  \begin{center}\textcolor{red}{
  %
  % INSERIRE IL TITOLO DELLA TESI
  %
  {\LARGE{\bf Geometria simplettica: una formulazione della meccanica libera da coordinate}}\\
  }\end{center}
  
  \vspace{50mm} \par \noindent
  
  \begin{minipage}[t]{0.47\textwidth}
  %
  % INSERIRE IL NOME DEL RELATORE CON IL RELATIVO TITOLO DI DOTTORE O PROFESSORE
  %
  {\large{\bf Relatore: \vspace{2mm}\\\textcolor{red}{
  Prof. Emanuele Latini}}}\\\\

  \end{minipage}
  %
  \hfill
  %
  \begin{minipage}[t]{0.47\textwidth}\raggedleft \textcolor{red}{
  {\large{\bf Presentata da:
  \vspace{2mm}\\
  %
  % INSERIRE IL NOME DEL CANDIDATO
  %
  Alessandro Cerati}}}
  \end{minipage}
  
  \vspace{40mm}
  
  \begin{center}
  %
  % INSERIRE L'ANNO ACCADEMICO
  %
  Anno Accademico \textcolor{red}{2023/2024}
  \end{center}
  
  \end{titlepage} 
  % \chapter*{Introduzione} \addcontentsline{toc}{chapter}{Introduzione}

Il formalismo lagrangiano della meccanica classica costituisce un vero e proprio cambio di paradigma rispetto a quello newtoniano. Esso infatti non si limita a sostituire la celeberrima legge $\vec{F}=m\vec{a}$ di Newton con il principio di minima azione, ma riduce al minimo la presenza dei vettori, entità matematiche principi della meccanica newtoniana. Obiettivo principale di questa operazione è la covarianza: a prescindere dal sistema di coordinate impiegato, siano esse cartesiane, sferiche o iperboliche, le equazioni di Eulero-Lagrange ottenute dal principio di minima azione avranno la stessa forma. Più simmetria e ancora maggiore libertà di scelta di coordinate per semplificare i calcoli si hanno passando al formalismo hamiltoniano, duale di quello lagrangiano. Il prezzo di questa potenza è però un pesante apparato computazionale che spesso offusca l'eleganza del formalismo.

Fortunatamente, questo prezzo può essere permutato. Gli strumenti della geometria differenziale, e in particolare le teorie delle varietà differenziabili, delle forme differenziali e dei gruppi di Lie, consentono di formulare la meccanica hamiltoniana in modo non solo covariante, ma addirittura del tutto libero dall'uso di coordinate e puramente \emph{geometrico}. La difficoltà viene così trasferita da un'ottusa complessità dei calcoli a una certa sottigliezza nella formulazione, dall'utilizzo degli strumenti alla loro definizione. La difficoltà concettuale che risulta da questo scambio è non solo notevolmente più soddisfacente da districare --- aspetto in ogni caso da non sottovalutare --- ma presenta anche notevoli vantaggi teorici e pratici. Dal punto di vista matematico, lo studio della \emph{geometria simplettica}, che generalizza le proprietà geometriche degli spazi delle fasi di sistemi meccanici, si è ormai affrancato dalla sua origine fisica e costituisce un fertile ambito di ricerca. Per quanto riguarda le applicazioni fisiche, invece, il formalismo simplettico è più semplice da espandere oltre la meccanica classica rispetto al newtoniano, e ha le stesse capacità di sfruttare le simmetrie di un sistema per ridurre la gravosità dei calcoli che rende l'hamiltoniano così utile.

Oltre agli impieghi diretti, la formulazione simplettica della meccanica classica costituisce un'opportunità di familiarizzare con concetti e strumenti che sono ormai onnipresenti nella fisica moderna, primi tra tutti quelli di \emph{varietà differenziabile}, fondamento della relatività generale, e \emph{gruppo di Lie} --- anche specificamente nel suo impiego per la formalizzazione delle simmetrie --- che costituisce invece uno degli elementi chiave del Modello Standard della fisica delle particelle. La teoria delle varietà costituisce inoltre un caso esemplare di impiego dell'astrazione matematica, nel suo separare concetti che nello spazio euclideo tridimensionale coincidono e nell'identificarne altri tra cui sussistono relazioni particolari.

Obiettivo di questo elaborato è giungere a una formulazione geometrica, in cui i concetti sono cioè definiti senza l'utilizzo di coordinate, della meccanica classica, attraverso la geometria simplettica definita canonicamente sui fibrati cotangenti delle varietà differenziabili, e definire il processo di riduzione simplettica che sfrutta le simmetrie di un sistema fisico per ridurre la dimensione dello spazio delle fasi trovando quantità conservate, nello specifico le componenti della mappa momento del gruppo di Lie associato alla simmetria. Nel primo capitolo verrà fornita una descrizione matematica dell'universo e dei moti dei corpi in esso e verranno velocemente passate in rassegna le formulazioni della meccanica classica trattate nel corso triennale di Fisica: newtoniana, lagrangiana, hamiltoniana. I successivi due capitoli si concentreranno sul fornire gli strumenti matematici necessari per la formulazione simplettica. Nel secondo capitolo si definiranno e si daranno gli strumenti per trattare varietà differenziabili e loro atlanti di coordinate, vettori e spazi tangenti e cotangenti, e forme differenziali che generalizzano questi ultimi. Nel terzo capitolo si forniranno gli strumenti più specifici alla meccanica simplettica: per prima cosa la forma simplettica stessa, per poi passare a gruppi e algebre di Lie, alla mappa momento e al teorema di riduzione simplettica. Nel quarto capitolo si applicheranno alla meccanica gli strumenti acquisiti, e come esempio di utilizzo della riduzione simplettica sarà analizzato il moto del corpo rigido libero.

  % \chapter{Formulazioni della meccanica}
La meccanica classica si propone di studiare il moto dei corpi macroscopici in movimento con velocità trascurabile rispetto a quella della luce. In questo capitolo esporremo tre \emph{formulazioni} della meccanica classica, ovvero tre modi per ricavare le equazioni del moto. Dopo aver richiamato l'intuitiva ma scomoda formulazione \emph{newtoniana} basata sulle forze, costruiremo la formulazione \emph{lagrangiana} dal principio di minima azione. Infine, grazie alla trasformata di Legendre, passeremo alla potente formulazione \emph{hamiltoniana}.

\section{Il modello classico dell'universo}
Per dare una formulazione assiomatica alla meccanica classica è innanzitutto necessario stabilire alcune caratteristiche dell'universo per quanto riguarda i corpi che si muovono su scala macroscopica (azione molto maggiore della costante di Planck $\hbar$) e a basse velocità (velocità molto minore di quella della luce $c$). Queste caratteristiche dovranno essere emulate dal modello di universo che costruiremo. Siccome l'ambito di applicazione di questo modello comprende l'esperienza quotidiana, è da essa che possiamo trarre spunto. Poniamo quindi i seguenti requisiti al nostro modello:
\begin{enumerate}
  \item L'universo è composto da \emph{spazio e tempo}. Lo spazio è tridimensionale ed euclideo, il tempo è unidimensionale.
  \item Localizziamo gli eventi nello spazio e nel tempo usando \emph{sistemi di coordinate}. Esistono sistemi, detti \emph{inerziali}, tali che:
  \begin{enumerate}
    \item A ogni istante, tutte le leggi della natura sono uguali in ogni sistema inerziale.
    \item Tutti i sistemi che si muovono di moto rettilineo e uniforme rispetto a un sistema inerziale sono inerziali.
  \end{enumerate}
  \item L'universo è popolato di \emph{particelle}, entità adimensionali dotate di una massa e una posizione. Esso è inoltre è \emph{newtonianamente deterministico}: l'insieme delle posizioni e delle velocità di tutti i particelle a un certo tempo determina tutto il suo moto, sia nel passato che nel futuro.
\end{enumerate}

Costruiamo ora un modello formale che presenti queste stesse caratteristiche. Per fare ciò, sarà necessaria una struttura matematica che formalizza l'idea di uno spazio-tempo in cui gli spostamenti sono vettori, ma non è definita un'origine. Tale struttura è nota come spazio affine.
\begin{definition}
  Uno \dfn{spazio affine $n$-dimensionale} è una struttura $(A,V,+)$ dove:
  \begin{itemize}
    \item $A$ è un insieme.
    \item $V$ è uno spazio vettoriale reale.
    \item $+:\, A\times V \to A$ è un'applicazion biunivoca detta \dfn{azione destra} di $V$ su $A$. L'immagine della coppia $(a,v)$ con $a \in  A, v \in V$ si denota con $a+v$.
  \end{itemize}
  La \dfn{dimensione} di $A$ è la dimensione di $V$.
\end{definition}
\begin{remark}
  Siccome $+$ è biunivoca, fissati $a,b \in A$ esiste uno e un solo $v \in V$ tale che $a + v = b$. Tale elemento di $V$ si denota con $v = b - a$. In tal modo, gli spostamenti (differenze) tra punti di $A$ sono effettivamente vettori.
\end{remark}
\begin{remark}
Fissato un $a \in A$, l'insieme dei $v$ tali che $v = b - a$ per un qualche $b \in A$ forma uno spazio vettoriale reale isomorfo a $V$. Si dice che $V = A - A$.
\end{remark}

Possiamo a questo punto formulare le componenti centrali del nostro modello: l'universo, il tempo, gli eventi contemporanei e la distanza fra di essi. Queste andranno a formare una struttura detta galileiana.
\begin{definition}
  Una \dfn{struttura spazio-temporale galileiana} è composta da:
  \begin{enumerate}
    \item L'\dfn{universo}, uno spazio affine quadridimensionale $\mathbb{A}^4$ i cui punti sono detti \emph{eventi}.
    \item Il \dfn{tempo}, un'applicazione lineare $t:\mathbb{A}^4 - \mathbb{A}^4 \to \R$. Si dice \dfn{intervallo temporale} fra due eventi $a,b \in \mathbb{A}^4$ il numero reale $t(a-b)$. Se esso è nullo, $a$ e $b$ si dicono \dfn{contemporanei}.
    \item La \dfn{distanza fra eventi contemporanei} \begin{equation}
    d(a,b) = \norm{a-b}_2 = \sqrt{\ev{a-b,a-b}}  
    \end{equation} 
  \end{enumerate}
\end{definition}

Queste definizioni tuttavia non consentono ancora di formulare la meccanica classica. Non abbiamo modo di identificare un punto nello spazio, siccome gli spazi di eventi contemporanei sono tutti separati tra loro. Per collegarli, introduciamo l'ultima componente, i sistemi di riferimento.
\begin{definition}
  Si dice \dfn{sistema di riferimento} un'applicazione lineare biunivoca $\phi:\, \mathbb{A}^4 \to \R\times \R^3$. Un sistema di riferimento $\phi_1$ si dice \dfn{in moto uniforme} rispetto a $\phi_2$ se $\phi_1 \phi_2^{-1}: \R\times \R^3\to \R\times \R^3$ conserva intervalli temporali e distanze.
\end{definition}

Siamo finalmente in grado di descrivere posizioni e moti dei particelle che popolano l'universo.
\begin{definition}
  Si dice \dfn{moto} di una particella in un sistema di riferimento $\phi$ un'applicazione differenziabile $\vec{x}:I \to \R^3$, dove $I$ è un intervallo aperto in $\R$, che manda $t\mapsto \vec{x}(t)$. Si dice \dfn{posizione} del punto al tempo $t \in I$ il vettore $\vec{x}(t)$. Si dicono \dfn{velocità} e \dfn{accelerazione} del punto al tempo $t \in I$ rispettivamente \begin{equation}
  \vec{\dot{x}} (t) = \dv{\vec{x}}{t}\, (t) \qqtext{e} \vec{\ddot{x}} (t) = \dv[2]{\vec{x}}{t}\, (t)
  \end{equation} 
\end{definition}
\begin{remark}
  Velocità e accelerazione così definiti sono vettori del medesimo spazio vettoriale $\R^3$ a cui appartiene la posizione.
\end{remark}

Per descrivere le posizioni di $N$ particelle servono $N$ vettori a 3 componenti. La risultante struttura può essere identificata con un vettore del prodotto diretto di $N$ copie di $\R^3$. Questo vettore descrive la configurazione dell'intero sistema, il che motiva la seguente definizione.
\begin{definition}
  Si dice \dfn{spazio delle configurazioni} di un sistema di $N$ particelle il prodotto diretto di $N$ copie di $\R^3$: \begin{equation}
  \mathbb{M} = \underbrace{\R^3 \times \R^3 \times \ldots \times \R^3}_{N \text{ volte}}
  \end{equation} 
  \dfn{Moto, configurazione, velocità} e \dfn{accelerazione} di un tale sistema sono definite analogamente a quanto fatto per una sola particella, sostituendo $\R^3$ con $\mathbb{M}$.
\end{definition}

\section{Meccanica newtoniana}
Finora ci siamo occupati solo di come descrivere i moti delle particelle nell'universo. L'obiettivo della meccanica, tuttavia, è determinare in anticipo i moti di queste particelle sulla base di informazioni note, grazie al principio di determinismo newtoniano. Nella formulazione originaria, dovuta a Newton stesso, i moti delle particelle sono previsti in base alle forze che agiscono su di esse, determinate sperimentalmente, tramite la legge di Newton. In versione moderna, questa può essere formulata come segue.
\begin{newton}
  Siano $\mathbb{M}$ lo spazio delle configurazioni di un sistema e $I$ un intervallo reale aperto, se il sistema segue un moto $\vec{x}:I \to \mathbb{M}$ esiste una funzione $\vec{F}:\mathbb{M} \times \mathbb{M} \times  I \to \mathbb{M}$ tale che \begin{equation}
  \vec{\ddot{x}} = \vec{F}\,(\vec{x},\vec{\dot{x}}, t) \label{eq:newton}
  \end{equation} 
  Questa è detta \dfn{equazione del moto} ed $\vec{F}$ è detta \dfn{forza} (generalizzata) agente sul sistema.
\end{newton}
\begin{remark}
  La funzione $\vec{F}$ si può ottenere dalla formulazione elementare della meccanica, che vede le forze come entità agenti su ciascuna particella in maniera distinta, moltiplicando la forza risultante su ciascuna particella per la massa di quest'ultima e concatenando i vettori risultanti.
\end{remark}
\begin{remark}
  La legge di Newton soddisfa il principio di determinismo grazie al teorema di esistenza e unicità di Cauchy: data la condizione iniziale, ovvero posizione e velocità del sistema a un tempo $t_0$, la soluzione dell'\autoref{eq:newton} esiste ed è unica nell'intervallo temporale $I$ in cui $\vec{F}$ è definita.
\end{remark}
La funzione $\vec{F}$ per un dato sistema fisico deve essere determinata sperimentalmente. Il modello matematico del sistema viene costruito nel momento in cui viene definita $\vec{F}$, motivo per cui essa è spesso identificata con il sistema stesso.

La conoscenza di $\vec{F}$ consente in linea di principio di conoscere il moto del sistema per qualsiasi posizione iniziale. Ciò tuttavia è molto più facile a dirsi che a farsi: per un sistema di $N$ particelle, questo metodo richiede di risolvere $3N$ equazioni differenziali, un processo che spesso è impossibile dal punto di vista analitico e inefficiente da quello numerico. È però possibile fare affermazioni qualitative sul moto dei corpi anche quando l'equazione del moto non è analiticamente risolvibile, attraverso i concetti legati all'energia.
\begin{definition}
  Si dice \dfn{energia cinetica di un sistema} formato da $N$ particelle di masse $m_i$ nelle posizioni $\vec{x}_i \in \R^3$ la quantità \begin{equation}
    T(\vec{\dot{x}}) = \sum_{i=1}^{N} \frac{1}{2}m_i\norm{\vec{\dot{x}}_i}^2
  \end{equation} 
\end{definition}
\begin{definition}
  Una forza dipendente solo dalle posizioni del sistema si dice \dfn{campo di forze}. Si dice \dfn{lavoro} del campo di forze $\vec{F}\, (x)$ sul cammino $\gamma \subset \mathbb{M}$ la quantità \begin{equation}
  L = \int\limits_{\gamma} \vec{F}\,(\vec{x})\cdot \dd \vec{x}
  \end{equation} 
  Un campo di forze e il corrispondente sistema si dicono \dfn{conservativi} se il lavoro del campo su un cammino qualsiasi non dipende dal cammino stesso. In tal caso, si dice \dfn{potenziale} rispetto a $\vec{x}_0 \in \mathbb{M}$ la quantità definita simbolicamente come \begin{equation}
  V(\vec{x}) = \int_{\vec{x}_0}^{\vec{x}} \vec{F}\,(\vec{x})\cdot \dd \vec{x}
  \end{equation} 
  dove l'integrale è compiuto lungo un qualsiasi percorso $\gamma$ di estremi $\vec{x}_0$ e $\vec{x}$.
\end{definition}
\begin{definition}
  Si dice \dfn{energia totale} di un sistema conservativo\begin{equation}
  E(\vec{x},\vec{\dot{x}}) = T(\vec{\dot{x}}) + V(\vec{x})
  \end{equation} 
\end{definition}
\begin{theorem} \label{thm:energyCons}
L'energia totale di un sistema conservativo è costante nel tempo.
\end{theorem}

Il \autoref{thm:energyCons} consente ad esempio di determinare la regione di spazio delle configurazioni ammessa per il moto di un sistema, date le condizioni iniziali.

\section{Meccanica lagrangiana}
La conservazione dell'energia di un sistema è un metodo piuttosto semplice per ottenere informazioni qualitative sul suo comportamento, ma queste non sono particolarmente dettagliate. La formulazione newtoniana è inoltre fortemente lineare: si è sostanzialmente forzati a usare le coordinate dello spazio vettoriale $\mathbb{M}$ per individuare un punto, anche quando il moto sarebbe più conveniente da descrivere attraverso altri sistemi di coordinate.

Si può ovviare a questi problemi attraverso la formulazione \dfn{lagrangiana} della meccanica, che ricava le equazioni del moto da una funzione scalare, non vettoriale come la forza generalizzata.

\begin{definition}
  Si dice \dfn{lagrangiana} di un sistema la funzione \begin{equation}
  \mathcal{L}(\vec{x},\vec{\dot{x}},t) = T(\vec{\dot{x}}) - V(\vec{x},t)
  \end{equation} 
  Si dice \dfn{azione} di un moto $\vec{x}:t\mapsto \vec{x}(t)$ l'integrale della lagrangiana nel tempo \begin{equation} \label{eq:action}
  S(t_0,t_1; \vec{x}) = \int_{t_0}^{t_1} \mathcal{L}(\vec{x}(t),\vec{\dot{x}}(t),t) \dd{t}
  \end{equation}
\end{definition}

Fissati un tempo iniziale e finale $t_0$ e $t_1$ e una configurazione iniziale e finale $\vec{x}_0$ e $\vec{x}_1$, l'azione è un funzionale definito sullo spazio dei moti $\vec{x}$ tali che $\vec{x}(t_0) = \vec{x}_0$ e $\vec{x}(t_1)=\vec{x}_1$. Per avere una formulazione della meccanica, è necessario un modo per determinare quale moto viene effettivamente realizzato. Intuitivamente, questo dovrebbe avere un qualche tipo di \textquotedblleft efficienza\textquotedblright: non dovrebbe seguire un percorso inutilmente lungo, né variare la sua velocità più del necessario. Queste idee intuitive sono raccolte in una formulazione matematica dal principio di minima azione, originariamente dovuto a Maupertuis.
\begin{minaction}
Il moto fisicamente realizzato da un sistema fra due punti nello spazio delle configurazioni è quello per cui l'azione è minima.
\end{minaction}
Siccome il principio di minima azione riguarda il moto nella sua interezza, il quale essendo un'applicazione differenziabile appartiene a uno spazio vettoriale infinito-dimensionale, il problema della determinazione pratica di quale sia effettivamente il moto lungo cui l'azione è minima necessita lo sviluppo del calcolo delle variazioni per essere risolto; nel presente elaborato ci limiteremo a presentare il teorema che risulta da una trattazione completa.
% \begin{definition}
  %   Si dice \dfn{funzionale} un'applicazione tra uno spazio di funzioni differenziabili e l'asse reale.
  % \end{definition}
\begin{theorem}
  Affinché un moto $\vec{x}: t\mapsto \vec{x}(t)$ sia un minimo dell'azione $S(\vec{x})$ (definita dall'\autoref{eq:action}) fissati tempi e punti iniziali e finali $t_0, t_1, \vec{x}_0, \vec{x}_1$ è necessario che esso soddisfi le \dfn{equazioni di Eulero-Lagrange} \begin{equation}
    \dv{t}\Bigg[\pdv{\mathcal{L}}{\vec{\dot{x}}}()(\vec{x}, \vec{\dot{x}}, t)\Bigg] - \pdv{\mathcal{L}}{\vec{x}}()(\vec{x}, \vec{\dot{x}}, t) = 0
  \end{equation} 
  dove $\pdv{\vec{x}}$ indica il gradiente $\nabla_{\vec{x}}$.
\end{theorem}

È possibile dimostrare che, come si spera, i moti ottenute per un dato sistema meccanico dalla formulazione newtoniana e da quella lagrangiana coincidono. Come menzionato sopra, la formulazione lagrangiana semplifica però notevolmente i calcoli in molte situazioni. Un primo vantaggio è dovuto al fatto che essa consente di svolgere i calcoli non solo usando le componenti dei vettori di $\mathbb{M}$, ma anche con coordinate arbitrarie.

\begin{definition}
  In un sottoinsieme dello spazio delle configurazioni $U \subseteq \mathbb{M}$ si dicono \dfn{coordinate} $\vec{q}=(q_1, \ldots, q_n)$ le $n$-uple appartenenti a un insieme $W$ tale che esiste una funzione biunivoca $\phi: \vec{q} \in W \mapsto \vec{x}\in U$.
\end{definition}
\begin{remark}
  In generale, non è detto che $\vec{\dot{x}}$ dipenda solo da $\vec{\dot{q}}$, ma potrebbe dipendere anche da $\vec{q}$, come in effetti spesso è il caso.
\end{remark}
\begin{theorem}
  Un moto soddisfa le equazioni di Eulero-Lagrange in $\vec{x}$ se e solo se le soddisfa in coordinate $\vec{q}$ arbitrarie per $\mathbb{M}$: \begin{equation}
    \dv{t}\Bigg[\pdv{\mathcal{L}}{\vec{\dot{q}}}()\big(\vec{x}(\vec{q}), \vec{\dot{x}}(\vec{q},\vec{\dot{q}}), t\big)\Bigg] - \pdv{\mathcal{L}}{\vec{q}}()\big(\vec{x}(\vec{q}), \vec{\dot{x}}(\vec{q},\vec{\dot{q}}), t\big) = 0
  \end{equation} 
\end{theorem}

Questo significa che le equazioni del moto di un sistema ad esempio in coordinate polari possono essere ottenute immediatamente una volta scritte l'energia cinetica $T\big(\vec{\dot{x}}(\vec{q},\vec{\dot{q}})\big)$ e il potenziale $V\big(\vec{x}(\vec{q})\big)$ in coordinate polari.

Fra i casi più comuni in cui è naturale individuare una configurazione con coordinate non lineari vi sono i sistemi vincolati, ovvero che non sono liberi di occupare l'intero spazio vettoriale delle configurazioni. La meccanica lagrangiana è particolarmente adatta anche a questo caso specifico.

\begin{definition}
  Sia $\mathbb{M}$ uno spazio delle configurazioni di dimensione $n$ con coordinate arbitrarie $\vec{q} = (q_1, \ldots, q_n)$. Si dice \dfn{vincolo olonomo} la richiesta che il moto abbia un'immagine contenuta nel sottoinsieme $M$ di $\mathbb{M}$ definito da \begin{equation}
    M = \left\{ a \in \mathbb{M} \left|\ \begin{cases} 
      F_1(\vec{q}) = 0 \\
      \vdots \\
      F_k(\vec{q}) = 0
    \end{cases} \right. \right\} \qqtext{con} \vec{q} = \phi^{-1}(a)
  \end{equation} 
  dove $k \le n$ e $F_1, \ldots, F_k$ sono funzioni differenziabili da $\mathbb{M}$ a $\R$, tali che il rango della matrice $\pdv{F_i}{q_j}(a)$ sia $k$ in ogni $a \in M$. $M$ è detto \dfn{spazio delle configurazioni di un sistema vincolato}.
\end{definition}
\begin{remark}
  Lo spazio vincolato $M$ non è necessariamente uno spazio vettoriale, ma può essere un qualsiasi sottoinsieme di $\mathbb{M}$.
\end{remark}
\begin{remark}
  Un vincolo non olonomo, detto \dfn{anolonomo}, può vincolare anche la velocità del sistema. Non parleremo qui di vincoli anolonomi, quindi nel seguito con \emph{vincolo} si intenderà \emph{vincolo olonomo}.
\end{remark}

Come si può imporre formalmente un vincolo su un sistema? Intuitivamente, un vincolo è un'entità che applica una forza infinita sul sistema quando esso tenta di uscire dallo spazio vincolato $M$. Le forze sono prerogativa della meccanica newtoniana, ma possiamo pensare di modellizzare una forza infinita con un potenziale a pendenza tendente all'infinito lungo le direzioni che si allontanano dal vincolo. Esprimendo per il teorema del Dini le prime $k$ coordinate in funzione delle altre $n-k$ \begin{equation}
\begin{cases}
  q_1 = f_1(q_{k+1}, \ldots, q_n) \\
  \vdots \\
  q_k = f_k(q_{k+1}, \ldots, q_n)
\end{cases}
\end{equation} 
il potenziale che ci serve è \begin{equation}
  V_{\alpha}(\vec{q}) = V(\vec{q}) + \alpha\big[(q_1-f_1)^2 + \ldots + (q_k-f_k)^2 \big] \qqtext{per} \alpha \to +\infty
\end{equation} 
dove $V$ è il potenziale dello stesso sistema senza vincoli. Usando questo potenziale è possibile dimostrare il seguente teorema.
\begin{theorem}
  Sia $M \subset \mathbb{M}$ lo spazio delle configurazioni di un sistema vincolato con lagrangiana $\mathcal{L}_{\alpha} = T -V_{\alpha}$, siano date le condizioni iniziali $\vec{q}_0 \in  M$ e $\vec{\dot{q}}_0$ tangente a $M$ e sia $\vec{\phi}_{\alpha}(t)$ il moto del sistema. Allora esiste il limite \begin{equation}
  \lim_{\alpha \to +\infty} \vec{\phi}_{\alpha}(t) = \vec{\psi}(t)
  \end{equation} 
  e la funzione limite $\vec{q}(t) = \vec{\psi}(t)$ soddisfa le equazioni di Eulero-Lagrange \begin{equation}
    \dv{t}\Bigg[\pdv{\mathcal{L}_{*}}{\vec{\dot{q}}}()(\vec{q},\vec{\dot{q}},t)\Bigg] - \pdv{\mathcal{L}_{*}}{\vec{q}}()(\vec{q},\vec{\dot{q}},t) = 0
  \end{equation} 
  dove $\mathcal{L}_*$ è detta \dfn{lagrangiana ridotta} ed è definita come \begin{equation}
  \mathcal{L}_*(\vec{q},\vec{\dot{q}}) = \eval{T(\vec{q},\vec{\dot{q}})\,}_{\begin{subarray}{l}
    q_1=f_1, \ldots, q_k=f_k \\
    \dot{q}_1, \ldots, \dot{q}_k = 0
  \end{subarray}} - \eval{V(\vec{q})\,}_{q_1=f_1, \ldots, q_k=f_k}
  \end{equation} 
\end{theorem} 
Questo teorema consente in pratica di scrivere la lagrangiana esprimendo le coordinate vincolate in funzione delle altre, per poi calcolare le equazioni del moto solo per le coordinate non vincolate. 
\section{Meccanica hamiltoniana}
Le equazioni di Eulero-Lagrange sono di secondo ordine nelle coordinate $\vec{q}$, il che rende complicate operazioni come i cambi di variabili una volta scritta la lagrangiana. La formulazione \emph{hamiltoniana} consente invece di ottenere le equazioni del moto da equazioni di primo ordine. La formulazione hamiltoniana è sempre basata sul principio di minima azione, ma usa una diversa funzione scalare, detta hamiltoniana, che è legata alla lagrangiana da un'applicazione nota come trasformata di Legendre.

\begin{definition}
  Una funzione $f:\R^n \to \R$ si dice \dfn{convessa} se l'hessiana $\pdv{f}{x_i}{x_j}()(x)$ è definita positiva $\forall x \in \R^{n}$.
\end{definition}

\begin{definition}
  Data una funzione convessa $f:\R^{n}\to \R$, sia \begin{equation}
  F(\vec{x},\vec{p}): \R^{n} \times (\R^{n})^* \to \R \qqtext{tale che} F(\vec{x},\vec{p}) = \vec{p}\vec{x} - f(x)
  \end{equation} 
  e sia $\vec{x}(\vec{p})$ la funzione che a un vettore duale $\vec{p} \in (\R^{n})^*$ associa il valore di $\vec{x}$ tale che $F(\vec{x},\vec{p})$ è massimo. Si dice \dfn{trasformata di Legendre} di $f$ la funzione $f^*:(\R^{n})^* \to \R$ tale che \begin{equation}
  f^*(\vec{p}) = F(\vec{x}(\vec{p}), \vec{p})
  \end{equation} 
\end{definition}
\begin{remark} \label{rem:legendreDist}
  La funzione $F$ non è altro che la distanza verticale tra il punto del grafico di $f$ con coordinate $\vec{x}$ e il grafico del duale $\vec{p}$ (che nel caso multidimensionale è un piano a $n-1$ dimensioni passante per l'origine). 
\end{remark}
\begin{remark} \label{rem:legendreSlope}
  A $\vec{p}$ fissato, $F$ è massima quando $\pdv{F}{\vec{x}}=0$, da cui segue che nel punto $\vec{x}(\vec{p})$ vale \begin{equation}
    \vec{p}=\pdv{f}{\vec{x}}()(\vec{x})
  \end{equation}
\end{remark}
\begin{remark}
  Nel caso in cui $f$ è una funzione differenziabile definita sull'asse reale, siccome il grafico di un vettore duale $\vec{p} \in  (\R)^*$ è una retta passante per l'origine, l'\autoref{rem:legendreSlope} significa che $x(p)$ è il punto in cui la tangente al grafico di $f$ è parallela al grafico di $p$. Insieme all'\autoref{rem:legendreDist}, ciò implica che in questo caso $f^*(p)$ è l'intercetta della retta con pendenza $p$ tangente al grafico di $f$.
\end{remark}
\begin{remark}
  La trasformata di Legendre è comunque definita anche per funzioni convesse ma non differenziabili.
\end{remark}

Possiamo a questo punto formulare le definizioni centrali di questo paragrafo.
\begin{definition}
  Si dice \dfn{hamiltoniana} $\mathcal{H}$ di un sistema con lagrangiana $\mathcal{L}$ espresso nelle coordinate $\vec{q}$ la trasformata di Legendre di $\mathcal{L}$ rispetto alle velocità generalizzate $\vec{\dot{q}}$. 
\end{definition}
\begin{remark}
  La condizione di convessità necessaria per la trasformata di Legendre è sempre soddisfatta per i sistemi meccanici conservativi, siccome l'energia cinetica è definita positiva e il potenziale non dipende dalle $\vec{\dot{q}}$.
\end{remark}

In meccanica hamiltoniana rivestono un'importanza fondamentale grandezze dette \emph{momenti}, che sono duali delle velocità e ne prendono il ruolo come variabili.
\begin{definition}
  Dato un sistema con lagrangiana $\mathcal{L}$ espresso nelle coordinate $\vec{q}=(q_1, \ldots, q_n)$, si dice \dfn{momento generalizzato} rispetto a $q_j$ la quantità \begin{equation}
  p_j(\dot{q}_j) = \pdv{\mathcal{L}}{\dot{q}_j}() (\dot{q}_j)
  \end{equation} 
\end{definition}
\begin{remark}
  La relazione $\vec{p}(\vec{q})$ può essere invertita per ottenere $\vec{q}(\vec{p})$.
\end{remark}

\begin{theorem}
  L'hamiltoniana di un sistema con lagrangiana $\mathcal{L}$ espresso nelle coordinate $\vec{q}=(q_1, \ldots, q_n)$ è data da \begin{equation}
  \mathcal{H} = \left[\sum_{j=1}^{n} p_j \dot{q}_j (p_j)\right] - \mathcal{L}\big(\vec{q}, \vec{\dot{q}}(\vec{p}), t\big)
  \end{equation} 
\end{theorem}

Come anticipato, le equazioni del moto possono essere ottenute dall'hamiltoniana risolvendo un sistema di equazioni di prim'ordine.
\begin{theorem}
  Il moto seguito da un sistema fisico soddisfa le equazioni \begin{equation}
  \vec{\dot{q}}(\vec{q},\vec{p},t) = \pdv{\mathcal{H}}{\vec{p}}()(\vec{q},\vec{p},t) \qquad \vec{\dot{p}}(\vec{q},\vec{p},t) = -\pdv{\mathcal{H}}{\vec{q}}()(\vec{q},\vec{p},t)
  \end{equation}
  Esse sono dette \dfn{equazioni di Hamilton}.
\end{theorem}

\begin{theorem}
  Se la lagrangiana ha la forma $\mathcal{L} = T - V$, l'hamiltoniana non è altro che \begin{equation}
  \mathcal{H}(\vec{q},\vec{p},t) = T\big(\vec{q},\vec{\dot{q}}(\vec{p})\big) + V(\vec{q},t)
  \end{equation} 
  ovvero $\mathcal{H} = E$, l'energia totale.
\end{theorem}

La conservazione dell'energia totale si generalizza alla conservazione di qualsiasi hamiltoniana:
\begin{theorem}
  Se l'hamiltoniana $\mathcal{H}$ non dipende esplicitamente dal tempo $t$, le equazioni di Hamilton conservano $\mathcal{H}$.
\end{theorem}

In meccanica lagrangiana si codifica lo stato del sistema nello spazio delle configurazioni, con la velocità data dall'evoluzione a partire dalle condizioni iniziali; in meccanica hamiltoniana è necessario usare un altro tipo di spazio, lo spazio delle fasi, che include anche i momenti generalizzati.
\begin{definition}
  Si dice \dfn{spazio delle fasi} $X$ lo spazio di tutte le possibili coppie di posizione e momento generalizzato $(\vec{q},\vec{p})$.
\end{definition}
\begin{remark}
  Se il sistema non è vincolato, le sue posizioni appartengono allo spazio delle configurazioni $\mathbb{M}$ isomorfo a un qualche $\R^{n}$, e dunque anche i momenti appartengono a $(\R^{n})^*$ isomorfo a $\R^{n}$. Lo spazio delle fasi è quindi isomorfo a $\R^{2n}$.
\end{remark}

\begin{definition}
  Per un sistema non vincolato, con hamiltoniana $\mathcal{H}$, si dice \dfn{campo vettoriale hamiltoniano} \begin{equation}
    \vec{V}_{\mathcal{H}}\,(\vec{z}) = -\mathsf{J}\; \big(\grad{\mathcal{H}} (z)\big)
  \end{equation}
  dove $\vec{z}=(\vec{q},\vec{p})$, $\grad$ rappresenta il gradiente totale sullo spazio delle fasi \begin{equation}
    \grad = \Bigg(\pdv{\vec{q}},\pdv{\vec{p}}\Bigg) 
    \end{equation}
    e $\mathsf{J}$ è una matrice a blocchi data da \begin{equation}
    \mathsf{J} = \bmqty{0 & -I \\ I & 0}
    \end{equation}
\end{definition}

In tal modo le equazioni di Hamilton prendono la forma \begin{equation}
\vec{\dot{z}} = -J \big(\grad{\mathcal{H}} (\vec{z})\big)
\end{equation} 
da cui risulta che
\begin{theorem}
  Il moto di un sistema non vincolato è la linea di flusso del campo vettoriale hamiltoniano che ha inizio nelle condizioni iniziali $(\vec{q}_0, \vec{p}_0)$.
\end{theorem} 

Siamo quasi riusciti a rimuovere le coordinate dalla meccanica di un sistema non vincolato: il comportamento delle linee di flusso ne è indipendente, così come il campo vettoriale. Esse sono rimaste soltanto nella definizione dell'hamiltoniana. Questa formulazione della meccanica è il punto da cui partiremo per liberarci completamente delle coordinate nella definizione dei sistemi meccanici. Tuttavia, nel caso di sistemi generici, che possono anche essere vincolati e quindi avere proprietà topologiche non banali, ci sarà necessario impostare la teoria delle varietà differenziabili, con la quale andremo a descrivere geometricamente gli spazi delle fasi. Questo sarà l'oggetto del prossimo capitolo.
  % \chapter{Varietà differenziabili}
La teoria delle varietà differenziabili si occupa di generalizzare le caratteristiche geometriche di $\R^n$ a spazi con topologie diverse. In questo capitolo si daranno per prima cosa le definizioni di varietà differenziabile e di applicazione differenziabile fra varietà. Si definiranno poi i vettori tangenti e cotangenti sulle varietà, mostrandone il collegamento alle operazioni di derivazione e definendo spazi tangenti e cotangenti. Infine ci si occuperà del potente strumento delle forme differenziali, che sarà il fondamento della formulazione simplettica della meccanica.

\section{Varietà e applicazioni differenziabili} \label{sec:smoothMfd}
Gli elementi di uno spazio vettoriale possono essere individuati tramite coordinate lineari. Una struttura di varietà differenziabile consente invece di individuare gli elementi di un insieme con coordinate non lineari, espandendo di molto i tipi di insiemi su cui è possibile definire coordinate. La non linearità complica però notevolmente l'individuazione dei punti, e rende necessario l'avere più di un sistema di coordinate. Sono inoltre posti alcuni requisiti per evitare casi patologici.

\begin{definition}
  Si dice \dfn{varietà differenziabile} una struttura $(X,\mathcal{U})$ dove $X$ è un insieme e $\mathcal{U}$, detto \dfn{atlante}, è una collezione di coppie $(U_{\nu}, \phi_{\nu})$ per $\nu \in S \subset \R$, dette \dfn{carte di coordinate}, dove gli $U_{\nu}$ sono sottoinsiemi di $X$ e le $\phi_{\nu}:U_{\nu}\to \R^{n}$ sono applicazioni biunivoche su un sottoinsieme aperto di $\R^{n}$ detto $\phi_{\nu}(U_{\nu})$, tali che:
  \begin{enumerate}
    \item le carte siano \dfn{compatibili}, cioè se $U_{\nu} \cap U_{\mu} \neq \emptyset$ allora $\phi_{\nu}(U_{\nu} \cap U_{\mu})$ e $\phi_{\mu}(U_{\nu} \cap U_{\mu})$ sono aperti nelle rispettive versioni di $\R^{n}$ e l'applicazione $\phi_{\mu} \circ \phi_{\nu}^{-1}: \phi_{\nu}(U_{\nu} \cap U_{\mu}) \to \phi_{\mu}(U_{\nu} \cap U_{\mu})$ è differenziabile fra questi due sottoinsiemi aperti di $\R^{n}$. $F_{\nu\mu} = \phi_{\mu} \circ \phi_{\nu}^{-1}$ è detta \dfn{mappa di transizione} fra le carte $\phi_{\nu}$ e $\phi_{\mu}$.
    \item esista un sottoinsieme numerabile $J \subset S$ tale che $X = \bigcup_{j \in  J} U_j$.
    \item Definendo \dfn{aperti in $X$} i sottoinsiemi di $X$ la cui immagine è aperta sotto tutte le carte nel cui dominio sono inclusi, $X$ sia uno spazio di Hausdorff, ossia per ogni $x,y \in X$ con $x \neq y$ esistano due insiemi aperti $U_x, U_y \subset X$ tali che $x \in U_x$ e $y \in  U_y$ mentre $U_x \cap U_y = \emptyset$.
  \end{enumerate}
   Dato che non saranno considerati altri tipi di varietà, l'aggettivo \emph{differenziabile} sarà omesso nel seguito. Spesso si indicherà inoltre la varietà $(X,\mathcal{U})$ semplicemente come $X$.
\end{definition}
\begin{remark}
  Si noti che, per come sono definite qui, le $\phi_{\nu}$ associano \emph{a un elemento della varietà una $n$-upla di reali}, e non viceversa. Siccome le $\phi_{\nu}$ sono biunivoche per definizione, tuttavia, è possibile riottenere i punti della varietà note le loro coordinate.
\end{remark}
\begin{remark}
  È possibile aggiungere nuove carte all'atlante $\mathcal{U}$ senza cambiare la struttura di varietà se queste nuove carte sono compatibili con le vecchie. Se un atlante non ammette nuove carte compatibili, esso è detto \dfn{massimale}. Spesso si assumerà che sia questo il caso.
\end{remark}
\begin{remark}
  Definendo gli aperti come al punto $3$, una varietà differenziabile ha una topologia naturale. In questa topologia, tutte le carte sono mappe continue (la retroimmagine di ogni aperto è aperta) per definizione. Viceversa, uno spazio topologico ha una struttura di varietà differenziabile, scegliendo gli $U_{\nu}$ fra gli aperti, se è possibile ricoprire $X$ di aperti e scegliere i $\phi_{\nu}$ in modo che siano omeomorfismi con $\R^{n}$.
\end{remark}
\begin{remark}
  La topologia naturale rende possibile determinare quando una varietà è connessa. Se $X$ è una varietà connessa, la dimensione $n$ di $\R^{n}$ codominio di una $\phi$ appartenente a una carta è lo stesso per ogni $\phi$. In tal caso, $n$ si dice \dfn{dimensione} di $X$. Nel seguito, saranno considerate solo varietà connesse.
\end{remark}

Data una carta $(U,\phi)$ e un punto $x \in U$, spesso per brevità si indicano le componenti $\big(\phi_1(x), \ldots, \phi_{n}(x)\big)$ con $(x_1, \ldots, x_n)$, identificando sostanzialmente $U$ con $\phi(U)$. Spesso si assume anche, senza perdita di generalità, che $\phi$ sia stata traslata in modo che per un qualche $x_0 \in  U$ si abbia $\phi(x_0) = 0$. 

Definito il teatro della teoria, restano da definire le azioni che in essa si possono compiere, vale a dire le applicazioni differenziabili da una varietà a un'altra.
\begin{definition}
  Un'applicazione $f:X_1 \to X_2$ tra varietà $X_1, X_2$ è detta \dfn{differenziabile} se per ogni $x \in  X_1$ esistono una carta $(U_1, \phi_1)$ su $X_1$ con $x \in U_1$ e una carta $(U_2, \phi_2)$ su $X_2$ tali che $f(U_1) \subset U_2$ e $\phi_2 \circ f \circ \phi_1^{-1}: \phi_1(U_1) \to \phi_2(U_2)$ sia una funzione differenziabile tra sottoinsiemi di $\R^{n}$. $f_{\phi_1 \phi_2} = \phi_2 \circ f \circ \phi_1^{-1}$ è detta \dfn{rappresentazione in coordinate} dell'applicazione $f$. $f$ è inoltre detta \dfn{diffeomorfismo} o \dfn{isomorfismo di varietà} se anche la sua inversa è differenziabile.
\end{definition}
\begin{remark}
  Se $f$ è differenziabile, allora $f$ è anche continua fra le topologie naturali delle due varietà. In tal caso infatti ogni sottoinsieme aperto di $X_2$ ha come retroimmagine un sottoinsieme aperto di $X_1$, dato che ogni punto $x$ di questa retroimmagine ammette un intorno aperto $U_1$.
\end{remark}

Nel caso, specifico ma significativo, in cui $f$ sia a valori reali, ovvero $X_2 = \R$, $X_2$ ha un'unica carta $(\R, \identity)$ (dove $\identity$ è l'identità) e la definizione diventa la seguente.
\begin{definition}
  Un'applicazione a valori reali definita su una varietà $f:X\to \R$ si dice \dfn{differenziabile} se per ogni $x \in X$ esiste una carta $(U, \phi)$ su $X$ con $x \in  U$ tale che $f\circ \phi^{-1}: \phi(U) \to \R$ funzione differenziabile tra sottoinsiemi di $\R^{n}$.
\end{definition}

Un modo naturale per produrre nuove varietà è prendere il prodotto cartesiano di varietà già definite. Si può infatti dimostrare il seguente teorema.
\begin{theorem}
  Se $X^1$ e $X^2$ sono varietà con struttura data dalle carte rispettivamente $\big\lbrace(U^1_{\mu}, \phi^1_{\mu})\big\rbrace_{\mu \in R}$ e $\big\lbrace (U^2_{\nu}, \phi^2_{\nu})\big\rbrace_{\nu \in S}$, il prodotto cartesiano $X^1 \times X^2$ ha una struttura naturale di varietà data dalle carte $\Big\lbrace\big( U^1_{\mu} \times U^2_{\nu}, (\phi^1_{\mu},\phi^2_{\nu})\big)\Big\rbrace$, dove $\mu \in  R,\ \nu \in S$ e \begin{equation*}
    \big((\phi^1_{\mu},\phi^2_{\nu})\big): (x^1,x^2) \mapsto \bigg(\big(\phi^1_{\mu}(x^1),\phi^2_{\nu}(x^2)\big)\bigg) \qqtext{per} x^1 \in U^1_{\mu},\ x^2 \in  U^2_{\nu}
  \end{equation*} 
\end{theorem}
\begin{remark}
  In sostanza, la varietà prodotto ha per punti le coppie di punti e per coordinate le coppie di coordinate delle due varietà fattore.
\end{remark}

\section{Vettori tangenti e cotangenti}
Per poter definire una forma di calcolo differenziale sulle varietà, è necessario formalizzare la nozione di spostamento infinitesimo, definendo i vettori tangenti.

\begin{definition}
  Si dice \dfn{vettore tangente} alla varietà $X$ nel punto $x \in  X$ la classe di equivalenza $[\gamma]$, dove $\gamma:]-\epsilon,\epsilon\;[\ \to X$, con $\epsilon > 0$, è un cammino differenziabile con $\gamma(0) = x$ e $\gamma_1$ equivale a $\gamma_2$ se esiste una carta $(U, \phi)$ tale che \begin{equation*}
  \eval{\dv{\phi}{t}\big(\gamma_1(t)\big)}_{t=0} = \eval{\dv{\phi}{t}\big(\gamma_2(t)\big)}_{t=0}
  \end{equation*} 
\end{definition}
\begin{remark}
  Intuitivamente, un vettore tangente in $x \in X$ è definito come un cammino infinitesimo che attraversa $x$ con una certa velocità. La relazione di equivalenza rimuove dalla considerazione il dominio di definizione e il comportamento nei punti diversi da $x$.
\end{remark}

Lo spazio tangente a $\R^{n}$ in un qualsiasi punto $y \in \R^{n}$, denotato $T_y \R^{n}$, può essere identificato con $\R^{n}$ facendo corrispondere a ogni $v \in \R^{n}$ la classe di equivalenza del cammino $\gamma_v(t) = y + tv$. Siccome qualsiasi cammino $\gamma_0:]-\epsilon,\epsilon\;[\ \to X$, con $\epsilon > 0$ e $\gamma_0(0)=y$, ammette nella sua classe di equivalenza (si ricordi che in $\R^{n}$ $\phi = \identity$) il cammino $\gamma(t) = y + t\dot{\gamma}_0(0)$ e quindi corrisponde a un $v_{\gamma} = \dot{\gamma}_0(0) \in \R^{n}$, la corrispondenza è suriettiva. Dato che è anche iniettiva per definizione, essa è biunivoca. 

Inoltre, una carta $(U, \phi)$ intorno a un punto $x \in  X$, con $X$ varietà generica, fornisce un modo di identificare $T_x X$ con $\R^{n}$, associando a ogni $v \in \R^{n}$ il percorso su $X$ $\gamma_{v}(t) = \phi^{-1}(\phi(x)+tv)$. Analogamente a sopra, questa corrispondenza è infatti biunivoca. Ogni spazio tangente può quindi essere identificato con $\R^{n}$, e quindi ogni spazio tangente è uno spazio vettoriale.

È possibile estendere il concetto di campo vettoriale alle varietà: intuitivamente, un campo vettoriale associa a ogni punto un vettore tangente in quel punto, in maniera differenziabile. Il tecnicismo principale sta nella definizione di differenziabilità per un'applicazione da una varietà a uno spazio tangente.
\begin{definition} \label{def:vecField}
  Si dice \dfn{campo vettoriale} su una varietà $X$ un'applicazione $V: x \in  X \mapsto V (x) \in T_x X$, tale che per ogni $x \in X$ e per ogni carta $(U, \phi)$ con $x \in U$ sia differenziabile la mappa $V_\phi:U \to \R^{n}$ risultante per ciascun $y \in  U$ dall'identificazione di $T_y X$ con $\R^{n}$ data da $\phi$.
\end{definition}

Le identificazioni locali consentono di scrivere localmente i campi vettoriali su varietà come campi vettoriali su $\R^{n}$, una volta fissata una carta. In un cambio di carta, le componenti dei vettori tangenti si trasformano in maniera ben definita.
\begin{theorem} 
  Sia $X$ una varietà e siano $(U,\phi)$ e $(V,\psi)$ due carte. Sia $x \in  U \cap V$ e sia $[\gamma] \in  T_x X$. Siano $u \in \R^{n}$ e $v \in  \R^{n}$ i corrispondenti di $[\gamma]$ secondo $\phi$ e $\psi$, rispettivamente. Allora, dette $u_j$ e $v_i$ le rispettive componenti, e detta $F^{\phi\psi}:\R^{n} \to \R^{n}$ la funzione di transizione, con componenti $F^{\phi \psi}_i$, vale \begin{equation} \label{eq:vecTrans}
  v_i = \sum_{j=1}^n \pdv{F^{\phi \psi}_i}{x_j} u_j
  \end{equation} 
\end{theorem}

I vettori tangenti consentono di individuare spostamenti infinitesimi su una varietà, e rendono quindi possibile la formulazione e la risoluzione di equazioni differenziali.

\begin{definition}
  Sia $X$ una varietà, un'applicazione $x:\R \to X$ si dice \dfn{linea di flusso} per un campo vettoriale $V$ su $X$ se soddisfa l'equazione \begin{equation*}
  \dot{x}(t) = V\big(x(t)\big)
  \end{equation*} 
  dove $\dot{x}$ è il vettore tangente in $x(t)$ definito da $[x]$.
\end{definition}
\begin{remark}
  Scelta una carta, questa equazione equivale a un'equazione di primo grado, la cui soluzione per una data condizione iniziale esiste ed è unica per via del teorema di Cauchy. Si dimostra quindi il seguente teorema.
\end{remark}
\begin{theorem}
  Sia $X$ una varietà compatta (cioè in cui ogni successione ammette una sottosuccessione convergente) e sia $V$ un campo vettoriale su $X$. Allora esistono un $\epsilon > 0$ e un'applicazione differenziabile $\Phi:]-\epsilon, \epsilon\;[\ \times X \to X$, detta \dfn{flusso del campo vettoriale}, tale che \begin{equation*}
  \dv{t} \Phi(t,x) = V\big( \Phi(x,t)\big) \quad \forall\, x \in  X
  \end{equation*} 
  Inoltre, la mappa $\Phi^t(x):X \to X$ definita da $\Phi^t:x \mapsto \Phi(t,x)$ è un diffeomorfismo di $X$ in se stesso.
\end{theorem}

Un'altra possibilità che i vettori tangenti consentono di espandere da $\R^{n}$ alle varietà in generale è la derivazione.
\begin{definition}
  Siano $X,Y$ varietà e sia $f:X\to Y$ un'applicazione differenziabile. Si dice \dfn{derivata} di $f$ in $x \in X$ l'applicazione lineare $D_x f:T_x X \to T_{f(x)}Y$ che porta il cammino $\gamma: ]-\epsilon, \epsilon\;[\ \to X$ con $\epsilon>0$ e $\gamma(0) \in X$ nel cammino $f \circ \gamma:]-\epsilon, \epsilon\;[\ \to Y$.
\end{definition}
\begin{remark}
  Intuitivamente, la derivata dell'applicazione $f$ in un punto $x \in X$ associa a ogni spostamento infinitesimo da $x$ lo spostamento infinitesimo da $f(x) \in Y$ che ne risulta.
\end{remark}
\begin{remark}
  Questa definizione di derivata generalizza la derivata direzionale. Se infatti $X = \R^{n}$ e $Y = \R^{m}$, a un vettore $v \in \R^{n}$ si può associare un percorso $x + tv$. La derivata manda questo percorso in $f(x+tv)$, a cui si può associare $w = \eval{\dv{t}(x+tv)}_{t=0}$ e quest'ultima è esattamente la definizione di derivata direzionale. Nel caso di $X,Y$ varietà generiche, ci si può ricondurre al caso $\R^{n} \to \R^{m}$ tramite le carte di coordinate intorno a $x \in X$ e $f(x)\in Y$, rispettivamente $(U, \phi)$ e $(V,\psi)$. In tal caso, la derivata è rappresentata dal jacobiano di $f_{\phi \psi}$, la rappresentazione in coordinate di $f$.
\end{remark}

Se $f:X \to \R$ è una funzione differenziabile a valori reali, per ogni $x \in X$ si ha che $D_x f: T_x X \to T_{f(x)}\R$, ma siccome $T_{f(x)}\R$ può essere identificato con $\R$ si può scrivere $D_x f: T_x X \to \R$. Ciò significa che la derivata di una funzione a valori reali appartiene allo spazio duale di quello tangente.

\begin{definition}
  Sia $X$ una varietà e sia $x \in X$. Si dice \dfn{spazio cotangente} a $X$ in $x$ $T_x^* X$ lo spazio duale di $T_x X$. I suoi elementi si dicono \dfn{vettori cotangenti}.
\end{definition}
\begin{definition}
  Si dice \dfn{1-forma} su una varietà $X$ un'applicazione $\alpha: x \in X \mapsto \alpha(x) \in T_x^* X$, differenziabile in senso analogo a quello usato per i campi vettoriali nella \autoref{def:vecField} (cioè tale che si ottenga un'applicazione differenziabile identificando gli spazi tangenti con $\R^{n}$ secondo le carte di coordinate).
\end{definition}

L'applicazione che a ogni punto associa la derivata di una funzione $f$ fissata in quel punto è quindi una 1-forma.
\begin{definition} \label{def:differential}
  Sia $X$ una varietà e sia $f:X\to \R$ un'applicazione differenziabile, si dice \dfn{differenziale} di $f$ la 1-forma \begin{equation}
  \dd f: x \in X \mapsto D_x f \in T_x^* X
  \end{equation} 
\end{definition}

Anche i vettori cotangenti hanno una rappresentazione in coordinate, data dalla base duale di quella scelta per lo spazio tangente, e le loro coordinate si trasformano in maniera ben definita in un cambio di carte. Questa trasformazione è però inversa rispetto a quella per i vettori tangenti.
\begin{theorem} 
  Sia $X$ una varietà e siano $(U,\phi)$ e $(V,\psi)$ due carte. Sia $x \in  U \cap V$ e sia $\alpha \in  T_x^* X$. Siano $u^* \in \R^{n}$ e $v^* \in  \R^{n}$ i corrispondenti di $\alpha$ secondo $\phi$ e $\psi$, rispettivamente. Allora, dette $u^*_i$ e $v^*_j$ le rispettive componenti, e detta $F^{\phi\psi}:\R^{n} \to \R^{n}$ la funzione di transizione, con componenti $F^{\phi \psi}_i$, vale \begin{equation} \label{eq:cvcTrans}
  u^*_i = \sum_{j=1}^n \pdv{F^{\phi \psi}_i}{x_j} v^*_j
  \end{equation} 
\end{theorem}

Inoltre, siccome la derivata di una funzione a valori reali definita su una varietà è un vettore cotangente alla varietà stessa, data una funzione $f:X \to \R$ è possibile associare a ogni vettore $\gamma \in T_x X$, con $x \in X$, un valore $\gamma\,(f) \defeq (D_x f) (\gamma) \in \R$. Con questa associazione, un campo vettoriale $V$ diventa una funzione differenziabile $V(f): X \to \R$. Si possono quindi identificare i vettori tangenti a una varietà in un punto con le derivate direzionali su funzioni reali in quel punto e i campi vettoriali con campi di derivate direzionali di funzioni reali. Sia $(\phi,U)$ una carta, se si identificano $U$ e $\phi(U)$ come indicato nella \autoref{sec:smoothMfd} è conveniente identificare i vettori $\gamma_j = \phi^{-1}(tx_j)$ con le derivate parziali $\pdv{x_j}$. Inoltre, con questa identificazione il differenziale $\dd \phi_j$ della funzione $\phi_j:X\to \R$ data dalla $j$-esima componente di $\phi$ si indica semplicemente $\dd x_j$. In questo modo, siccome i rispettivi spazi di appartenenza sono spazi vettoriali, campi vettoriali e $1$-forme si possono scrivere come \begin{equation*}
V(x) = \sum_{j=1}^{n} a_j(x) \pdv{x_j} \qquad \alpha(x) = \sum_{j=1}^{n} a^*_j(x) \dd{x_j}
\end{equation*} 
dove le $a_j$ e $a^*_j$ sono funzioni differenziabili da $\phi(U)$ a $\R$, mentre le equazioni \ref{eq:vecTrans} ed \ref{eq:cvcTrans} dei cambi di carta da $(U, \phi)$ a $(V, \psi)$, con $y_i \simeq \psi_i(y)$, diventano quelle note dall'analisi \begin{equation*}
\pdv{y_i} = \sum_{j=1}^{n} \pdv{x_j}{y_i} \pdv{x_j} \qquad \dd{y_i} = \sum_{j=1}^{n} \pdv{y_i}{x_j} \dd{x_j}
\end{equation*} 

È possibile costruire una nuova varietà a partire da una varietà e dagli spazi a essa tangenti o cotangenti. Questi tipi di varietà sono noti come fibrati.

\begin{definition} \label{def:tanBundle}
  Sia $X$ una varietà, si dice \dfn{fibrato tangente} $TX$ l'insieme delle coppie $(x,[\gamma])$ con $x \in X$ e $[\gamma] \in T_x X$, con l'atlante dato dalle carte $(U, \Phi)$ definite nel seguente modo. Sia $(U, \phi)$ una carta su $X$ e sia $T_U X$ l'insieme delle coppie $(x,[\gamma])$ con $x \in X$ e $[\gamma] \in T_x X$. Allora $\Phi:T_U X \to \phi(U) \times \R^{n}$ è definita come \begin{equation}
  \Phi \big(x,[\gamma]\big) = \big(\phi(x), v\big)
  \end{equation} 
  dove $v \in \R^{n}$ è il vettore reale corrispondente a $[\gamma]$ secondo la carta $\phi$.
\end{definition}
\begin{definition}
  Sia $X$ una varietà, si dice \dfn{fibrato cotangente} $T^*X$ l'insieme delle coppie $(x,\alpha)$ con $x \in X$ e $\alpha \in T^*_x X$, con l'atlante definito in maniera analoga a quanto fatto per il fibrato tangente nella \autoref{def:tanBundle}.
\end{definition}

Queste varietà vengono dette fibrati per analogia al caso della circonferenza, in cui ogni punto possiede una \textquotedblleft fibra\textquotedblright\ monodimensionale come spazio tangente. Da un elemento dello spazio tangente si può sempre rimuovere la fibra con l'applicazione seguente.
\begin{definition}
  Sia $X$ una varietà e $B=\{(x,b)\}$ un suo fibrato (tangente o cotangente). Si dice \dfn{proiezione} l'applicazione $\pi: B \to X$ definito da $\pi(x,b) = x$.
\end{definition}
  % \chapter{Meccanica simplettica}
Si possono ora rivisitare le formulazioni lagrangiana ed hamiltoniana esposte nel capitolo 1, avvalendosi degli strumenti sviluppati nel capitolo 2, per chiarire la natura dello spazio delle configurazioni, di quello delle fasi, e il legame tra le due formulazioni. Si definiranno poi i concetti fondamentali della \dfn{meccanica simplettica}, la formulazione della meccanica libera da coordinate. Si enuncerà infine il teorema di Darboux, il quale afferma che ogni varietà simplettica ammette coordinate in cui la struttura simplettica è quella canonica.

\section{Meccanica su varietà}
Nel capitolo 1 si è definito lo spazio delle configurazioni $\mathbb{M}$ di un sistema non vincolato di $N$ particelle come il prodotto di $N$ copie di $\R^{3}$. Questo spazio è identificabile con $\R^{3N}$. Nel seguito, per brevità, porremo $n = 3N$. Se sulle particelle viene imposto un vincolo, lo spazio delle configurazioni del sistema diventa un qualche insieme $M \subset \R^{3N}$. Come già affermato, questo insieme non avrà in generale una struttura di spazio vettoriale. I suoi punti $x$ potranno però essere individuati da $n$-uple $\vec{q}$, dette coordinate, tramite funzioni biunivoche $\phi^{-1}: \vec{q} \mapsto x$. $M$ avrà quindi la struttura più generale di una \emph{varietà differenziabile}, con atlante dato dalle funzioni $\phi: x \mapsto \vec{q}$ e dai loro domini di biunivocità. Si dice che lo spazio delle configurazioni di un sistema di $N$ particelle con $n-k$ vincoli è una varietà $k$-dimensionale \dfn{immersa} in $\R^{n}$, lo spazio $3N$-dimensionale delle configurazioni senza vincoli.

Per individuare univocamente un generico sistema sono necessari due elementi: il suo spazio delle configurazioni $M$ e l'energia potenziale, definita proprio sullo spazio delle configurazioni, $V: x \in M \mapsto V(x) \in \R$.  

% La velocità $\dot{x}$ di un sistema è interpretabile come un vettore tangente allo spazio delle configurazioni nel punto $x(t)$, facendo corrispondere a ogni $\gamma \in T_x M$ il vettore $\dot{x} = \dot{\gamma}(0)$, che appartiene a $\R^{n}$ poiché $\gamma: ]-\epsilon,\epsilon[ \to M \subset \R^{n}$ ed è tangente nel senso ordinario (ortogonale al gradiente dei vincoli) alla varietà immersa $M$. Su ogni spazio tangente è quindi definita l'energia cinetica come forma quadratica $K_x: \dot{x} \mapsto \frac{1}{2} \dot{x}^2$.

La velocità $\dot{x}$ di un sistema è definita come derivata temporale $\dv{\overline{x}}{t}()(t)$ del suo moto $\overline{x}: t \in I \mapsto \overline{x}(t) \in M$, dove $I$ è un intervallo reale. Per definizione, un vettore tangente alla varietà $M$ nel punto $x \in M$ è una classe di equivalenza di cammini infinitesimi che passano da $x$ al tempo $0$. Per un dato $x =\overline{x}(t)$, si può traslare il tempo e ridurne il dominio in modo da ottenere un nuovo cammino $\gamma:]-\epsilon, \epsilon\,[$ tale che $x = \gamma(0)$. Questo cammino sarà inoltre tale che $\dot{x} = \dv{\gamma}{t}()(0)$. Siccome $M \subset \R^{n}$, una scelta valida per $\phi$ è l'identità $\identity$. Quindi, perché un secondo cammino $\gamma'$ sia equivalente a $\gamma$, è necessario e sufficiente che \begin{equation*}
\eval{\dv{t}}_{t=0} \gamma'(t) = \eval{\dv{t}}_{t=0} \gamma(t) = \dot{x}
\end{equation*}
per definizione: la velocità $\dot{x}$ caratterizza la classe $[\gamma]$ dello spazio tangente a $M$ in $x$. Questo potrà quindi essere identificato con lo spazio delle velocità possibili quando il sistema ha configurazione $x$. Si può quindi dire che la velocità di un sistema è un vettore tangente allo spazio delle configurazioni nella configurazione attuale del sistema. L'energia cinetica, funzione della velocità, è quindi definita per ciascun $x \in M$ su $T_x M$ da $K_x: \dot{x} \mapsto \frac{1}{2} \dot{x}^2$, dove la norma è la norma in $\R^{n}$.

La lagrangiana di un sistema dipende sia dall'energia potenziale che dall'energia cinetica, e dunque sia dalla configurazione del sistema che dalla sua velocità. Essa deve quindi essere definita su $TM$, il fibrato tangente di $M$. Le energie cinetica e potenziale dovranno quindi cambiare dominio da rispettivamente $T_x M$ e $M$ a $TM$ per poter formare la lagrangiana. Sia $(x,\dot{x}) \in TM$, con $x \in M$ e $\dot{x} \in T_x M$ ciò si può fare definendo \begin{equation*}
\begin{aligned}
  &V(x,\dot{x})|_{TM} = V(x)|_M \\
  &K(x, \dot{x})|_{TM} = K(\dot{x})|_{T_x M}
\end{aligned}
\end{equation*} 
Si può a questo punto definire la lagrangiana come una funzione differenziabile $\mathcal{L}:TM\to \R$ data per $x \in M$ e $\dot{x} \in T_x M$ da \begin{equation*}
\mathcal{L}(x,\dot{x}) = K(\dot{x}) - V(x)
\end{equation*}

La trasformata di Legendre rispetto alle velocità trasforma la lagrangiana $L(x,\dot{x})$ nell'hamiltoniana $\mathcal{H}(x,\dot{x}^*)$ dove gli $\dot{x}^*$ sono elementi dello spazio duale a quello degli $\dot{x}$. Ma poiché $\dot{x} \in T_x M$, ciò significa che $\dot{x}^* \in T_x^* M$, e quindi l'hamiltoniana di un sistema è definita sul fibrato cotangente del suo spazio delle configurazioni. Il momento generalizzato $\dot{x}^*$ è infatti un covettore, dato da \begin{equation*}
  \dot{x}^* = \dd_{\dot{x}} \mathcal{L}
\end{equation*}
Lo spazio delle fasi può quindi essere definito in maniera indipendente dalle coordinate come il fibrato cotangente $T^* M$ dello spazio delle configurazioni. Se una regione di $M$ è coperta dalla carta $(U,\phi)$, un punto $x \in U$ si può individuare con $\vec{q} = (q_1, \ldots, q_n) = (\phi_1(x), \ldots, \phi_n(x))$ e il momento generalizzato cotangente a $M$ in $x$ si può individuare con $\vec{p}=(p_1, \ldots, p_n)$, le coordinate nella base duale data da $\{\dd_x \phi_1, \ldots, \dd_x \phi_n\} $. Si recupera così la definizione basata sulle coordinate.

\section{Forma simplettica canonica}
% \symplecticForm*

\begin{restatable}{definition}{symplecticForm}
  Si dice \dfn{forma simplettica} una $2$-forma differenziale $\omega$ chiusa e non degenere. Si dice \dfn{varietà simplettica} una varietà $2n$-dimensionale su cui è definita una forma simplettica.
\end{restatable} % PLACEHOLDER FOR INDEPENDENT COMPILATION, REMOVE FOR FINAL VERSION
  \chapter{Meccanica simplettica}
Questo capitolo costituisce il punto di arrivo dell'esposizione. Si esamineranno le formulazioni della meccanica esposte nel capitolo 1 dal punto di vista della teoria delle varietà differenziabili, e si ricaverà dalla meccanica hamiltoniana l'agognata \emph{formulazione simplettica} della meccanica, la quale non fa uso di coordinate. Uno dei più potenti strumenti a disposizione di questa formulazione è la riduzione simplettica trattata nel capitolo 3, la quale qui verrà applicata allo studio del moto del corpo rigido.

\section{Meccanica e varietà}
Nel capitolo 1 si è definito lo spazio delle configurazioni $\mathbb{M}$ di un sistema non vincolato di $N$ particelle come il prodotto di $N$ copie di $\R^{3}$. Questo spazio è identificabile con $\R^{3N}$. Se sulle particelle viene imposto un vincolo, lo spazio delle configurazioni del sistema diventa un qualche insieme $M \subset \R^{3N}$. Come già affermato, questo insieme non avrà in generale una struttura di spazio vettoriale. I suoi punti $x$ potranno però essere individuati da $n$-uple $\vec{q}$, dette \emph{coordinate}, tramite funzioni biunivoche $\phi^{-1}: \vec{q} \mapsto x$. $M$ avrà quindi la struttura più generale di una \emph{varietà differenziabile}, con atlante dato dalle funzioni $\phi: x \mapsto \vec{q}$ e dai loro domini di biunivocità. Come menzionato nel capitolo 1, lo spazio delle configurazioni di un sistema di $N$ particelle con $3N-n$ vincoli è quindi una varietà $n$-dimensionale \dfn{immersa} in $\R^{3N}$, lo spazio $3N$-dimensionale delle configurazioni senza vincoli. 

Per individuare univocamente un generico sistema sono necessari due elementi: il suo spazio delle configurazioni $M$ e l'energia potenziale, definita proprio sullo spazio delle configurazioni, $V: x \in M \mapsto V(x) \in \R$.  

% La velocità $\dot{x}$ di un sistema è interpretabile come un vettore tangente allo spazio delle configurazioni nel punto $x(t)$, facendo corrispondere a ogni $\xi = [\gamma] \in T_x M$ il vettore $\dot{x} = \dot{\gamma}(0)$, che appartiene a $\R^{n}$ poiché $\gamma: ]-\epsilon,\epsilon[ \to M \subset \R^{n}$ ed è tangente nel senso ordinario (ortogonale al gradiente dei vincoli) alla varietà immersa $M$. Su ogni spazio tangente è quindi definita l'energia cinetica come forma quadratica $K_x: \dot{x} \mapsto \frac{1}{2} \dot{x}^2$.

La velocità $\dot{x}$ di un sistema è definita come derivata temporale $\dv{\overline{x}}{t}()(t)$ del suo moto $\overline{x}: t \in I \mapsto \overline{x}(t) \in M$, dove $I$ è un intervallo reale. Per definizione, un vettore tangente alla varietà $M$ nel punto $x \in M$ è una classe di equivalenza di cammini infinitesimi che passano da $x$ al tempo $0$. Per un dato $x =\overline{x}(t)$, si può traslare il tempo e ridurne il dominio in modo da ottenere un nuovo cammino $\gamma:]-\epsilon, \epsilon\,[$ tale che $x = \gamma(0)$. Questo cammino sarà inoltre tale che $\dot{x} = \dv{\gamma}{t}()(0)$. Siccome $M \subset \R^{n}$, una scelta valida per $\phi$ è l'identità $\identity$. Quindi, perché un secondo cammino $\gamma'$ sia equivalente a $\gamma$, è necessario e sufficiente che \begin{equation}
\eval{\dv{t}}_{t=0} \gamma'(t) = \eval{\dv{t}}_{t=0} \gamma(t) = \dot{x}
\end{equation}
per definizione: la velocità $\dot{x}$ caratterizza la classe $\xi = [\gamma]$ dello spazio tangente a $M$ in $x$. Questo potrà quindi essere identificato con lo spazio delle velocità possibili quando il sistema ha configurazione $x$. Si può quindi dire che la velocità di un sistema è un vettore tangente allo spazio delle configurazioni nella configurazione attuale del sistema. L'energia cinetica, funzione della velocità, è quindi definita per ciascun $x \in M$ su $T_x M$ da $K_x: \dot{x} \mapsto \frac{1}{2} \dot{x}^2$, dove la norma tiene conto delle masse delle particelle che costituiscono il sistema.

La lagrangiana ridotta di un sistema dipende sia dall'energia potenziale che dall'energia cinetica, e dunque sia dalla configurazione del sistema che dalla sua velocità. Essa deve quindi essere definita su $TM$, il fibrato tangente di $M$. Le energie cinetica e potenziale dovranno quindi cambiare dominio da rispettivamente $T_x M$ e $M$ a $TM$ per poter formare la lagrangiana. Sia $(x,\dot{x}) \in TM$, con $x \in M$ e $\dot{x} \in T_x M$ ciò si può fare definendo \begin{equation}
\begin{aligned}
  &V(x,\dot{x})|_{TM} \defeq V(x)|_M \\
  &K(x, \dot{x})|_{TM} \defeq K(\dot{x})|_{T_x M}
\end{aligned}
\end{equation} 
Si può a questo punto definire la lagrangiana (ridotta) di un sistema indipendente dal tempo come una funzione differenziabile $\mathcal{L}:TM\to \R$ data per $x \in M$ e $\dot{x} \in T_x M$ da \begin{equation}
\mathcal{L}(x,\dot{x}) \defeq K(\dot{x}) - V(x)
\end{equation}

In coordinate, la trasformata di Legendre rispetto alle velocità trasforma la lagrangiana $\mathcal{L}(x,\dot{x})$ nell'hamiltoniana $\mathcal{H}(x,\dot{x}^*)$ dove gli $\dot{x}^*$ sono elementi dello spazio duale a quello degli $\dot{x}$. Siccome $\dot{x} \in T_x M$, è quindi naturale imporre che l'hamiltoniana $\mathcal{H}$ accetti come argomenti $\dot{x}^* \in T_x^* M$, ovvero che sia definita sul fibrato cotangente $T^*M$. Lo spazio delle fasi, dominio dell'hamiltoniana, può quindi essere definito come il fibrato cotangente dello spazio delle configurazioni. Se una regione di $M$ è coperta dalla carta $(U,\phi)$, un punto $x \in U$ si può individuare con $\vec{q} = (q_1, \ldots, q_n) = \big(\phi_1(x), \ldots, \phi_n(x)\big)$ e il momento generalizzato cotangente a $M$ in $x$ si può individuare con $\vec{p}=(p_1, \ldots, p_n)$, nella base duale data da $\{\dd_x \phi_1, \ldots, \dd_x \phi_n\} $. Si recupera così la definizione basata sulle coordinate.

Essendo un fibrato cotangente, lo spazio delle fasi ha una struttura simplettica canonica ed è quindi possibile sfruttare la \autoref{def:hamField} di campo vettoriale hamiltoniano per estendere la \autoref{eq:hamFieldUnconstr} al caso di varietà che non costituiscono spazi vettoriali. Si consideri infatti una varietà simplettica $2n$-dimensionale $X$. In una carta simplettica data dal teorema di Darboux $(U,\phi)$ siano i suoi punti $x \in U$ individuati dalle coordinate $(x_1, \ldots, x_{2n})$ e i vettori $\xi$ tangenti a $X$ in un punto $y \in U$ individuati dalle coordinate $(\xi_1, \ldots, \xi_{2n})$. Allora il differenziale $\dd{\mathcal{H}}$ è rappresentato dalla trasposta del gradiente su queste coordinate $(\grad{\mathcal{H}})^T$, così che \begin{equation}
\dd{\mathcal{H}}(\xi) = (\grad{\mathcal{H}})^T \begin{pmatrix} \xi_1\\ \vdots\\ \xi_{2n} \end{pmatrix}
\end{equation} 
Allo stesso tempo, la forma simplettica $\omega$ sarà rappresentata da $\mathsf{J} \defeq \left( \begin{smallmatrix}
  0 & -\mathbb{1} \\ \mathbb{1} & 0
\end{smallmatrix}  \right) $ e il campo $X_{\mathcal{H}}$ da un vettore $\mathsf{X_{\mathcal{H}}}$, così che
\begin{equation}
\dd{\mathcal{H}}(\xi) = \omega(X_{\mathcal{H}}, \xi) = \mathsf{X_{\mathcal{H}}}^T\, \mathsf{J}\, \begin{pmatrix} \xi_1\\ \vdots\\ \xi_{2n} \end{pmatrix}
\end{equation} 
Da ciò segue che $\mathsf{X_{\mathcal{H}}}^T \mathsf{J} = (\grad{\mathcal{H}})^T$. Trasponendo e moltiplicando per $\mathsf{J}$, siccome $\mathsf{J}^2 =- \mathbb{1}$, si ha
\begin{equation}
\mathsf{X}_{\mathcal{H}} = -\mathsf{J} \grad{\mathcal{H}} 
\end{equation}
ovvero che per le coordinate vale \begin{equation}
\dot{q}_j = \pdv{\mathcal{H}}{p_j} \qquad \dot{p}_j = - \pdv{\mathcal{H}}{q_j}
\end{equation}
che sono le equazioni di Hamilton. Le linee di flusso del campo vettoriale hamiltoniano minimizzano quindi l'azione associata alla lagrangiana ridotta sullo spazio delle configurazioni vincolato. Esse minimizzano quindi l'azione su tutto lo spazio delle configurazioni, e dunque rappresentano le posizioni e le velocità che il sistema ha nel suo moto. Con quest'ultimo tassello è finalmente possibile porre una formulazione della meccanica di sistemi conservativi con vincoli olonomi arbitrari puramente geometrica e libera da coordinate.

Un sistema fisico è definito dal suo \emph{spazio delle configurazioni}, una varietà differenziabile $n$-dimensionale $M$ in cui ogni punto $m$ corrisponde a una diversa configurazione delle componenti del sistema, e da una funzione \emph{hamiltoniana} $\mathcal{H}: M\to \R$, che ne governa il movimento. Lo \emph{stato} di un sistema è rappresentato da un punto nello \emph{spazio delle fasi}, il fibrato cotangente $X = T^* M$ dello spazio delle configurazioni. Sullo spazio delle fasi è definita la \emph{forma simplettica canonica} $\omega$. Il \emph{campo vettoriale hamiltoniano} $X_{\mathcal{H}}$ è definito come il campo corrispondente al differenziale dell'hamiltoniana $\dd_{x}{\mathcal{H}}$ secondo la forma simplettica. Se il sistema parte da uno stato iniziale $(x_0, \dot{x}^*_0)$, il suo stato percorre la linea di flusso di $X_{\mathcal{H}}$ che parte da $(x_0, \dot{x}^*_0)$. L'evoluzione di una qualsiasi funzione di stato $F$ è data dalla parentesi di Poisson $\{F, \mathcal{H}\}$. In particolare, il valore di $\mathcal{H}$ è conservato durante l'evoluzione del sistema. Questa conserva inoltre il \emph{volume simplettico}, ovvero l'integrale di $\omega^{\wedge n}$ sull'immagine di un $U \subset X$ attraverso il flusso hamiltoniano.

Se lo spazio delle fasi ammette un'azione hamiltoniana $\Phi_g$ di un qualche gruppo di Lie $G$ e $\mathcal{H}$ è invariante sotto questa azione, la mappa momento avrà il valore $P_0$ fissato dalle condizioni iniziali lungo tutta l'evoluzione del sistema, e la dinamica di questo nello spazio delle fasi ridotto $\mu^{-1}(P_0)/G_{P_0}$ potrà essere studiata grazie al teorema di riduzione simplettica. 

\section{Esempio: moto del corpo rigido libero}
Si consideri un sistema costituito da un corpo rigido libero. Per definire questo sistema nella formulazione simplettica, è necessario individuare il suo spazio delle configurazioni e la sua hamiltoniana. Un \dfn{corpo rigido} è definito come una collezione di punti materiali $P$ le cui distanze relative sono tutte fissate. La definizione di punti e distanze equivale alla definizione della forma del corpo. Note le distanze, la posizione di ciascun punto del corpo in un dato sistema di riferimento può essere ottenuta conoscendo la posizione di un suo punto e il modo in cui è ruotato il corpo. La configurazione del corpo è quindi rappresentata da un punto nella varietà $M = \R^3 \times SO(3)$, che costituisce lo spazio delle configurazioni. Si noti che in questa prima fase $SO(3)$, il gruppo di Lie delle rotazioni in tre dimensioni, sarà considerato solo in quanto varietà differenziabile: le sue proprietà di gruppo saranno completamente irrilevanti. Per quanto riguarda l'hamiltoniana $\mathcal{H}$, siccome supponiamo che il corpo sia libero essa è data semplicemente dall'energia cinetica (per la precisione, dalla \emph{trasformata di Legendre} dell'energia cinetica, in modo da essere definita su $T^*M$ e non su $TM$). 

Lo spazio delle fasi del sistema è a rigore $T^*M = T^*\big(\R^3 \times SO(3)\big)$. Esso è tuttavia isomorfo a $T^*\R^3 \times T^*SO(3)$, la cui trattazione è notevolmente più semplice per via della possibilità di sfruttare le proprietà di identificazione tra spazi tangenti e cotangenti di $\R^3$. Si userà quindi come spazio delle fasi $X \defeq T^*\R^3 \times T^*SO(3)$, lasciando implicito il ritorno al fibrato cotangente di $M$ propriamente detto.

Si consideri il gruppo di Lie delle \dfn{traslazioni} $\R^3$. Questa volta a essere rilevante è la natura di gruppo. Esso ha un'azione naturale su $\R^3$ inteso come \emph{varietà}, data per $\vec{a} \in \R^3\text{-gruppo}$ e $\vec{q} \in \R^3\text{-varietà}$ da
\begin{equation}
\Phi_{\vec{a}}^{\R^3}: \vec{q} \mapsto  \vec{a} + \vec{q}
\end{equation}
dove il $+$ denota l'operazione di addizione definita in $\R^3$ inteso come spazio vettoriale. Si definisca la sua azione su $M$ come 
\begin{equation}
  \Phi_{\vec{a}}^M: (\vec{q},\rho) \longmapsto (\vec{a} + \vec{q},\rho)
\end{equation}
dove $\vec{q} \in \R^3, \rho \in SO(3), \vec{\dot{q}}^* \in T^*_{\vec{q}}\R^3$ e $\dot{\rho}^* \in T^*_{\rho}SO(3)$. Questa azione è ben definita su $M$ e per il teorema di Noether genera quindi un'azione hamiltoniana sul fibrato cotangente propriamente detto. Questa azione corrisponde a un'azione $\Phi^X_a$ su $X$, definita sfruttando la fattorizzazione come segue. Sia $(\vec{q},\vec{\dot{q}}^*, \rho, \dot{\rho}^*) \in X$. Ciò significa innanzitutto che $\vec{\dot{q}}^* \in T^*_{\vec{q}}\R^3$. Ma siccome $T^*_{\vec{q}}\R^3 \simeq \R^3 \simeq T^*_{\vec{q}+\vec{a}}\R^3$, si può anche considerare che $\vec{\dot{q}}^* \in T^*_{\vec{q}+\vec{a}}\R^3$. D'altra parte, anche $\dot{\rho}^* \in  T^*_{\rho}SO(3)$. Da ciò segue che $(\vec{q} +\vec{a},\vec{\dot{q}}^*, \rho, \dot{\rho}^*) \in X$, siccome i covettori appartengono agli spazi cotangenti appropriati. È quindi possibile definire l'azione su $X$ senza pull-back dei covettori, che invece è necessario nel caso dell'azione sul fibrato cotangente propriamente detto, come 
\begin{equation}
\Phi^X_{\vec{a}}: (\vec{q},\vec{\dot{q}}^*, \rho, \dot{\rho}^*) \longmapsto (\vec{q}+ \vec{a},\vec{\dot{q}}^*, \rho, \dot{\rho}^*) 
\end{equation}  

Come ci si può aspettare, anche $\Phi^X$ è hamiltoniana, con mappa momento data da
\begin{equation}
\mu^{\R^3}: (\vec{q}, \vec{\dot{q}}^*, \rho, \dot{\rho}^*) \longmapsto\; \vec{\dot{q}}^*
\end{equation} 
dove l'immagine $\vec{\dot{q}}^*$ è considerata come elemento di $T^*_{\vec{q}}\R^3 \simeq \R^3 \simeq T^*_{\vec{0}}\R^3 = \big(\mathfrak{r}^{3}\big)^*$, il duale dell'algebra di Lie di $\R^3$-gruppo. Si noti che, come anticipato, questa quantità corrisponde fisicamente al momento lineare. Nel seguito, per semplicità, si scriverà $\mu \defeq \mu^{\R^3}$

Bisogna ora dimostrare che $\mu^{\R^3}$ è effettivamente una mappa momento. L'azione infinitesima di $\vec{e}_i$, l'$i$-esimo elemento della base canonica di $\R^3 \simeq T_e \R^3 = \mathfrak{r}^3$, su $X$ è data dal campo vettoriale costante
\begin{equation}
\phi_{\vec{e}_i}^X: (\vec{q},\vec{\dot{q}}^*, \rho, \dot{\rho}^*)\longmapsto (\vec{e}_i, 0, 0, 0)
\end{equation}
dove nell'immagine $\vec{e}_i$ è inteso come elemento della base canonica di $\R^3 \simeq T_{\vec{q}}\R^3$. Infatti,l'azione infinitesima di $\mathfrak{r}^3$ su $\R^3$ in $\vec{x} \in \R^3$ è \begin{equation}
  \Big(\phi^{\R^3}_{\vec{e}_i}\Big)\vec{_x} = \Phi^{\R^3}_{\vec{0} + \vec{e}_i t}(x) = \vec{x} + \vec{e}_i t \simeq \vec{e}_i
\end{equation}
siccome $\R^3 \ni \vec{e}_i \simeq (\vec{0}+\vec{e}_i t) \in \vec{T_0} \R^3 $ e l'azione di $\R^3$ sulle altre componenti è banale. Allo stesso tempo, il valore della mappa comomento associata a $\vec{e}_i$ in ciascun punto è 
\begin{equation}
  P_{\vec{e}_i}(\vec{q},\vec{\dot{q}}^*, \rho, \dot{\rho}^*) =\ \vec{\dot{q}}^*\! \vec{e}_i = \dot{q}^*_i
\end{equation} 
dove nel membro di mezzo $\vec{\dot{q}}^*$ è considerato come elemento di $T^*_{\vec{q}}\R^3 \simeq \R^3 \simeq T^*_{\vec{0}}\R^3 = \mathfrak{r}^{3*}$ ed $\vec{e}_i$ come elemento di $\R^3 \simeq T_{\vec{0}}\R^3 = \mathfrak{r}^3$. Nella base simplettica di $T^*\big(\R^3 \times  SO(3)\big)$, data semplicemente dalla concatenazione delle basi simplettiche di $\R^3$ e $SO(3)$, siccome la mappa comomento è costante sulle componenti derivanti da $T^*SO(3)$ il campo hamiltoniano della mappa comomento per un dato $\vec{e}_i$ è fornito da \begin{equation}
V_{P_{\vec{e}_i}} = - \mathsf{J}\, \grad{P_{\vec{e}_i}} = - \begin{pmatrix}
  0 & 0 & \mathbb{1} & 0 \\
  0 & 0 & 0 & \mathbb{1} \\
  -\mathbb{1} & 0 & 0 & 0 \\ 
  0 & -\mathbb{1} & 0 & 0 
\end{pmatrix} \begin{pmatrix}
0 \\ 0 \\ \vec{e}_i \\ 0
\end{pmatrix} = (\vec{e}_i, 0, 0, 0)
\end{equation} 
ovvero esattamente il campo dell'azione infinitesima. Quanto all'equivarianza, siccome $\R^{n}$ è commutativo l'azione aggiunta $\mathrm{Ad}$ è banale, e quindi lo è anche l'azione coaggiunta. L'equivarianza è soddisfatta poiché in effetti $\mu$ non dipende dalla configurazione in $\R^3$, così che $\mu(\vec{q}+\vec{a}) = \mu(\vec{q}) = \mathrm{Ad}^*\,\big(\mu(\vec{q})\big)$. Quindi $\mu$ è una mappa momento e $\Phi^X$ è un'azione hamiltoniana.

Siccome $\R^{n}$ è commutativo e l'azione coaggiunta è banale, si ha anche che $G_{\vec{p}} = G$ per ogni $\vec{p} \in \mathfrak{r}^3$, e siccome per ogni coppia di elementi $\vec{a}, \vec{b} \in \R^3$ esiste un terzo elemento $\vec{c}$ tale che $\vec{a} = \vec{c} + \vec{b}$, tutti gli elementi di $\R^3$ sono equivalenti sotto l'azione di $\R^3$. Da queste osservazioni e dal fatto che per un dato $\vec{p} \in \R^3 \simeq T^*_{\vec{0}} \R^3 = \mathfrak{r}^3$ si ha che
\begin{equation}
\mu^{-1}(\vec{p}) = \big\lbrace(\vec{q},\vec{p}, \rho, \dot{\rho}^*) \mid \vec{q}\in \R^3, \rho \in SO(3), \dot{\rho}^* \in T^*_{\rho} SO(3)\big\rbrace \simeq \R^3 \times T^*SO(3)
\end{equation} 
segue che che $\mu^{-1}(p)/\R^3 \simeq T^*SO(3)$. Per il teorema di Marsden-Weinstein, $T^* SO(3)$ ha una struttura simplettica data dalla forma simplettica ridotta, mentre l'evoluzione del sistema è governata dall'hamiltoniana su $X$, ristretta a $T^* SO(3)$. Si può dimostrare che la forma simplettica ridotta è quella canonica, mentre l'hamiltoniana differisce da quella su $X$ per una costante. Ciò significa che il moto di rotazione di un corpo rigido libero che trasla condivide le proprietà fisiche di quello dello stesso corpo che ruota senza traslare.

Si ha quindi una sorta di spazio delle configurazioni ridotto $\pi\big(T^*SO(3)\big) = SO(3)$. In modo simile a quanto fatto per $\R^3$, si faccia agire $SO(3)$-gruppo su $SO(3)$-varietà secondo la moltiplicazione a sinistra:
\begin{equation}
L_{\sigma}: \rho \longmapsto \sigma \rho 
\end{equation} 
Il moto del sistema in $SO(3)$ sarà un'applicazione $\overline{\rho}: \R \to SO(3)$. La velocità generalizzata in un punto $\rho$ della varietà sarà naturalmente un vettore $\dot{\rho}(t) \in T_{\overline{\rho}(t)}SO(3)$. Tuttavia, si considera solitamente una quantità equivalente alla velocità, la \dfn{velocità angolare} $\psi$ definita dal push-forward di $\dot{\rho}$ nello spazio tangente all'identità $T_e SO(3) = \mathfrak{so(3)}$:
\begin{equation}
\psi \defeq \big(D L_{\rho^{-1}}\big)\big(\dot{\rho}\big)
\end{equation}
Analogamente, si definisce \dfn{momento angolare} (del corpo rigido) la quantità, equivalente al momento generalizzato $\dot{\rho}^*(t) \in T^*_{\overline{\rho}(t)}SO(3)$, definita dal pull-back di $\dot{\rho}^*$ nello spazio cotangente all'identità $T^*_e SO(3) = \mathfrak{so(3)}^*$
\begin{equation}
\lambda \defeq \big(R^*_{\rho} \big)\big(\dot{\rho}^*\big)
\end{equation} 
dove $R_\rho$ indica la moltiplicazione a destra: $R_\rho: \sigma \mapsto \sigma \rho$. Siccome l'algebra di Lie di $SO(3)$ è tridimensionale, sia la velocità angolare che il momento angolare possono essere identificati con vettori di $\R^3 \simeq (\R^3)^*$, e siccome l'azione di $SO(3)$ su $\mathfrak{so(3)}$ è la stessa che su $\R^3$ essi vengono trasformati da una rotazione del corpo rigido nella stessa maniera in cui lo sono i vettori posizione delle particelle che compongono il corpo stesso. Per questo motivo, velocità e momento angolari sono trattati come vettori in meccanica newtoniana; la loro natura nella formulazione simplettica è tuttavia appunto quella di elementi di algebra e coalgebra di Lie di $SO(3)$.

Dato che si sta considerando il corpo rigido libero, l'hamiltoniana coinciderà sostanzialmente con l'energia cinetica. Per sua natura, questa dovrà essere definita tramite una norma sulle velocità: l'introduzione della velocità angolare consente di definire questa norma solo su $\mathfrak{so(3)}$ invece che su ciascun $T_{\rho}SO(3)$. La norma viene solitamente introdotta tramite un prodotto scalare definito positivo $\langle \cdot, \cdot\rangle: \mathfrak{so(3)} \times  \mathfrak{so(3)} \to \R^+$. Esso consente di definire un'identificazione $I$, detta \dfn{tensore d'inerzia}, tra $\mathfrak{so(3)}$ e $\mathfrak{so(3)}^*$, tramite
\begin{equation}
I: \psi \mapsto \psi^* \qqtext{tale che} \psi^* (\psi) = \langle \psi, \psi \rangle
\end{equation} 
L'identificazione tra velocità e momenti angolari fornita da $I$ è analoga a quella tra velocità e momenti lineari fornita dalla massa. I valori numerici del prodotto scalare dipendono dalla distribuzione di massa del corpo in considerazione. L'energia cinetica (e quindi l'hamiltoniana) può essere definita su $T^* SO(3)$ tramite il tensore d'inerzia, la norma su $\mathfrak{so(3)}$ e il pullback dei covettori da $T^*_{\rho}SO(3)$ a $T^*_{e}SO(3)$, come
\begin{equation}
\mathcal{H}(\rho, \dot{\rho}^*) \defeq \Big\langle I^{-1}\big(L^*_\rho \dot{\rho}^*\big), I^{-1}\big(L^*_\rho \dot{\rho}^*\big) \Big\rangle
\end{equation}

$SO(3)$-varietà ammette chiaramente $SO(3)$-gruppo come simmetria, e per il teorema di Noether genera quindi un'azione hamiltoniana su $T^*SO(3)$. Si può dimostrare che la mappa momento di questa azione è data da 
\begin{equation}
\mu^{SO(3)}: (\rho, \dot{\rho}^*) \longmapsto R_{\rho}^* \dot{\rho}^* 
\end{equation}
ovvero proprio dal momento angolare corrispondente al momento generalizzato $\dot{\rho}^*$. Per semplicità, si ponga $\mu \defeq \mu^{SO(3)}$. Dato che i valori di $\mu$ si conservano, si è ritrovata la legge di conservazione del momento angolare. Il riottenimento di questa legge, nota dalla meccanica newtoniana, è la motivazione principale per la definizione di velocità e momento angolari in $\mathfrak{so(3)}$ ed $\mathfrak{so(3)}^*$ e per l'uso dell'azione destra nella definizione del momento angolare.

% Siccome $\lambda = R^*_{\rho} \dot{\rho}^*$ è un elemento di $\mathfrak{so(3)}^*$, esso è duale alle velocità angolari $\psi$. La sua azione su di esse è data per definizione di pull-back da
% \begin{equation}
% \lambda(\psi) = \big[(R_\rho^* \dot{\rho}^*)\big]\big((DL_{\rho^{-1}})\dot{\rho}\big) = \dot{\rho}^* \Big((DR_\rho)\big((DL_{\rho^{-1}})\dot{\rho}\big)\Big) \defeq \dot{\rho}^*(\dot{\rho}')
% \end{equation}
% Si noti innanzitutto che l'effetto combinato delle due azioni $L_{\rho^{-1}}$ e $R_\rho$ è di portare $\rho$ in $e$ e poi di nuovo in $\rho$. Quindi anche $\dot{\rho}'$ è tangente a $SO(3)$ in $\rho$. Un calcolo diretto mostra poi che per un cammino su $SO(3)$ che definisce $\dot{\rho}$ le differenze dalla sua immagine attraverso le due azioni, che definisce $\dot{\rho}'$, sono di ordine maggiore o uguale a $2$. Per i vettori tangenti, intesi come classi di equivalenza di moti al primo ordine, si ha quindi che $\dot{\rho} = \dot{\rho}'$. Insomma vale 
% \begin{equation}
% \lambda(\psi) = \dot{\rho}^*(\dot{\rho})
% \end{equation}
% L'interpretazione fisica del valore assunto da $\mu^{SO(3)}$ è quindi semplice: le sue componenti sono le componenti del momento angolare nella base in cui è espressa la velocità angolare, identificando $\mathfrak{so(3)} \simeq \R^3 \simeq (\R^3)^* \simeq \mathfrak{so(3)}^*$.

Per un dato $\lambda \in \mathfrak{so(3)}^*$, è possibile dimostrare che $\mu^{-1}(\lambda)/SO(3)_{\lambda} \simeq S^2$, l'insieme degli elementi $\psi \in \mathfrak{so(3)}$ con norma pari a $\norm{\lambda}_{\R^3}$, dove $\psi, \lambda$ sono visti come elementi di $\mathfrak{so(3)}^* \simeq \R^3$ e $\norm{\cdot}_{\R^3}$ è la norma euclidea in $\R^3$. Di nuovo, essa avrà una struttura simplettica data dalla forma simplettica ridotta, mentre la dinamica sarà governata dall'hamiltoniana $\mathcal{H}$, ristretta a $S^2$. Il significato della riduzione dello spazio delle fasi a $S^2$ è che lo stato di rotazione del sistema (sia la sua configurazione che i suoi momenti) è completamente specificato dalla posizione del suo asse di rotazione, una volta fissate le condizioni iniziali.

È inoltre possibile fare una considerazione del tutto generale sulla stabilità delle rotazioni. Siccome l'hamiltoniana si conserva nel corso del moto, questo dovrà svolgersi sulle intersezioni di $S^2$ con le superfici di livello di $\mathcal{H}$. Essendo una forma quadratica, essa avrà superfici di livello ellissoidali, per via del teorema spettrale. La sfera intersecherà l'ellissoide intorno agli assi. Le intersezioni vicine all'asse maggiore e minore resteranno nell'intorno dell'asse, mentre quelle vicine all'asse intermedio ne usciranno. Ciò significa che, a livello fisico, le rotazioni di un corpo rigido libero attorno agli assi con momento d'inerzia minore e maggiore sono stabili, mentre le rotazioni attorno all'asse intermedio sono instabili. Questo risultato è noto come \dfn{teorema di Poinsot}.

\end{document}