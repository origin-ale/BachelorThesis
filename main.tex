\documentclass[%
  twoside,           % carta B5
  corpo=12pt,        % corpo normale 11pt
  tipotesi=custom, % laurea triennale
  draft
]{toptesi}
\usepackage{toptesi}
\usepackage{xcolor}

\begin{pdfxmetadata}
  \Title{Titolo}
  \Author{Alessandro Cerati}
  \Publisher{Alessandro Cerati}
  \Keywords{Monografia di laurea \sep
            Geometria simplettica}
\end{pdfxmetadata}

\usepackage[a-1b]{pdfx}% oppure [a-2b]

\usepackage{fontspec}
\setmainfont[Ligatures=TeX]{Libertinus Serif}
\setsansfont[Ligatures=TeX]{Libertinus Sans}
\setmonofont{NewCMMono10-Regular.otf}
\setmainlanguage[babelshorthands]{italian}
\usepackage{amsmath}
\usepackage{unicode-math}
\setmathfont{XITS Math}
\setotherlanguage[variant=ancient]{greek}
\newfontfamily{\greekfont}{GFS Didot}

\begin{document}
\begin{titlepage}
  %
  %
  % UNA VOLTA FATTE LE DOVUTE MODIFICHE SOSTITUIRE "RED" CON "BLACK" NEI COMANDI \textcolor
  %
  %
  \begin{center}
  {{\Large{\textsc{Alma Mater Studiorum $\cdot$ Università di Bologna}}}} 
  \rule[0.1cm]{\textwidth}{0.1mm}
  \rule[0.5cm]{\textwidth}{0.6mm}
  \\\vspace{3mm}
  
  {\small{\bf Scuola di Scienze \\ 
  Dipartimento di Fisica e Astronomia\\
  Corso di Laurea in Fisica}}
  
  \end{center}
  
  \vspace{23mm}
  
  \begin{center}\textcolor{red}{
  %
  % INSERIRE IL TITOLO DELLA TESI
  %
  {\LARGE{\bf TITOLO TESI}}\\
  }\end{center}
  
  \vspace{50mm} \par \noindent
  
  \begin{minipage}[t]{0.47\textwidth}
  %
  % INSERIRE IL NOME DEL RELATORE CON IL RELATIVO TITOLO DI DOTTORE O PROFESSORE
  %
  {\large{\bf Relatore: \vspace{2mm}\\\textcolor{red}{
  Prof. Emanuele Latini}}}\\\\

  \end{minipage}
  %
  \hfill
  %
  \begin{minipage}[t]{0.47\textwidth}\raggedleft \textcolor{red}{
  {\large{\bf Presentata da:
  \vspace{2mm}\\
  %
  % INSERIRE IL NOME DEL CANDIDATO
  %
  Alessandro Cerati}}}
  \end{minipage}
  
  \vspace{40mm}
  
  \begin{center}
  %
  % INSERIRE L'ANNO ACCADEMICO
  %
  Anno Accademico \textcolor{red}{ 2023/2024}
  \end{center}
  
  \end{titlepage}

\chapter{Introduzione}
Vogliamo parlare di \emph{geometria simplettica}.

\end{document}