\chapter{Formulazioni della meccanica}
La meccanica classica si propone di studiare il moto dei corpi macroscopici in movimento con velocità trascurabile rispetto a quella della luce. In questo capitolo esporremo tre \emph{formulazioni} della meccanica classica, ovvero tre modi per ricavare le equazioni del moto. Dopo aver richiamato l'intuitiva ma scomoda formulazione \emph{newtoniana} basata sulle forze, costruiremo la formulazione \emph{lagrangiana} dal principio di minima azione e mostreremo la sua equivalenza con quella newtoniana. Infine, grazie alla trasformata di Legendre, passeremo alla potente formulazione \emph{hamiltoniana}.

\section{Il modello classico dell'universo}
Per dare una formulazione assiomatica alla meccanica classica è innanzitutto necessario stabilire alcune caratteristiche dell'universo per quanto riguarda i corpi che si muovono su scala macroscopica (azione molto maggiore della costante di Planck $\hbar$) e a basse velocità (velocità molto minore di quella della luce $c$). Queste caratteristiche dovranno essere emulate dal modello di universo che costruiremo. Siccome l'ambito di applicazione di questo modello comprende l'esperienza quotidiana, è da essa che possiamo trarre spunto. Poniamo quindi i seguenti requisiti al nostro modello:
\begin{enumerate}
  \item L'universo è composto da \emph{spazio e tempo}. Lo spazio è tridimensionale ed euclideo, il tempo è unidimensionale.
  \item Localizziamo gli eventi nello spazio e nel tempo usando \emph{sistemi di coordinate}. Esistono sistemi, detti \emph{inerziali}, tali che:
  \begin{enumerate}
    \item A ogni istante, tutte le leggi della natura sono uguali in ogni sistema inerziale.
    \item Tutti i sistemi che si muovono di moto rettilineo e uniforme rispetto a un sistema inerziale sono inerziali.
  \end{enumerate}
  \item L'universo è popolato di \emph{particelle}, entità adimensionali dotate di una massa e una posizione. Esso è inoltre è \emph{newtonianamente deterministico}: l'insieme delle posizioni e delle velocità di tutti i particelle a un certo tempo determina tutto il suo moto, sia nel passato che nel futuro.
\end{enumerate}

Costruiamo ora un modello formale che presenti queste stesse caratteristiche. Per fare ciò, sarà necessaria una struttura matematica che formalizza l'idea di uno spazio-tempo in cui gli spostamenti sono vettori, ma non è definita un'origine. Tale struttura è nota come spazio affine.
\begin{definition}
  Uno \dfn{spazio affine $n$-dimensionale} è una struttura $(A,V,+)$ dove:
  \begin{itemize}
    \item $A$ è un insieme.
    \item $V$ è uno spazio vettoriale reale.
    \item $+:\, A\times V \to A$ è biunivoca. L'immagine della coppia $(a,v)$ con $a \in  A, v \in V$ si denota con $a+v$.
  \end{itemize}
  La \dfn{dimensione} di $A$ è la dimensione di $V$.
\end{definition}
\begin{remark}
  Siccome $+$ è biunivoca, fissati $a,b \in A$ esiste uno e un solo $v \in V$ tale che $a + v = b$. Tale elemento di $V$ si denota con $v = b - a$. In tal modo, gli spostamenti (differenze) tra punti di $A$ sono effettivamente vettori.
\end{remark}
\begin{remark}
Fissato un $a \in A$, l'insieme dei $v$ tali che $v = b - a$ per un qualche $b \in A$ forma uno spazio vettoriale reale isomorfo a $V$. Si dice che $V = A - A$.
\end{remark}

Possiamo a questo punto formulare le componenti centrali del nostro modello: l'universo, il tempo, gli eventi contemporanei e la distanza fra di essi. Queste andranno a formare una struttura detta galileiana.
\begin{definition}
  Una \dfn{struttura spazio-temporale galileiana} è composta da:
  \begin{enumerate}
    \item L'\dfn{universo}, uno spazio affine quadridimensionale $\mathbb{A}^4$ i cui punti sono detti \emph{eventi}.
    \item Il \dfn{tempo}, un'applicazione lineare $t:\mathbb{A}^4 - \mathbb{A}^4 \to \R$. Si dice \dfn{intervallo temporale} fra due eventi $a,b \in \mathbb{A}^4$ il numero reale $t(a-b)$. Se esso è nullo, $a$ e $b$ si dicono \dfn{contemporanei}.
    \item La \dfn{distanza fra eventi contemporanei} \begin{equation}
    d(a,b) = \norm{a-b}_2 = \sqrt{\ev{a-b,a-b}}  
    \end{equation} 
  \end{enumerate}
\end{definition}

Queste definizioni tuttavia non consentono ancora di formulare la meccanica classica. Non abbiamo modo di identificare un punto nello spazio, siccome gli spazi di eventi contemporanei sono tutti separati tra loro. Per collegarli, introduciamo l'ultima componente, i sistemi di riferimento.
\begin{definition}
  Si dice \dfn{sistema di riferimento} un'applicazione lineare biunivoca $\phi:\, \mathbb{A}^4 \to \R\times \R^3$. Un sistema di riferimento $\phi_1$ si dice \dfn{in moto uniforme} rispetto a $\phi_2$ se $\phi_1 \phi_2^{-1}: \R\times \R^3\to \R\times \R^3$ conserva intervalli temporali e distanze.
\end{definition}

Siamo finalmente in grado di descrivere posizioni e moti dei particelle che popolano l'universo.
\begin{definition}
  Si dice \dfn{moto} di una particella in un sistema di riferimento $\phi$ un'applicazione differenziabile $\vec{x}:I \to \R^3$, dove $I$ è un intervallo aperto in $\R$, che manda $t\mapsto \vec{x}(t)$. Si dice \dfn{posizione} del punto al tempo $t \in I$ il vettore $\vec{x}(t)$. Si dicono \dfn{velocità} e \dfn{accelerazione} del punto al tempo $t \in I$ rispettivamente \begin{equation}
  \vec{\dot{x}} (t) = \dv{\vec{x}}{t}\, (t) \qqtext{e} \vec{\ddot{x}} (t) = \dv[2]{\vec{x}}{t}\, (t)
  \end{equation} 
\end{definition}
\begin{remark}
  Velocità e accelerazione così definiti sono vettori del medesimo spazio vettoriale $\R^3$ a cui appartiene la posizione.
\end{remark}

Per descrivere le posizioni di $N$ particelle servono $N$ vettori a 3 componenti. La risultante struttura può essere identificata con un vettore del prodotto diretto di $N$ copie di $\R^3$. Questo vettore descrive la configurazione dell'intero sistema, il che motiva la seguente definizione.
\begin{definition}
  Si dice \dfn{spazio delle configurazioni} di un sistema di $N$ particelle il prodotto diretto di $N$ copie di $\R^3$: \begin{equation}
  \mathbb{M} = \underbrace{\R^3 \times \R^3 \times \ldots \times \R^3}_{N \text{ volte}}
  \end{equation} 
  Moto, posizione, velocità e accelerazione di un tale sistema sono definite analogamente a quanto fatto per una sola particella, sostituendo $\R^3$ con $\mathbb{M}$.
\end{definition}

\section{Meccanica newtoniana}
Finora ci siamo occupati solo di come descrivere i moti delle particelle nell'universo. L'obiettivo della meccanica, tuttavia, è determinare in anticipo i moti di queste particelle sulla base di informazioni note, grazie al principio di determinismo newtoniano. Nella formulazione originaria, dovuta a Newton stesso, i moti delle particelle sono previsti in base alle forze che agiscono su di esse, determinate sperimentalmente. In una versione moderna, la meccanica può essere formulata come segue.
\begin{newton}
  Siano $\mathbb{M}$ lo spazio delle configurazioni di un sistema e $I$ un intervallo reale aperto, esiste una funzione $\vec{F}:\mathbb{M} \times \mathbb{M} \times  I \to \mathbb{M}$ tale che \begin{equation}
  \vec{\ddot{x}} = \vec{F}\,(\vec{x},\vec{\dot{x}}, t) \label{eq:newton}
  \end{equation} 
  Questa equazione è detta \dfn{equazione del moto} ed $\vec{F}$ è detta \dfn{forza} agente sul sistema.
\end{newton}
\begin{remark}
  La funzione $\vec{F}$ si può ottenere dalla formulazione elementare della meccanica, che vede le forze come entità agenti su ciascuna particella in maniera distinta, moltiplicando la forza risultante su ciascuna particella per la massa di quest'ultima e concatenando i vettori risultanti.
\end{remark}
\begin{remark}
  La legge di Newton soddisfa il principio di determinismo grazie al teorema di esistenza e unicità di Cauchy: data la condizione iniziale, ovvero posizione e velocità del sistema a un tempo $t_0$, la soluzione dell'\autoref{eq:newton} esiste ed è unica nell'intervallo temporale $I$ in cui $\vec{F}$ è definita.
\end{remark}
La funzione $\vec{F}$ per un dato sistema fisico deve essere determinata sperimentalmente. Il modello matematico del sistema viene costruito nel momento in cui viene definita $\vec{F}$, motivo per cui essa è spesso identificata con il sistema stesso.

La conoscenza di $\vec{F}$ consente in linea di principio di conoscere il moto del sistema per qualsiasi posizione iniziale. Ciò tuttavia è molto più facile a dirsi che a farsi: per un sistema di $N$ particelle, questo metodo richiede di risolvere $3N$ equazioni differenziali, un processo che spesso è impossibile dal punto di vista analitico e inefficiente da quello numerico. I concetti legati all'energia consentono però di fare affermazioni qualitative sul moto dei corpi anche quando l'equazione del moto non è analiticamente risolvibile.
\begin{definition}
  Si dice \dfn{energia cinetica di un sistema} formato da $N$ particelle di masse $m_i$ nelle posizioni $\vec{x}_i \in \R^3$ la quantità \begin{equation}
    T(\vec{\dot{x}}) = \sum_{i=1}^{N} \frac{1}{2}m_i\norm{\vec{\dot{x}}_i}^2
  \end{equation} 
\end{definition}
\begin{definition}
  Una forza dipendente solo dalle posizioni del sistema si dice \dfn{campo di forze}. Si dice \dfn{lavoro} del campo di forze $\vec{F}\, (x)$ sul cammino $\gamma \subset \mathbb{M}$ la quantità \begin{equation}
  L = \int\limits_{\gamma} \vec{F}\,(\vec{x})\cdot \dd \vec{x}
  \end{equation} 
  Un campo di forze e il corrispondente sistema si dicono \dfn{conservativi} se il lavoro del campo su un cammino qualsiasi non dipende dal cammino stesso. In tal caso, si dice \dfn{potenziale} rispetto a $\vec{x}_0 \in \mathbb{M}$ la quantità definita simbolicamente come \begin{equation}
  V(\vec{x}) = \int_{\vec{x}_0}^{\vec{x}} \vec{F}\,(\vec{x})\cdot \dd \vec{x}
  \end{equation} 
  dove l'integrale è compiuto lungo un qualsiasi percorso $\gamma$ di estremi $\vec{x}_0$ e $\vec{x}$.
\end{definition}
\begin{definition}
  Si dice \dfn{energia totale} di un sistema conservativo\begin{equation}
  E(\vec{x},\vec{\dot{x}}) = T(\vec{\dot{x}}) + V(\vec{x})
  \end{equation} 
\end{definition}
\begin{theorem} \label{thm:energyCons}
L'energia totale di un sistema conservativo è costante nel tempo.
\end{theorem}

Il \autoref{thm:energyCons} consente ad esempio di determinare la regione di spazio delle configurazioni ammessa per un sistema, date le condizioni iniziali.

\section{Meccanica lagrangiana}
La conservazione dell'energia di un sistema è un metodo piuttosto semplice per ottenere informazioni qualitative sul suo comportamento, ma queste non sono particolarmente dettagliate. Si possono ottenere maggiori dettagli sul moto di un sistema attraverso la formulazione \dfn{lagrangiana} della meccanica. Il formalismo della teoria è basato su un principio \emph{variazionale}. Poniamo per prima cosa le definizioni.

\begin{definition}
  Si dice \dfn{lagrangiana} di un sistema conservativo la funzione \begin{equation}
  \mathcal{L}(\vec{x},\vec{\dot{x}}) = T(\vec{\dot{x}}) - V(\vec{x})
  \end{equation} 
  Si dice \dfn{azione} di un moto $\vec{x}:t\mapsto \vec{x}(t)$ l'integrale della lagrangiana nel tempo \begin{equation} \label{eq:action}
  S(t_0,t_1, \vec{x}) = \int_{t_0}^{t_1} \mathcal{L}(\vec{x}(t),\vec{\dot{x}}(t),t) \dd{t}
  \end{equation}
\end{definition}
\begin{definition}
  Si dice \dfn{funzionale} un'applicazione tra uno spazio di funzioni differenziabili e l'asse reale.
\end{definition}
Fissati un tempo iniziale e finale $t_0$ e $t_1$ e una configurazione iniziale e finale $\vec{x}_0$ e $\vec{x}_1$, l'azione è un funzionale definito sullo spazio dei moti $\vec{x}$ tali che $\vec{x}(t_0) = \vec{x}_0$ e $\vec{x}(t_1)=\vec{x}_1$. Per avere una formulazione della meccanica, è necessario un modo per determinare quale moto viene effettivamente realizzato. Intuitivamente, questo dovrebbe avere un qualche tipo di \textquotedblleft efficienza\textquotedblright: non dovrebbe seguire un percorso inutilmente lungo, né variare la sua velocità più del necessario. Queste idee intuitive sono raccolte in una formulazione matematica dal principio di minima azione, originariamente dovuto a Maupertuis.
\begin{minaction}
Il moto fisicamente realizzato da un sistema fra due punti nello spazio delle configurazioni è quello per cui l'azione è minima.
\end{minaction}
Il problema della determinazione pratica di quale sia effettivamente il moto che richiede la minima azione richiede lo sviluppo del calcolo delle variazioni per essere risolto completamente; nel presente elaborato ci limiteremo a presentare il teorema che risulta da una trattazione completa.
\begin{theorem}
  Affinché un moto $\vec{x}: t\mapsto \vec{x}(t)$ sia un estremale dell'azione $S(\vec{x})$ (definita dall'\autoref{eq:action}) fissati tempi e punti iniziali e finali $t_0, t_1, \vec{x}_0, \vec{x}_1$ è necessario che esso soddisfi le \dfn{equazioni di Eulero-Lagrange} \begin{equation}
    \dv{t}(\pdv{\mathcal{L}}{\vec{\dot{x}}}) - \pdv{\mathcal{L}}{\vec{x}} = 0
  \end{equation} 
  dove $\pdv{\vec{x}}$ indica il gradiente $\nabla_{\vec{x}}$.
\end{theorem}

È possibile dimostrare che, come si spera, i moti ottenute per un dato sistema meccanico dalla formulazione newtoniana e da quella lagrangiana coincidono. Come menzionato sopra, la formulazione lagrangiana semplifica però notevolmente i calcoli in molte situazioni. Un primo vantaggio è dovuto al fatto che essa consente di svolgere i calcoli non solo usando le componenti dei vettori di $\mathbb{M}$, ma anche con coordinate arbitrarie.

\begin{definition}
  In un sottospazio dello spazio delle configurazioni $U \subseteq \mathbb{M}$ di dimensione $n$ si dicono \dfn{coordinate} $n$ numeri reali $\vec{q}=(q_1, \ldots, q_n)$ tali che esiste una funzione biunivoca $\phi: \vec{q} \mapsto \vec{x}\in U$.
\end{definition}
\begin{theorem}
  Un moto soddisfa le equazioni di Eulero-Lagrange in $\vec{x}$ se e solo se le soddisfa in coordinate $\vec{q}$ arbitrarie per $\mathbb{M}$: \begin{equation}
    \dv{t}(\pdv{\mathcal{L}}{\vec{\dot{q}}}) - \pdv{\mathcal{L}}{\vec{q}} = 0
  \end{equation} 
\end{theorem}

Questo significa che le equazioni del moto di un sistema ad esempio in coordinate polari possono essere ottenute immediatamente una volta scritte l'energia cinetica $T\big(\vec{\dot{x}}(\vec{q},\vec{\dot{q}})\big)$ e il potenziale $V\big(\vec{x}(\vec{q})\big)$ in coordinate polari.

Un'altra caratteristica della formulazione lagrangiana è il fatto che esse sono \emph{intrinseche} ai sistemi vincolati: vediamo di precisare cosa s'intende.

\begin{definition}
  Sia $\mathbb{M}$ uno spazio delle configurazioni di dimensione $n$ con coordinate arbitrarie $\vec{q} = (q_1, \ldots, q_n)$. Si dice \dfn{vincolo olonomo} la richiesta che il moto abbia un'immagine contenuta nel sottoinsieme $M$ di $\mathbb{M}$ definito da \begin{equation}
    M = \left\{ a \in \mathbb{M} \left|\ \begin{cases} 
      F_1(\vec{q}) = 0 \\
      \vdots \\
      F_k(\vec{q}) = 0
    \end{cases} \right. \right\} \qqtext{con} \vec{q} = \phi^{-1}(a)
  \end{equation} 
  dove $k \le n$ e $F_1, \ldots, F_k$ sono funzioni differenziabili da $\mathbb{M}$ a $\R$, tali che il rango della matrice $\pdv{F_i}{q_j}(a)$ sia $k$ in ogni $a \in M$. $M$ è detto \dfn{spazio delle configurazioni di un sistema vincolato}.
\end{definition}
\begin{remark}
  Un vincolo non olonomo, detto \dfn{anolonomo}, può vincolare anche la velocità del sistema. Non parleremo qui di vincoli anolonomi, quindi nel seguito con \emph{vincolo} si intenderà \emph{vincolo olonomo}.
\end{remark}

Come si può imporre formalmente un vincolo su un sistema? Intuitivamente, un vincolo è un'entità che applica una forza infinita sul sistema quando esso tenta di uscire dal sottoinsieme di configurazioni permesse $M$. Avendo a disposizione solo potenziali, possiamo pensare di modellizzare questa forza infinita con un potenziale a pendenza tendente all'infinito lungo le direzioni che si allontanano dal vincolo. Esprimendo per il teorema del Dini le prime $k$ coordinate in funzione delle altre $n-k$ \begin{equation}
\begin{cases}
  q_1 = f_1(q_{k+1}, \ldots, q_n) \\
  \vdots \\
  q_k = f_k(q_{k+1}, \ldots, q_n)
\end{cases}
\end{equation} 
il potenziale che ci serve è \begin{equation}
  V_{\alpha}(\vec{q}) = V(\vec{q}) + \alpha\big[(q_1-f_1)^2 + \ldots + (q_k-f_k)^2 \big] \qqtext{per} \alpha \to +\infty
\end{equation} 
dove $V$ è il potenziale dello stesso sistema senza vincoli. Usando questo potenziale è possibile dimostrare il seguente teorema.
\begin{theorem}
  Sia $M \subset \mathbb{M}$ lo spazio delle configurazioni di un sistema vincolato con lagrangiana $\mathcal{L}_{\alpha} = T -V_{\alpha}$, siano date le condizioni iniziali $\vec{q}_0 \in  M$ e $\vec{\dot{q}}_0$ tangente a $M$ e sia $\vec{\phi}_{\alpha}(t)$ il moto del sistema. Allora esiste il limite \begin{equation}
  \lim_{\alpha \to +\infty} \vec{\phi}_{\alpha}(t) = \vec{\psi}(t)
  \end{equation} 
  e la funzione limite $\vec{q}(t) = \vec{\psi}(t)$ soddisfa le equazioni di Eulero-Lagrange \begin{equation}
    \dv{t}(\pdv{\mathcal{L}_{*}}{\vec{\dot{q}}}) - \pdv{\mathcal{L}_{*}}{\vec{q}} = 0
  \end{equation} 
  dove $\mathcal{L}_*$ è detta \dfn{lagrangiana ridotta} ed è definita come \begin{equation}
  \mathcal{L(\vec{q},\vec{\dot{q}})}_* = \eval{T(\vec{q},\vec{\dot{q}})\,}_{\begin{subarray}{l}
    q_1=f_1, \ldots, q_k=f_k \\
    \dot{q}_1, \ldots, \dot{q}_k = 0
  \end{subarray}} - \eval{V(\vec{q})\,}_{q_1=f_1, \ldots, q_k=f_k}
  \end{equation} 
\end{theorem}
Questo teorema consente in pratica di scrivere la lagrangiana esprimendo le coordinate vincolate in funzione delle altre, per poi calcolare le equazioni del moto solo per le coordinate non vincolate. 
\section{Meccanica hamiltoniana}