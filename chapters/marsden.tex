\chapter{Meccanica simplettica}
Questo capitolo costituisce il punto di arrivo del percorso svolto finora. Si esamineranno le formulazioni della meccanica esposte nel capitolo 1 dal punto di vista della teoria delle varietà differenziabili, e si ricaverà dalla meccanica hamiltoniana l'agognata \emph{formulazione simplettica} della meccanica, la quale non fa uso di coordinate. Uno dei più potenti strumenti a disposizione di questa formulazione è la riduzione simplettica trattata nel capitolo 3, la quale qui verrà applicata allo studio del moto del corpo rigido.

\section{Meccanica e varietà}
Nel capitolo 1 si è definito lo spazio delle configurazioni $\mathbb{M}$ di un sistema non vincolato di $N$ particelle come il prodotto di $N$ copie di $\R^{3}$. Questo spazio è identificabile con $\R^{3N}$. Se sulle particelle viene imposto un vincolo, lo spazio delle configurazioni del sistema diventa un qualche insieme $M \subset \R^{3N}$. Come già affermato, questo insieme non avrà in generale una struttura di spazio vettoriale. I suoi punti $x$ potranno però essere individuati da $n$-uple $\vec{q}$, dette coordinate, tramite funzioni biunivoche $\phi^{-1}: \vec{q} \mapsto x$. $M$ avrà quindi la struttura più generale di una \emph{varietà differenziabile}, con atlante dato dalle funzioni $\phi: x \mapsto \vec{q}$ e dai loro domini di biunivocità. Si dice che lo spazio delle configurazioni di un sistema di $N$ particelle con $3N-n$ vincoli è una varietà $n$-dimensionale \dfn{immersa} in $\R^{3N}$, lo spazio $3N$-dimensionale delle configurazioni senza vincoli.

Per individuare univocamente un generico sistema sono necessari due elementi: il suo spazio delle configurazioni $M$ e l'energia potenziale, definita proprio sullo spazio delle configurazioni, $V: x \in M \mapsto V(x) \in \R$.  

% La velocità $\dot{x}$ di un sistema è interpretabile come un vettore tangente allo spazio delle configurazioni nel punto $x(t)$, facendo corrispondere a ogni $\gamma \in T_x M$ il vettore $\dot{x} = \dot{\gamma}(0)$, che appartiene a $\R^{n}$ poiché $\gamma: ]-\epsilon,\epsilon[ \to M \subset \R^{n}$ ed è tangente nel senso ordinario (ortogonale al gradiente dei vincoli) alla varietà immersa $M$. Su ogni spazio tangente è quindi definita l'energia cinetica come forma quadratica $K_x: \dot{x} \mapsto \frac{1}{2} \dot{x}^2$.

La velocità $\dot{x}$ di un sistema è definita come derivata temporale $\dv{\overline{x}}{t}()(t)$ del suo moto $\overline{x}: t \in I \mapsto \overline{x}(t) \in M$, dove $I$ è un intervallo reale. Per definizione, un vettore tangente alla varietà $M$ nel punto $x \in M$ è una classe di equivalenza di cammini infinitesimi che passano da $x$ al tempo $0$. Per un dato $x =\overline{x}(t)$, si può traslare il tempo e ridurne il dominio in modo da ottenere un nuovo cammino $\gamma:]-\epsilon, \epsilon\,[$ tale che $x = \gamma(0)$. Questo cammino sarà inoltre tale che $\dot{x} = \dv{\gamma}{t}()(0)$. Siccome $M \subset \R^{n}$, una scelta valida per $\phi$ è l'identità $\identity$. Quindi, perché un secondo cammino $\gamma'$ sia equivalente a $\gamma$, è necessario e sufficiente che \begin{equation*}
\eval{\dv{t}}_{t=0} \gamma'(t) = \eval{\dv{t}}_{t=0} \gamma(t) = \dot{x}
\end{equation*}
per definizione: la velocità $\dot{x}$ caratterizza la classe $[\gamma]$ dello spazio tangente a $M$ in $x$. Questo potrà quindi essere identificato con lo spazio delle velocità possibili quando il sistema ha configurazione $x$. Si può quindi dire che la velocità di un sistema è un vettore tangente allo spazio delle configurazioni nella configurazione attuale del sistema. L'energia cinetica, funzione della velocità, è quindi definita per ciascun $x \in M$ su $T_x M$ da $K_x: \dot{x} \mapsto \frac{1}{2} \dot{x}^2$, dove la norma è la norma in $\R^{n}$.

La lagrangiana di un sistema dipende sia dall'energia potenziale che dall'energia cinetica, e dunque sia dalla configurazione del sistema che dalla sua velocità. Essa deve quindi essere definita su $TM$, il fibrato tangente di $M$. Le energie cinetica e potenziale dovranno quindi cambiare dominio da rispettivamente $T_x M$ e $M$ a $TM$ per poter formare la lagrangiana. Sia $(x,\dot{x}) \in TM$, con $x \in M$ e $\dot{x} \in T_x M$ ciò si può fare definendo \begin{equation*}
\begin{aligned}
  &V(x,\dot{x})|_{TM} = V(x)|_M \\
  &K(x, \dot{x})|_{TM} = K(\dot{x})|_{T_x M}
\end{aligned}
\end{equation*} 
Si può a questo punto definire la lagrangiana come una funzione differenziabile $\mathcal{L}:TM\to \R$ data per $x \in M$ e $\dot{x} \in T_x M$ da \begin{equation*}
\mathcal{L}(x,\dot{x}) = K(\dot{x}) - V(x)
\end{equation*}

La trasformata di Legendre rispetto alle velocità trasforma la lagrangiana $\mathcal{L}(x,\dot{x})$ nell'hamiltoniana $\mathcal{H}(x,\dot{x}^*)$ dove gli $\dot{x}^*$ sono elementi dello spazio duale a quello degli $\dot{x}$. Ma poiché $\dot{x} \in T_x M$, ciò significa che $\dot{x}^* \in T_x^* M$, e quindi l'hamiltoniana di un sistema è definita sul fibrato cotangente del suo spazio delle configurazioni. Il momento generalizzato $\dot{x}^*$ è infatti un vettore cotangente, dato da \begin{equation*}
  \dot{x}^* = \dd_{\dot{x}} \mathcal{L}
\end{equation*}
Lo spazio delle fasi può quindi essere definito in maniera indipendente dalle coordinate come il fibrato cotangente $T^* M$ dello spazio delle configurazioni. Se una regione di $M$ è coperta dalla carta $(U,\phi)$, un punto $x \in U$ si può individuare con $\vec{q} = (q_1, \ldots, q_n) = (\phi_1(x), \ldots, \phi_n(x))$ e il momento generalizzato cotangente a $M$ in $x$ si può individuare con $\vec{p}=(p_1, \ldots, p_n)$, le coordinate nella base duale data da $\{\dd_x \phi_1, \ldots, \dd_x \phi_n\} $. Si recupera così la definizione basata sulle coordinate.

La \autoref{def:hamField} estende la \autoref{eq:hamFieldUnconstr} al caso di varietà che non costituiscono spazi vettoriali. Si consideri infatti una varietà simplettica $2n$-dimensionale $X$. In una carta $(U,\phi)$ siano i suoi punti $x \in U$ individuati dalle coordinate $(x_1, \ldots, x_{2n})$ e i vettori $\gamma$ tangenti a $X$ in un punto $y \in U$ individuati dalle coordinate $(\gamma_1, \ldots, \gamma_{2n})$. Allora il differenziale $\dd{\mathcal{H}}$ è rappresentato dalla trasposta del gradiente su queste coordinate $(\grad{\mathcal{H}})^T$, così che \begin{equation*}
\dd{\mathcal{H}}(\gamma) = (\grad{\mathcal{H}})^T \begin{pmatrix} \gamma_1\\ \vdots\\ \gamma_{2n} \end{pmatrix}
\end{equation*} 
Allo stesso tampo, la forma simplettica $\omega$ sarà rappresentata da $\mathsf{J} = \left( \begin{smallmatrix}
  0 & -\mathsf{I} \\ \mathsf{I} & 0
\end{smallmatrix}  \right) $ e il campo $X_{\mathcal{H}}$ da un vettore $\mathsf{X_{\mathcal{H}}}$, così che
\begin{equation*}
\dd{\mathcal{H}}(\gamma) = \omega(X_{\mathcal{H}}, \gamma) = \mathsf{X_{\mathcal{H}}}^T\, \mathsf{J}\, \begin{pmatrix} \gamma_1\\ \vdots\\ \gamma_{2n} \end{pmatrix}
\end{equation*} 
Da ciò segue che $\mathsf{X_{\mathcal{H}}}^T \mathsf{J} = (\grad{\mathcal{H}})^T$. Trasponendo e moltiplicando per $\mathsf{J}$, siccome $\mathsf{J}^2 =- \mathsf{I}$, si ha
\begin{equation*}
\mathsf{X}_{\mathcal{H}} = -\mathsf{J} \grad{\mathcal{H}} 
\end{equation*}
ovvero per le coordinate vale \begin{equation*}
\dot{q}_i = \pdv{\mathcal{H}}{p_i} \qquad \dot{p}_i = - \pdv{\mathcal{H}}{q_i}
\end{equation*}
che sono le equazioni di Hamilton. Le linee di flusso del campo vettoriale hamiltoniano minimizzano quindi l'azione associata alla lagrangiana ridotta sullo spazio delle configurazioni vincolato. Esse minimizzano quindi l'azione, e dunque rappresentano le posizioni e le velocità che il sistema ha nel suo moto. Con quest'ultimo tassello è finalmente possibile porre una formulazione della meccanica di sistemi con vincoli olonomi arbitrari puramente geometrica e libera da coordinate.

Un sistema fisico è definito dal suo \emph{spazio delle configurazioni}, una varietà differenziabile $n$-dimensionale $M$ in cui ogni punto $m$ corrisponde a una diversa configurazione delle componenti del sistema, e da una funzione \emph{hamiltoniana} $\mathcal{H}: M\to \R$, che ne governa il movimento. Lo \emph{stato} di un sistema è rappresentato da un punto nello \emph{spazio delle fasi}, il fibrato cotangente $X = T^* M$ dello spazio delle configurazioni. Sullo spazio delle fasi è definita la \emph{forma simplettica canonica} $\omega$. Tramite essa è definito il \emph{campo vettoriale hamiltoniano} $X_{\mathcal{H}}$, ovvero quel campo tale che, per un qualsiasi vettore $\gamma$ tangente allo spazio delle fasi in un qualche punto $x \in X$, prenderne la forma simplettica con il campo hamiltoniano $\omega_{x}(X_{\mathcal{H}}, \gamma)$ sia uguale a prenderne il differenziale dell'hamiltoniano $\dd_{x}{\mathcal{H}}(\gamma)$. Se il sistema parte da uno stato iniziale $(x_0, \dot{x}^*_0)$, il suo stato percorre la linea di flusso di $X_{\mathcal{H}}$ che parte da $(x_0, \dot{x}^*_0)$. L'evoluzione di una qualsiasi funzione di stato $F$ è data dalla parentesi di Poisson $\{F, \mathcal{H}\}$.

Se lo spazio delle fasi ammette un'azione hamiltoniana $\Phi_g$ di un qualche gruppo di Lie $G$, la mappa momento avrà il valore $p_0$ fissato dalle condizioni iniziali lungo tutta l'evoluzione del sistema, e la dinamica di questo nello spazio delle fasi ridotto $\mu^{-1}(p_0)/G_{p_0}$ potrà essere studiata grazie al teorema di riduzione simplettica. 

\section{Esempio: moto del corpo rigido}

