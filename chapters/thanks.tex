\chapter*{Ringraziamenti} \addcontentsline{toc}{chapter}{Ringraziamenti}
Una laurea triennale non è tanto una fine quanto un punto di snodo. Ciò non toglie che a questo punto di snodo non avrebbe avuto senso arrivare senza le persone che ho avuto accanto in questi tre anni e, ancora prima, lungo tutto il mio cammino. La fisica è umana, come tutto: varietà differenziabili e forme simplettiche non hanno ragione di esistere senza le persone.

Ringrazio per prima cosa i miei genitori, Elena e Luca, che lungo tutta la mia vita mi hanno sostenuto e indirizzato con il loro amore, i loro consigli e il loro esempio. Ringrazio i miei nonni, Adriana, Anna e Novello, per l'aiuto e l'amore che hanno dato sia a loro che a me.

Ringrazio Sara, che cammina instancabilmente con me ormai da quattro anni, per il suo amore sincero e profondo, per la sua immancabile presenza e per aver contribuito in maniera insostituibile alla persona che ora sono.

Ringrazio Lorenzo, Claudio, Luca, Lucia, Marco, Piero e Viola, per la loro amicizia da che io sono io. Ringrazio Emma e Rebecca, per questi tre anni. Ringrazio Camilla, per aver condiviso con me l'esperienza di Groningen dandomi aiuto e una possibilità di confronto. Ringrazio Margherita, per avermi aiutato a metabolizzare e fissare Groningen. Ringrazio Alessandro, Andrea, Beatrice, Camilla, Fabio, Francesco, Giovanni, Lorenzo, Lorenzo, Maria, Riccardo, Simone e Veronica, che hanno condiviso con me questa triennale.

Ringrazio i collegiali, troppi per nominarvi tutti: vivere con voi, con ognuno di voi nella sua unicità comprensiva di uno sfuggente qualcosa che ci accomuna, ha reso veramente unici questi tre anni. Ringrazio in particolare il primo piano, che con me ha condiviso pasti, tisane e pensieri: Alberto, Caterina, Chiara, Elisa, Francesco, Francesco, Gaetano, Giulia, Giulia, Lorenzo, Luca, Marco, Nicolò e Shivani.

Ringrazio, nonostante il tempo trascorso, chi mi ha indirizzato su questa strada, che mi sembra fatta su misura: prof. Fertili, prof. Targa e prof. Brognara, ma anche e soprattutto Andrea, Benedetta, Edoardo, Francesco, Giulia, Leonardo, Luca, Pietro e Vanessa, che per primi mi hanno mostrato il lato umano di questa disciplina.

Ringrazio infine il mio relatore, prof. Latini, per la disponibilità, l'interesse e la competenza che mi ha mostrato durante la redazione di questa tesi.