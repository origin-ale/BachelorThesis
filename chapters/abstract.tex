\patchcmd{\abstract}{\titlepage}{\thispagestyle{empty}}{}{}
\patchcmd{\endabstract}{\endtitlepage}{}{}{}

\begin{abstract}
La geometria simplettica è una branca della geometria differenziale che si propone di generalizzare le proprietà geometriche, ovvero indipendenti dalle coordinate, del moto dei sistemi meccanici. Questo elaborato si propone di esplorarne le radici fisiche e dare un esempio delle sue applicazioni in tale campo. Si passano in rassegna, ponendole su base formale, le formulazioni newtoniana, lagrangiana ed hamiltoniana della meccanica classica, e viene data una formulazione geometrica della meccanica di sistemi non vincolati. Viene poi esposta la teoria delle varietà differenziabili e si definiscono vettori, forme differenziali e operazioni su di essi. In seguito, si definisce la struttura simplettica canonica dei fibrati cotangenti e si espongono alcune sue proprietà, si introduce la teoria di gruppi e algebre di Lie, si definisce la mappa momento, che associa alle simmetrie di un sistema quantità conservate nella sua evoluzione, e si enuncia il teorema di riduzione simplettica, che consente di sfruttare queste quantità conservate per ridurre la dimensionalità del problema del moto. Infine la meccanica hamiltoniana viene riformulata nel linguaggio della geometria simplettica e il teorema di riduzione simplettica viene applicato all'analisi del moto del corpo rigido libero.
\end{abstract}
\selectlanguage{english}
{\itshape \begin{abstract}
Symplectic geometry is a branch of differential geometry which aims to generalize the geometric (coordinate-independent) properties of the motion of physical systems. This work aims to explore its physical roots and give an example of its applciations in this field. The newtonian, lagrangian and hamiltonian formulations of classical mechanics are reviewed and put on a formal basis. A geometrical formulation of the mechanics of unconstrained systems is given. Subsequently, the theory of differential manifolds is explored and vectors, differential forms and operations on these objects are defined. The canonical symplectic structure of cotangent bundles is then defined and some of its properties are reviewed. The theory of Lie groups and associated algebras is introduced. The moment map is defined, providing a way to associate symmetries of a system to quantities conserved in its evolution. The symplectic reduction theorem is formulated, allowing one to exploit these conserved quantities to reduce the dimensionality of the motion problem. Finally, hamiltonian mechanics is reformulated in the language of symplectic geometry. The symplectic reduction theorem is applied to the analysis of the motion of the free rigid body.
\end{abstract}}
\selectlanguage{italian}