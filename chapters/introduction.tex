\chapter*{Introduzione} \addcontentsline{toc}{chapter}{Introduzione}
Il formalismo lagrangiano della meccanica classica costituisce un vero e proprio cambio di paradigma rispetto a quello newtoniano. Esso infatti non si limita a sostituire la celeberrima legge $\vec{F}=m\vec{a}$ di Newton con il principio di minima azione, ma riduce al minimo la presenza dei vettori, entità matematiche principi della meccanica newtoniana. Obiettivo principale di questa operazione è la covarianza: a prescindere dal sistema di coordinate impiegato, siano esse cartesiane, sferiche o iperboliche, le equazioni di Eulero-Lagrange ottenute dal principio di minima azione avranno la stessa forma. Più simmetria e ancora maggiore libertà di scelta di coordinate per semplificare i calcoli si hanno passando al formalismo hamiltoniano, duale di quello lagrangiano. Il prezzo di questa potenza è però un pesante apparato computazionale che spesso offusca l'eleganza del formalismo.

Fortunatamente, questo prezzo può essere permutato. Gli strumenti della geometria differenziale, e in particolare le teorie delle varietà differenziabili, delle forme differenziali e dei gruppi di Lie, consentono di formulare la meccanica hamiltoniana in modo non solo covariante, ma addirittura del tutto libero dall'uso di coordinate e puramente \emph{geometrico}. La difficoltà viene così trasferita da un'ottusa complessità dei calcoli a una certa sottigliezza nella formulazione, dall'utilizzo degli strumenti alla loro definizione. La difficoltà concettuale che risulta da questo scambio è non solo notevolmente più soddisfacente da districare --- aspetto in ogni caso da non sottovalutare --- ma presenta anche notevoli vantaggi teorici e pratici. Dal punto di vista matematico, lo studio della \emph{geometria simplettica}, che generalizza le proprietà geometriche degli spazi delle fasi di sistemi meccanici, si è ormai affrancato dalla sua origine fisica e costituisce un fertile ambito di ricerca. Per quanto riguarda le applicazioni fisiche, invece, il formalismo simplettico è più semplice da espandere oltre la meccanica classica rispetto al newtoniano, e ha le stesse capacità di sfruttare le simmetrie di un sistema per ridurre la gravosità dei calcoli che rende l'hamiltoniano così utile. Quest'ultimo punto è un processo di riduzione della dimensione dello spazio delle fasi noto come \emph{riduzione simplettica}, che si realizza tramite la \emph{mappa momento}, uno strumento che consente di associare a ogni simmetria continua di un sistema una quantità conservata.

Oltre agli impieghi diretti, la formulazione simplettica della meccanica classica costituisce un'opportunità di familiarizzare con concetti e strumenti che sono ormai onnipresenti nella fisica moderna, primi tra tutti quelli di \emph{varietà differenziabile}, fondamento della relatività generale, e \emph{gruppo di Lie} --- anche specificamente nel suo impiego per la formalizzazione delle simmetrie --- che costituisce invece uno degli elementi chiave del Modello Standard della fisica delle particelle. La teoria delle varietà costituisce inoltre un caso esemplare di impiego dell'astrazione matematica, nel suo separare concetti che nello spazio euclideo tridimensionale coincidono e nell'identificarne altri tra cui sussistono relazioni particolari.

Obiettivo di questo elaborato è giungere a una formulazione geometrica, in cui i concetti sono cioè definiti senza l'utilizzo di coordinate, della meccanica classica, attraverso la geometria simplettica definita canonicamente sui fibrati cotangenti delle varietà differenziabili, e definire il processo di riduzione simplettica che sfrutta le simmetrie di un sistema fisico per ridurre la dimensione dello spazio delle fasi trovando quantità conservate. Nel primo capitolo verrà fornita una descrizione matematica dell'universo e dei moti dei corpi in esso e verranno velocemente passate in rassegna le formulazioni della meccanica classica trattate nel corso triennale di Fisica: newtoniana, lagrangiana, hamiltoniana. I successivi due capitoli si concentreranno sul fornire gli strumenti matematici necessari per la formulazione simplettica. Nel secondo capitolo si definiranno e si daranno gli strumenti per trattare varietà differenziabili e loro atlanti di coordinate, vettori e spazi tangenti e cotangenti, e forme differenziali che generalizzano questi ultimi. Nel terzo capitolo si forniranno gli strumenti più specifici alla meccanica simplettica: per prima cosa la forma simplettica stessa, per poi passare a gruppi e algebre di Lie e al teorema di riduzione simplettica. Nel quarto capitolo si applicheranno alla meccanica gli strumenti acquisiti, e come esempio di utilizzo della riduzione simplettica sarà analizzato il moto del corpo rigido libero.