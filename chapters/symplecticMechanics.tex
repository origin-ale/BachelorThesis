\chapter{Varietà simplettiche}
In questo capitolo si userà la teoria delle varietà differenziabili per costruire gli strumenti che saranno poi impiegati nella formulazione simplettica della meccanica. Si definirà una \emph{struttura simplettica} canonica sui fibrati cotangenti, che consentirà di associare un campo vettoriale a ogni funzione definita su di essi. Si formalizzeranno poi le simmetrie possedute da una data varietà tramite il concetto di \emph{azione di un gruppo di Lie}. Infine, si vedrà che i gradi di libertà su cui vale una simmetria possono essere eliminati tramite un processo noto come \emph{riduzione simplettica}.

\section{Struttura simplettica}
% \symplecticForm*

\begin{restatable}{definition}{symplecticForm}
  Si dice \dfn{forma simplettica} una $2$-forma differenziale $\omega$ chiusa e non degenere. Si dice \dfn{varietà simplettica} una varietà $2n$-dimensionale su cui è definita una forma simplettica.
\end{restatable}
Su ogni fibrato cotangente è definita una forma simplettica grazie alla proiezione $\pi$ e ai covettori $\dot{x}^*$ che formano le fibre.
\begin{definition}
  Sia $M$ una varietà $n$-dimensionale e sia $X = T^*M$ il suo fibrato cotangente, con proiezione $\pi:X\to M$. Si dice \dfn{$1-$forma tautologica} $\lambda$ su $X$ la $1$-forma differenziale tale che \begin{equation*}
  \lambda_{(x, \dot{x}^*)}(\gamma) = \dot{x}^* \bigg[\big(D_{(x, \dot{x}^*)} \pi\big)(\gamma)\bigg]
  \end{equation*} 
  dove $x \in M$, $\dot{x}^* \in T_x^* M$ e $\gamma \in T_{(x,\dot{x}^*)}X$.
\end{definition}
\begin{remark}
  Intuitivamente, $D_{(x,\dot{x}^*)} \pi$ rimuove le componenti della velocità nello spazio delle fasi $\gamma$ che sono lungo le direzioni cotangenti, lasciando solo la velocità nello spazio delle configurazioni. La $1$-forma tautologica misura poi la lunghezza di questa velocità lungo una direzione data da $\dot{x}^*$. 
\end{remark}

\begin{definition}
  Sia $X$ un fibrato cotangente, si dice \dfn{$2$-forma canonica} $\omega$ l'opposto della derivata esterna della forma tautologica: \begin{equation}
  \omega = -\dd{\lambda}
  \end{equation} 
\end{definition}
\begin{remark}
  Per definizione di derivata esterna, la forma simplettica $\omega(\gamma_1,\gamma_2)$ misura la circuitazione di $\lambda$ sul rettangolo che ha per lati $\gamma_1$ e $\gamma_2$. 
\end{remark}
Si dimostra che 
\begin{theorem}
  La forma canonica su un fibrato cotangente è una forma simplettica.
\end{theorem}

Se $M = \R^{n}$, il suo fibrato cotangente $X$ è isomorfo a $\R^{2n}$ tramite le coordinate $(q_1, \ldots, q_n, p_1, \ldots, p_n)$. In tal caso quindi $\pi: (q_1, \ldots, q_n, p_1, \ldots, p_n) \mapsto  (q_1, \ldots, q_n)$. Quindi, sia $\overline{\gamma} = (\overline{\gamma}_1, \ldots, \overline{\gamma}_{2n}):\R \to X$ un cammino con $\overline{\gamma}(0) = x \in X$, si ha \begin{equation*}
(\pi \circ \overline{\gamma}) (t) = \big(\overline{\gamma}_1(t), \ldots, \overline{\gamma}_n (t)\big)
\end{equation*} 
Ciò significa che la derivata di $\pi$ manda vettori $\gamma = (\gamma_1, \ldots, \gamma_{2n})$ tangenti a $\R^{2n}$ in vettori tangenti a $\R^{n}$ proiettandoli sulle loro prime $n$ coordinate:
\begin{equation*}
  D_{(\vec{q},\vec{p})}\pi: (\gamma_1, \ldots, \gamma_{2n}) \mapsto  (\gamma_1, \ldots, \gamma_n)
\end{equation*}
e, siccome $\dot{x}^*(\gamma) = \sum_j p_j \gamma_j$ (è l'applicazione di un vettore cotangente a uno tangente, scritta in coordinate) e per definizione delle coordinate $\gamma_j = (\dd{q}_j)(\gamma)$, si ha \begin{equation*}
\lambda_{(\vec{q},\vec{p})} (\gamma) = \sum_{j=1}^n p_j \gamma_j = \left[\sum_{j=1}^n p_j \dd{q_j}\right] (\gamma)
\end{equation*} 
da cui segue che \begin{equation}
  \lambda_{(\vec{q},\vec{p})} = \sum_{j=1}^n p_j \dd{q_j}
\end{equation} 
La forma simplettica su $\R^{2n}$ nelle coordinate $(q_1, \ldots, q_n)$ è quindi data dal \autoref{thm:derivCoords}: 
\begin{equation} \label{eq:R2nSympForm}
\omega = \sum_{j=1}^n \dd{q_i} \wedge \dd{p_i}
\end{equation} 
Si noti che questa forma ha coefficienti costanti su $\R^{2n}$, ed è quindi rappresentabile nella base $e_i$ come matrice quadrata $2n \times 2n$ con elementi dati da 
\begin{equation*}
\omega(e_i, e_j) = \sum_{k=1}^{n} \dd{q_k} \wedge \dd{p_k}(e_i, e_j) = \sum_{k=1}^{n} \det\pmqty{q_k(e_i) & q_k(e_j) \\ p_k(e_i) & p_k(e_j)}
\end{equation*}
ovvero, nella base con coordinate $(q_1, \ldots, q_n, p_1, \ldots, p_n)$, dalla matrice a blocchi
\begin{equation*}
- \mathsf{J} = - \pmqty{0 & -\mathsf{I} \\ \mathsf{I} & 0}
\end{equation*} 
tramite l'usuale rappresentazione delle $2$-forme, $\omega(\gamma_1, \gamma_2) = \gamma_1^T (-\mathsf{J}) \gamma_2$.

Si può dimostrare che la forma della forma simplettica in $\R^{2n}$ è comune a tutte le varietà simplettiche. Questo risultato è noto come \emph{teorema di Darboux}.

\begin{definition}
  Siano $(X, \omega_X)$ e $(Y, \omega_Y)$ due varietà simplettiche. Si dice \dfn{simplettomorfismo} un diffeomorfismo $F:X\to Y$ tale che $F^*(\omega_Y) = \omega_X$. 
\end{definition}
\begin{remark}
  Se $X = Y$ e $\omega_X = \omega_Y$, $F$ si dice \dfn{trasformazione canonica}.
\end{remark}
\begin{theorem}[Darboux]
  Sia $(X,\omega)$ una varietà simplettica $2n$-dimensionale. Allora per ogni $x \in X$ esiste una carta $(U,\phi)$ con $x \in U$ tale che $\phi$ è una trasformazione canonica tra $(X,\omega)$ e $(\R^{2n}, \Omega)$ dove $\Omega$ è la forma simplettica canonica su $\R^{2n}$ data dall'\autoref{eq:R2nSympForm}.
\end{theorem}
\begin{remark}
  Nelle coordinate individuate dal teorema di Darboux, la forma simplettica $\omega(\gamma_1,\gamma_2)$ è la somma delle proiezioni sui piani $(q_i, p_i)$ del rettangolo formato da $\gamma_1$ e $\gamma_2$.
\end{remark}

Una struttura simplettica su una varietà consente di associare un campo vettoriale a ogni funzione a valori reali definita su di essa. 
\begin{definition} \label{def:hamField}
  Sia $(X, \omega)$ una varietà simplettica. Si dice \dfn{hamiltoniana} una funzione differenziabile $H: X \to \R$. Si dice \dfn{campo vettoriale hamiltoniano} il campo vettoriale $X_H$ definito su $X$ e tale che per ogni $x \in X$, per ogni $\gamma \in T_x X$
  \begin{equation} \label{eq:hamField}
    \omega\big((X_H)_x, \gamma\big) = \dd{H}_x(\gamma)
  \end{equation}
  La mappa $H \mapsto X_H$ si dice \dfn{gradiente simplettico}.
\end{definition}
\begin{remark}
  La corrispondenza tra l'$1$-forma $\dd{H}_x$ e il vettore $(X_H)_x$ è analoga a quella del teorema di rappresentazione di Fischer-Riesz, in cui però il prodotto scalare è sostituito con la forma simplettica.
\end{remark}
\begin{remark}
  Il nome di gradiente simplettico è dovuto al fatto che la mappa svolge un ruolo simile a quello del gradiente ordinario $\grad$, ovvero consente di ottenere un campo vettoriale da una funzione.
\end{remark}

L'insieme delle funzioni reali differenziabili $f:X \to \R$ definite su una varietà simplettica $(X, \omega)$ forma uno spazio vettoriale con le operazioni di somma e prodotto per scalare punto per punto. Grazie al gradiente simplettico, questo insieme ha tuttavia una struttura ulteriore, quella di \emph{algebra di Lie}, ovvero ammette un'operazione analoga al commutatore fra matrici.
\begin{definition}
  Sia $X$ un insieme e $[\cdot , \cdot]:X \times X \to X$ un'operazione denotata con $(x,y) \mapsto [x,y]$. La struttura $\big(X, [\cdot ,\cdot ]\big)$ si dice \dfn{algebra di Lie} se valgono le seguenti proprietà: \begin{enumerate}
    \item \dfn{bilinearità}: $[\lambda x + \mu y, z] = \lambda [x, z] + \mu[y,z]$ per ogni $x,y,z \in X$ e $\lambda, \mu \in  \R$
    \item \dfn{antisimmetria}: $[x,y] = -[y,x]$ per ogni $[x,y] \in X$
    \item \dfn{identità di Jacobi}: $\big[x,[y,z]\big] + \big[z,[x,y]\big] + \big[y,[z,x]\big] = 0$
  \end{enumerate}
\end{definition}

\begin{definition}
  Sia $(X, \omega)$ una varietà simplettica e siano $H,K: X \to \R$ due funzioni differenziabili con campi vettoriali hamiltoniani $X_H$ e $X_K$, si dice \dfn{parentesi di Poisson} di $H$ e $K$ \begin{equation*}
  \{H,K\} = \omega(X_H,X_K) 
  \end{equation*} 
\end{definition}
\begin{theorem}
  Sia $(X, \omega)$ una varietà simplettica, sia $\mathcal{X} = \mathcal{C}^{\infty}(X,\R)$ lo spazio delle funzioni reali differenziabili su $X$ e siano $\{\cdot , \cdot \}$ le parentesi di Poisson su $X$. Allora $\big(\mathcal{X}, \{\cdot , \cdot\}\big)$ è un'algebra di Lie.
\end{theorem}

Per una data funzione $H:X\to \R$, le parentesi di Poisson forniscono la variazione di qualsiasi altra funzione $K$ lungo le linee di flusso di $X_H$.
\begin{theorem}
  Sia $(X, \omega)$ una varietà simplettica, sia $H$ un'hamiltoniana e sia $\overline{x}:t \mapsto \overline{x}(t)$ una linea di flusso del suo campo hamiltoniano $X_H$. Allora per ogni funzione differenziabile $K:X\to \R$ vale \begin{equation*}
  \dv{t} K\big(\overline{x}(t)\big) = \{K,H\} 
  \end{equation*}  
\end{theorem}
\begin{corollary}
  Se $\{K, H\} = 0$, $K$ si conserva lungo le linee di flusso. In tal caso, $K$ è detto \dfn{integrale del moto}.
\end{corollary}

La struttura simplettica consente di associare campi vettoriali a funzioni differenziabili. In maniera simile, l'algebra delle funzioni data dalle parentesi di Poisson può essere vista come il risultato, tramite la forma simplettica, di una struttura algebrica di Lie valida per i \emph{campi vettoriali} definiti su una varietà \emph{qualsiasi}, non necessariamente simplettica. Questa struttura è detta \emph{parentesi di Lie} ed è strettamente legata al concetto di derivata di Lie.
\begin{definition}
  Siano $X$ una varietà, $V$ un campo vettoriale su $X$ e $\alpha$ una $k$-forma differenziale su $X$. Si dice \dfn{derivata di Lie} $L_V\alpha$ la $k$-forma differenziale data per ogni $x \in X$ da \begin{equation*}
  (L_V \alpha)_x = \eval{\dv{t}}_{t=0} \big( (\Phi_V^t)^* \alpha_{\Phi_V^t(x)}\big)_x
  \end{equation*} 
  dove $\Phi_V$ è il flusso di $V$ e $\dv{t}$ agisce nello spazio vettoriale delle $k$-forme su $T_x X$.
\end{definition}
\begin{remark}
  La derivata di Lie di $\alpha$ è la derivata temporale della forma che si vede seguendo una linea di flusso, valutata in $t=0$.
\end{remark}
\begin{definition}
  Siano $X$ una varietà e $V,W$ campi vettoriali su $X$. Si dice \dfn{parentesi di Lie} di $V$ e $W$ il campo vettoriale dato per ogni $x \in X$ da \begin{equation*}
  [V,W]_x = - \eval{\dv{t}}_{t=0} \Big(\big(D_{\Phi_V^t(x)} (\Phi_V^t)^{-1}\big)\big(W_{\Phi_V^t(x)}\big) \Big)_x
  \end{equation*}
  dove $\Phi_V$ è il flusso di $V$ e $\dv{t}$ agisce nello spazio vettoriale $T_x X$.
\end{definition}
\begin{remark}
  La parentesi di Lie di due campi vettoriali può anche essere interpretata come la differenza infinitesima fra il punto che risulta seguendo $V$ per un tempo $\epsilon$ e poi $W$ per $\epsilon$ e quello che risulta seguendo $W$ per $\epsilon$ e poi $V$ per $\epsilon$. In alternativa, può anche essere interpretata come commutatore delle derivazioni lungo $V$ e lungo $W$.
\end{remark}
\begin{theorem}
  Sia $(X, \omega)$ una varietà simplettica, siano $\{\cdot , \cdot \}$ le parentesi di Poisson e $[\cdot , \cdot ]$ quelle di Lie su di essa. Allora il gradiente simplettico $H \mapsto X_H$ è un omomorfismo di algebre di Lie da $\big(\mathcal{X}, \{\cdot , \cdot \} \big)$ a $\big(\mathrm{Vec}\, X, [\cdot , \cdot]\big)$, ovvero se $H,K \in \mathcal{X}$ e $X_H, X_K$ sono i loro campi vettoriali hamiltoniani \begin{equation*}
  X_{\{H,K\}} = [X_H, X_K]
  \end{equation*} 
\end{theorem}

\section{Gruppi di Lie e algebre associate}
Un gruppo di Lie è sostanzialmente \textquotedblleft un gruppo che è anche una varietà\textquotedblright, con opportune richieste sulle applicazioni di gruppo.
\begin{definition}
  Si dice \dfn{gruppo di Lie} una varietà $G$ con due applicazioni differenziabili $\cdot: G \times G \to G$ e ${}^{-1}: G\to G$, dette rispettivamente \dfn{moltiplicazione} e \dfn{inversione}, e un punto $e \in G$ detto \dfn{identità}, tali che:
  \begin{enumerate}
    \item $\cdot $ sia \dfn{associativa}: per ogni $g,f,h \in G$ vale $g\cdot (f\cdot h) = (g\cdot f) \cdot h$.
    \item $e$ sia l'identità destra e sinistra: per ogni $g \in G$ vale $g \cdot e = e \cdot g = g$.
    \item per ogni $g \in G$, $g^{-1}$ sia l'\dfn{inverso} di $g$: $g \cdot g^{-1} = g^{-1}\cdot g = e$.
  \end{enumerate}
\end{definition}

I gruppi consentono di modellizzare matematicamente le \emph{simmetrie} di una varietà, tramite il concetto di azione.
\begin{definition}
  Sia $X$ una varietà e sia $G$ un gruppo di Lie. Si dice \dfn{azione sinistra} di $G$ su $X$ un'applicazione differenziabile $\Phi:(g,x) \in G \times X \mapsto \Phi_g(x) \in X$ tale che:
  \begin{enumerate}
    \item l'azione di $e$ sia la mappa identità: $\Phi_e (x) = x$ per ogni $x \in X$.
    \item azione e moltiplicazione siano associative fra loro: $\Phi_{g\cdot h} (x) = \Phi_g \big(\Phi_h (x)\big)$ per ogni $g,h \in G$ e $x \in X$.
  \end{enumerate}
Se esiste una tale mappa, si dice che $X$ ha \dfn{simmetria continua} sotto $G$. 
\end{definition}
\begin{remark}
  Il nome di simmetria è dovuto al fatto che per definizione la mappa di azione è \emph{interna} a $X$: l'azione di $G$ non \textquotedblleft genera\textquotedblright nuovi punti di $X$, l'insieme resta lo stesso.
\end{remark}
\begin{remark}
  Le \dfn{azioni} di un gruppo di Lie $G$, $\Phi_g: x \mapsto g\cdot x$ con $g \in G$, sono diffeomorfismi per definizione.
\end{remark}

Dato che un gruppo di Lie è esso stesso una varietà, un gruppo può agire su se stesso. Ciò è possibile in vari modi.

\begin{definition}
  Sia $G$ un gruppo di Lie. Si dice \dfn{azione diretta} di $G$ su se stesso l'applicazione data da $\Phi^{\text{dir}}_g(h) = g \cdot h$ per ogni $g,h \in X$.
\end{definition}
\begin{definition}
  Sia $G$ un gruppo di Lie, si dice \dfn{azione aggiunta} di $G$ su se stesso l'applicazione data da $\Phi^{\dagger}_g(h) = g \cdot h \cdot g^{-1}$ per ogni $g,h \in X$.
\end{definition}
\begin{remark} \label{rem:adjIdentity}
  L'azione aggiunta di un elemento $g$ di un gruppo di Lie $G$ manda l'identità in se stessa, siccome \begin{equation*}
  \Phi^{\dagger}_g e = g\cdot e\cdot g^{-1} = g\cdot g^{-1} = e
  \end{equation*} 
\end{remark}

Siccome un gruppo di Lie è anche una varietà, su di esso sono definiti spazi tangenti e derivate. In particolare, se $\Phi^{\dagger}_g$ è l'azione aggiunta di $g \in G$, per l'\autoref{rem:adjIdentity} si ha che $D_e \Phi^{\dagger}_g: T_e G \to T_e G$. Quindi esiste un'azione naturale di $G$ sul suo spazio tangente nell'identità.
\begin{definition}
  Si dice \dfn{azione aggiunta} di un gruppo di Lie $G$ sul suo spazio tangente nell'identità $T_e G$ la derivata dell'azione aggiunta di $G$ su se stesso $\mathrm{Ad}_g = D_e \Phi_g$, per ciascun $g \in G$. 
\end{definition}

Proprio lo spazio tangente nell'identità consente di associare a ogni gruppo di Lie un'algebra di Lie, tramite l'algebra di Lie dei campi vettoriali che è definita su ogni varietà. A ogni elemento $\gamma$ dello spazio tangente a $G$ nell'identità si può infatti associare un campo vettoriale su tutto $G$, sfruttando l'azione di $G$ per fare il push-forward di $\gamma$ su tutti gli spazi tangenti.
\begin{definition}
  Siano $G$ un gruppo di Lie, $e \in G$ la sua identità e $\gamma \in T_e G$. Si dice \dfn{campo associato} a $\gamma$ il campo vettoriale $V_\gamma$ su $G$, dato per ogni $g \in G$ da \begin{equation*}
  \big(V_\gamma\big)_g = \big(D_e \Phi^{\text{dir}}_g\big) (\gamma)
  \end{equation*} 
  dove $\Phi^{\text{dir}}_g$ è l'azione diretta di $G$ su se stesso.
\end{definition}
\begin{remark}
  Questo campo vettoriale può essere visto come il campo degli spostamenti infinitesimi di ciascun elemento di $G$, se essi vengono ottenuti come $g=\Phi^{\text{dir}}_g(e)$ e l'identità viene spostata infinitesimamente in modo dato da $\gamma$.
\end{remark}
\begin{definition}
  Siano $G$ un gruppo di Lie, $e \in G$ la sua identità e $\gamma, \delta \in T_e G$. Si dice \dfn{parentesi di Lie} dei due vettori tangenti $\gamma$ e $\delta$ il vettore \begin{equation*}
  [\gamma,\delta] = [V_{\gamma}, V_{\delta}]_e
  \end{equation*} 
  dove la parentesi al secondo membro è la parentesi di Lie di campi vettoriali su $X$.
\end{definition}
\begin{remark}
  Per le matrici, la parentesi di Lie corrisponde all'effettuare una rotazione infinitesima $\gamma$ seguita da $\delta$, per poi invertire $\delta$ e poi $\gamma$. Questo non è altro che il commutatore.
\end{remark}
\begin{definition}
  Sia $G$ un gruppo di Lie e $e$ la sua identità, si dice \dfn{algebra di Lie} $\mathfrak{g}$ di $G$ l'algebra di Lie $\big(T_e G, [\cdot ,\cdot ]\big)$.
\end{definition}
\begin{remark}
  Spesso verrà usato il simbolo $\mathfrak{g}$ anche per indicare $T_e G$.
\end{remark}
\begin{theorem}
  L'algebra di Lie $\mathfrak{g}$ di un gruppo di Lie è sempre a dimensione finita.
\end{theorem}

Grazie allo stretto legame fra le definizioni delle parentesi di Lie nei due casi, le algebre dei vettori tangenti nell'identità e dei campi vettoriali sono in realtà omomorfe.

\begin{definition} \label{def:infmAction}
  Sia $G$ un gruppo di Lie e sia $X$ una varietà di cui $G$ fornisce una simmetria continua con l'azione $\Phi$. Si dice \dfn{azione infinitesima} di $G$ su $X$ la seguente mappa $\phi: \mathfrak{g} \to \mathrm{Vec}\, X$. Per ogni $[\gamma] \in T_e G$, si scelga un rappresentante della classe di equivalenza $\gamma:\R\to G$ con $\gamma(0) = e$. L'azione infinitesima associa a $[\gamma]$ un campo vettoriale su $X$ dato dalle classi di equivalenza dei cammini su $X$
  \begin{equation*}
  \big(\phi_{\gamma}\big)_x = \Phi_{\gamma(t)} (x)
  \end{equation*}
  Se esiste una tale mappa, si dice che $X$ ha una \dfn{simmetria infinitesima} data da $G$.
\end{definition}
\begin{theorem}
  La mappa $\phi: \mathfrak{g} \to \mathrm{Vec}\, X$ definita nella \autoref{def:infmAction}, è un omomorfismo tra $\mathfrak{g}$ e l'algebra dei campi vettoriali su $X$.
\end{theorem}
\begin{definition}
  La mappa $\big(t, [\gamma]\big) \in \R \times  \mathfrak{g} \mapsto g \in G$ che costituisce il cammino $\gamma$ nella \autoref{def:infmAction} è detta \dfn{mappa esponenziale}, e si denota $\exp(t \gamma)$, se vale $\exp\big((t+s) \gamma\big) = \exp(t \gamma) \cdot  \exp (s \gamma)$ per ogni $t,s \in \R$.
\end{definition}
\begin{remark}
  Questa nomenclatura generalizza il caso in cui $G$ sia un gruppo di Lie composto da matrici.
\end{remark}

Un gruppo di Lie $G$ con opportune proprietà può essere usato per rimuovere informazioni ridondanti dai punti di una varietà $X$, sfruttando la simmetria a cui esso è associato.
\begin{definition}
  Sia $X$ una varietà e $G$ un gruppo di Lie. Un'azione di $G$ su $X$ si dice \dfn{libera} se nessuna azione di un elemento $g \in G$ diverso dall'identità ha punti fissi.
\end{definition}
\begin{definition}
  Siano $G$ un gruppo di Lie e $X$ una varietà su cui esso agisce secondo $\Phi$. Si dice \dfn{orbita} di $x \in X$ l'insieme $Gx = \{ \Phi_g(x)\mid g \in G\}$. Si dice \dfn{fetta} dell'azione di $G$ in $p \in Gx$ l'insieme $S_{\epsilon}(p) = \phi^{-1}\big(\phi(Gx)^{\perp} \cap B_{\epsilon}(0)\big)$ dove $(U,\phi)$ è una carta su $X$ e $\phi(Gx)^{\perp}$ è il complemento ortogonale dell'immagine di $Gx$ attraverso la carta.
\end{definition}
\begin{theorem}
  Sia $X$ una varietà e $G$ un gruppo di Lie compatto che agisce liberamente su $X$ secondo $\Phi$. Allora l'insieme quoziente $X/G$ definito dalla relazione di equivalenza $x \sim \Phi_g(x)$, con $x \in X$ e $g \in G$, ha una struttura naturale di varietà differenziabile, tale che l'applicazione $\pi:X \to X /G$ che manda un punto nella sua classe di equivalenza sia differenziabile. Le carte sono le carte $(U/G,\phi)$ dove $U = G\big(S_{\epsilon}(p)\big)$ con $p \in X$ e $\phi$ è una carta su $S_{\epsilon}(p)$, che è una sottovarietà di $X$.
\end{theorem}

\section{Riduzione simplettica}
Su una varietà generica un gruppo può agire in qualsiasi modo. Perché l'azione di un gruppo su una varietà simplettica abbia senso, è invece necessario che l'azione del gruppo non cambi la struttura simplettica.
\begin{definition}
  Siano $G$ un gruppo di Lie e $(X, \omega)$ una varietà simplettica. L'azione $\Phi$ di $G$ su $X$ si dice \dfn{simplettica} se ogni azione $\Phi_g:X\to X$, con $g \in G$, è una trasformazione canonica.
\end{definition}

Fra le azioni simplettiche di un gruppo, esiste una sottoclasse con particolare rilevanza fisica, detta delle azioni \emph{hamiltoniane}. Queste sono le azioni date dal flusso di un'hamiltoniana definita su tutta la varietà $X$ e in un certo senso compatibili con la struttura di $G$. La mappa momento implica l'entrata in scena di $\mathfrak{g}^*$, il duale dell'algebra di Lie. Il gruppo $G$ ha un'azione naturale, oltre che sull'algebra stessa, anche sul duale.
\begin{definition}
  Siano $G$ un gruppo di Lie e $\mathfrak{g}$ la sua algebra associata. Si dice \dfn{azione coaggiunta} di $g \in G$ su $\mathfrak{g}^*$ l'applicazione $\mathrm{Ad}_g^*$ che manda $\alpha \in \mathfrak{g}^*$ in $\mathrm{Ad}^*_g(\alpha) \in \mathfrak{g}^*$ definito da
  \begin{equation*}
  \big[\mathrm{Ad}^*_g(\alpha)\big](\gamma) = \alpha(\mathrm{Ad}_{g^{-1}}\gamma)
  \end{equation*}
  per ogni $\gamma \in \mathfrak{g}$, dove $\mathrm{Ad}$ indica l'azione aggiunta di $G$ su $\mathfrak{g}$.
\end{definition}

La condizione di compatibilità dell'azione con la struttura di $G$ può ora essere formulata più precisamente. 
\begin{definition}
  Siano $(X, \omega)$ una varietà simplettica e $G$ un gruppo di Lie che vi agisce simpletticamente secondo le mappe $\Phi_g$, con $g \in G$. L'azione $\Phi$ di $G$ si dice \dfn{hamiltoniana} se esiste un'applicazione $\mu: X \to \mathfrak{g}^*$ tale che  per ogni $x \in X$, per ogni $\gamma \in \mathfrak{g}$ per ogni $\xi \in T_x X$ valga
  \begin{equation*}
  \omega\big((\phi_{\gamma})_x, \xi\big) = \dd_x [\mu_x(\gamma)](\xi)
  \end{equation*} 
  e inoltre per ogni $g \in G$, per ogni $x \in X$ valga
  \begin{equation*}
  \mu\big(\Phi_g(x)\big) = \mathrm{Ad}_g^*\, \big(\mu(x)\big)
  \end{equation*} 
  In tal caso, l'applicazione $\mu$ è detta \dfn{mappa momento}.
\end{definition}
\begin{remark}
  La prima richiesta equivale al chiedere che l'azione infinitesima di ciascun $\gamma$ sia il campo vettoriale di un'hamiltoniana, data da $x \mapsto P_{\gamma}(x) = \mu_x(\gamma)$. L'hamiltoniana così ottenuta dipende linearmente da $\gamma$, ma è possibile dimostrare che non c'è perdita di generalità rispetto al caso di hamiltoniane generiche.
\end{remark}
\begin{remark}
  La seconda richiesta è che $G$ agisca su $X$ in maniera compatibile con la sua azione su $\mathfrak{g}^*$, cioè che le due operazioni formino un diagramma commutativo con la mappa momento, la quale collega gli insiemi. Questa richiesta, al contrario della prima, non può essere soddisfatta per hamiltoniane generiche.
\end{remark}
\begin{remark}
  L'applicazione $\mu^*:\mathfrak{g} \to \mathcal{C}^{\infty}(X)$ che manda $\gamma \mapsto P_{\gamma}$, detta \dfn{mappa comomento}, è un omomorfismo delle due algebre di Lie, ovvero 
  \begin{equation*}
  [\gamma,\delta] = \{P_{\gamma}, P_{\delta}\} 
  \end{equation*} 
  per ogni $\gamma,\delta \in \mathfrak{g}$.
\end{remark}

La principale rilevanza fisica delle azioni hamiltoniane sta nel fatto che esse consentono di generalizzare il teorema di Noether.

\begin{theorem}
  Siano $(X,\omega)$ una varietà simplettica e $G$ un gruppo di Lie con azione hamiltoniana $\Phi$ su di essa. Sia $\mu: X\to \mathfrak{g}^*$ la mappa momento. Sia $H: X\to \R$ una funzione hamiltoniana conservata dall'azione di $G$. Allora le funzioni $P^{\gamma}: X\to \R$ definite da $x \mapsto P^{\gamma}(x) = \mu_x(\gamma)$ sono conservate dal flusso hamiltoniano di $H$.
\end{theorem}
\begin{remark}
  L'invarianza delle $P^{\gamma}$ implica che le componenti di $\mu_x$ in una qualsiasi base di $\mathfrak{g}^*$ siano costanti. Si può quindi dire che \emph{le componenti del momento si conservano}. 
\end{remark}

Il teorema di Noether in senso stretto risulta un caso particolare dell'origine di flussi hamiltoniani.
\begin{theorem}
  Sia $Q$ una varietà e sia $G$ un gruppo di Lie che ne fornisce una simmetria continua secondo una qualche azione $\Phi$. Allora l'azione $\tilde{\Phi}$ di $G$ sul fibrato cotangente $X=T^*Q$ data per $x \in Q$ e $\alpha \in T^*_x Q$ da
  \begin{equation*}
  \tilde{\Phi}(x,\alpha) = \big(\Phi(x), (\Phi^{-1})^*(\alpha)\big)
  \end{equation*} 
  è hamiltoniana rispetto alla forma canonica.
\end{theorem}

Siccome $P_\gamma$ deve rimanere costante, un sistema che ammette un'azione hamiltoniana ha meno gradi di libertà effettivi di quanto appaia a primo impatto. È possibile rimuovere i gradi di libertà senza significato fisico tramite un processo noto come \emph{riduzione simplettica}, reso possibile dal teorema seguente.

\begin{theorem}[Marsden-Weinstein]
  Sia $(X,\omega)$ una varietà simplettica, sia $G$ un gruppo di Lie compatto con azione hamiltoniana su $X$ e sia $\mu:  X \to \mathfrak{g}^*$ la mappa momento. Sia $p \in \mathfrak{g}^*$, sia $X_p = \mu^{-1}(p)$ e sia $G_p = \{g \in G \mid \mathrm{Ad}_g^*\, p = p\}$. Se $G_p$ agisce liberamente su $M_p$ allora:
  \begin{enumerate}
    \item L'insieme $M_p/G_p$ è una varietà simplettica con forma simplettica $\omega_r$ tale che \begin{equation*}
    \pi^* \omega_r = \omega
    \end{equation*} 
    dove $\pi:M_p \to M_p/G_p$ è la proiezione sullo spazio quoziente e $\omega$ indica in realtà la \emph{restrizione} $\omega|_{X_p}$.
    \item Se $H:X\to \R$ è una funzione differenziabile invariante sotto l'azione di $G$, allora la sua restrizione a $M_p/G_p$ (il cui valore su una classe di equivalenza è dato dal valore su un rappresentante) è una funzione $H_r:M_p/G_p \to \R$ il cui campo hamiltoniano rispetto a $\omega_r$ è \begin{equation*}
    X_{H_r} = D \pi(X_H)
    \end{equation*}
    dove $X_H$ è il campo hamiltoniano di $H$.
  \end{enumerate}
\end{theorem}
\begin{remark}
  I due punti del teorema corrispondono al fatto che sia la \emph{struttura simplettica} della varietà che la sua \emph{dinamica hamiltoniana} possono essere ridotte, rimuovendo le informazioni che la simmetria rispetto a $G$ rende superflue.
\end{remark}