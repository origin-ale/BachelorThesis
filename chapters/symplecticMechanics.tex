\chapter{Meccanica simplettica}
In questo capitolo si esploreranno le proprietà di un tipo particolare di $1$-forma, detto \dfn{forma simplettica}. Una forma di questo tipo è definita naturalmente su ogni fibrato cotangente e rende possibile definire un'algebra di Lie delle funzioni su di esso tramite un'operazione detta \dfn{parentesi di Poisson}. Si passeranno poi in rassegna le formulazioni trattate nel capitolo 1, che ora potranno essere riscritte in maniera indipendente dalle coordinate. Lo spazio delle fasi in particolare risulterà essere un fibrato cotangente, e grazie alla sua struttura simplettica sarà possibile scrivere le equazioni di Hamilton in maniera indipendente dalle coordinate, completando così la formulazione geometrica della meccanica classica.

\section{Struttura simplettica}
% \symplecticForm*

\begin{restatable}{definition}{symplecticForm}
  Si dice \dfn{forma simplettica} una $2$-forma differenziale $\omega$ chiusa e non degenere. Si dice \dfn{varietà simplettica} una varietà $2n$-dimensionale su cui è definita una forma simplettica.
\end{restatable} % PLACEHOLDER FOR INDEPENDENT COMPILATION, REMOVE FOR FINAL VERSION

Su ogni fibrato cotangente è definita una forma simplettica grazie alla proiezione $\pi$ e ai covettori $\dot{x}^*$ che formano le fibre.
\begin{definition}
  Sia $M$ una varietà $n$-dimensionale e sia $X = T^*M$ il suo fibrato cotangente, con proiezione $\pi:X\to M$. Si dice \dfn{$1-$forma tautologica} $\lambda$ su $X$ la $1$-forma differenziale tale che \begin{equation*}
  \lambda_{(x, \dot{x}^*)}(\gamma) = \dot{x}^* \bigg[\big(D_{(x, \dot{x}^*)} \pi\big)(\gamma)\bigg]
  \end{equation*} 
  dove $x \in M$, $\dot{x}^* \in T_x^* M$ e $\gamma \in T_{(x,\dot{x}^*)}X$.
\end{definition}
\begin{remark}
  Intuitivamente, $D_{(x,\dot{x}^*)} \pi$ rimuove le componenti della velocità nello spazio delle fasi $\gamma$ che sono lungo le direzioni cotangenti, lasciando solo la velocità nello spazio delle configurazioni. La $1$-forma tautologica misura poi la lunghezza di questa velocità lungo una direzione data da $\dot{x}^*$. 
\end{remark}

\begin{definition}
  Sia $X$ un fibrato cotangente, si dice \dfn{$2$-forma canonica} $\omega$ l'opposto della derivata esterna della forma tautologica: \begin{equation}
  \omega = -\dd{\lambda}
  \end{equation} 
\end{definition}
\begin{remark}
  Per definizione di derivata esterna, la forma simplettica $\omega(\gamma_1,\gamma_2)$ misura la circuitazione di $\lambda$ sul rettangolo che ha per lati $\gamma_1$ e $\gamma_2$. 
\end{remark}
Si dimostra che 
\begin{theorem}
  La forma canonica su un fibrato cotangente è una forma simplettica.
\end{theorem}

Se $M = \R^{n}$, il suo fibrato cotangente $X$ è isomorfo a $\R^{2n}$ tramite le coordinate $(q_1, \ldots, q_n, p_1, \ldots, p_n)$. In tal caso quindi $\pi: (q_1, \ldots, q_n, p_1, \ldots, p_n) \mapsto  (q_1, \ldots, q_n)$. Quindi, sia $\overline{\gamma} = (\overline{\gamma}_1, \ldots, \overline{\gamma}_{2n}):\R \to X$ un cammino con $\overline{\gamma}(0) = x \in X$, si ha \begin{equation*}
(\pi \circ \overline{\gamma}) (t) = \big(\overline{\gamma}_1(t), \ldots, \overline{\gamma}_n (t)\big)
\end{equation*} 
Ciò significa che la derivata di $\pi$ manda vettori $\gamma = (\gamma_1, \ldots, \gamma_{2n})$ tangenti a $\R^{2n}$ in vettori tangenti a $\R^{n}$ proiettandoli sulle loro prime $n$ coordinate:
\begin{equation*}
  D_{(\vec{q},\vec{p})}\pi: (\gamma_1, \ldots, \gamma_{2n}) \mapsto  (\gamma_1, \ldots, \gamma_n)
\end{equation*}
e, siccome $\dot{x}^*(\gamma) = \sum_j p_j \gamma_j$ (è l'applicazione di un vettore cotangente a uno tangente, scritta in coordinate) e per definizione delle coordinate $\gamma_j = (\dd{q}_j)(\gamma)$, si ha \begin{equation*}
\lambda_{(\vec{q},\vec{p})} (\gamma) = \sum_{j=1}^n p_j \gamma_j = \left[\sum_{j=1}^n p_j \dd{q_j}\right] (\gamma)
\end{equation*} 
da cui segue che \begin{equation}
  \lambda_{(\vec{q},\vec{p})} = \sum_{j=1}^n p_j \dd{q_j}
\end{equation} 
La forma simplettica su $\R^{2n}$ nelle coordinate $(q_1, \ldots, q_n)$ è quindi data dal \autoref{thm:derivCoords}: 
\begin{equation} \label{eq:R2nSympForm}
\omega = \sum_{j=1}^n \dd{q_i} \wedge \dd{p_i}
\end{equation} 
Si noti che questa forma ha coefficienti costanti su $\R^{2n}$, ed è quindi rappresentabile nella base $e_i$ come matrice quadrata $2n \times 2n$ con elementi dati da 
\begin{equation*}
\omega(e_i, e_j) = \sum_{k=1}^{n} \dd{q_k} \wedge \dd{p_k}(e_i, e_j) = \sum_{k=1}^{n} \det\pmqty{q_k(e_i) & q_k(e_j) \\ p_k(e_i) & p_k(e_j)}
\end{equation*}
ovvero, nella base con coordinate $(q_1, \ldots, q_n, p_1, \ldots, p_n)$, dalla matrice a blocchi
\begin{equation*}
- \mathsf{J} = - \pmqty{0 & -\mathsf{I} \\ \mathsf{I} & 0}
\end{equation*} 
tramite l'usuale rappresentazione delle $2$-forme, $\omega(\gamma_1, \gamma_2) = \gamma_1^T (-\mathsf{J}) \gamma_2$.

Si può dimostrare che la forma della forma simplettica in $\R^{2n}$ è comune a tutte le varietà simplettiche. Questo risultato è noto come \emph{teorema di Darboux}.

\begin{definition}
  Siano $(X, \omega_X)$ e $(Y, \omega_Y)$ due varietà simplettiche. Si dice \dfn{simplettomorfismo} un diffeomorfismo $F:X\to Y$ tale che $F^*(\omega_Y) = \omega_X$. 
\end{definition}
\begin{remark}
  Se $X = Y$ e $\omega_X = \omega_Y$, $F$ si dice \dfn{trasformazione canonica}.
\end{remark}
\begin{theorem}[Darboux]
  Sia $(X,\omega)$ una varietà simplettica $2n$-dimensionale. Allora per ogni $x \in X$ esiste una carta $(U,\phi)$ con $x \in U$ tale che $\phi$ è una trasformazione canonica tra $(X,\omega)$ e $(\R^{2n}, \Omega)$ dove $\Omega$ è la forma simplettica canonica su $\R^{2n}$ data dall'\autoref{eq:R2nSympForm}.
\end{theorem}
\begin{remark}
  Nelle coordinate individuate dal teorema di Darboux, la forma simplettica $\omega(\gamma_1,\gamma_2)$ è la somma delle proiezioni sui piani $(q_i, p_i)$ del rettangolo formato da $\gamma_1$ e $\gamma_2$.
\end{remark}

Una struttura simplettica su una varietà consente di associare un campo vettoriale a ogni funzione a valori reali definita su di essa. 
\begin{definition} \label{def:hamField}
  Sia $(X, \omega)$ una varietà simplettica. Si dice \dfn{hamiltoniana} una funzione differenziabile $H: X \to \R$. Si dice \dfn{campo vettoriale hamiltoniano} il campo vettoriale $X_H$ definito su $X$ e tale che per ogni $x \in X$, per ogni $\gamma \in T_x X$
  \begin{equation} \label{eq:hamField}
    \omega\big((X_H)_x, \gamma\big) = \dd{H}_x(\gamma)
  \end{equation}
  La mappa $H \mapsto X_H$ si dice \dfn{gradiente simplettico}.
\end{definition}
\begin{remark}
  La corrispondenza tra l'$1$-forma $\dd{H}_x$ e il vettore $(X_H)_x$ è analoga a quella del teorema di rappresentazione di Fischer-Riesz, in cui però il prodotto scalare è sostituito con la forma simplettica.
\end{remark}
\begin{remark}
  Il nome di gradiente simplettico è dovuto al fatto che la mappa svolge un ruolo simile a quello del gradiente ordinario $\grad$, ovvero consente di ottenere un campo vettoriale da una funzione.
\end{remark}

L'insieme delle funzioni reali differenziabili $f:X \to \R$ definite su una varietà simplettica $(X, \omega)$ forma uno spazio vettoriale con le operazioni di somma e prodotto per scalare punto per punto. Grazie al gradiente simplettico, questo insieme ha tuttavia una struttura ulteriore, quella di \emph{algebra di Lie}, ovvero ammette un'operazione analoga al commutatore fra matrici.
\begin{definition}
  Sia $X$ un insieme e $[\cdot , \cdot]:X \times X \to X$ un'operazione denotata con $(x,y) \mapsto [x,y]$. La struttura $\big(X, [\cdot ,\cdot ]\big)$ si dice \dfn{algebra di Lie} se valgono le seguenti proprietà: \begin{enumerate}
    \item \dfn{bilinearità}: $[\lambda x + \mu y, z] = \lambda [x, z] + \mu[y,z]$ per ogni $x,y,z \in X$ e $\lambda, \mu \in  \R$
    \item \dfn{antisimmetria}: $[x,y] = -[y,x]$ per ogni $[x,y] \in X$
    \item \dfn{identità di Jacobi}: $\big[x,[y,z]\big] + \big[z,[x,y]\big] + \big[y,[z,x]\big] = 0$
  \end{enumerate}
\end{definition}

\begin{definition}
  Sia $(X, \omega)$ una varietà simplettica e siano $H,K: X \to \R$ due funzioni differenziabili con campi vettoriali hamiltoniani $X_H$ e $X_K$, si dice \dfn{parentesi di Poisson} di $H$ e $K$ \begin{equation*}
  \{H,K\} = \omega(X_H,X_K) 
  \end{equation*} 
\end{definition}
\begin{theorem}
  Sia $(X, \omega)$ una varietà simplettica, sia $\mathcal{X} = \mathcal{C}^{\infty}(X,\R)$ lo spazio delle funzioni reali differenziabili su $X$ e siano $\{\cdot , \cdot \}$ le parentesi di Poisson su $X$. Allora $\big(\mathcal{X}, \{\cdot , \cdot\}\big)$ è un'algebra di Lie.
\end{theorem}

Per una data funzione $H:X\to \R$, le parentesi di Poisson forniscono la variazione di qualsiasi altra funzione $K$ lungo le linee di flusso di $X_H$.
\begin{theorem}
  Sia $(X, \omega)$ una varietà simplettica, sia $H$ un'hamiltoniana e sia $\overline{x}:t \mapsto \overline{x}(t)$ una linea di flusso del suo campo hamiltoniano $X_H$. Allora per ogni funzione differenziabile $K:X\to \R$ vale \begin{equation*}
  \dv{t} K\big(\overline{x}(t)\big) = \{K,H\} 
  \end{equation*}  
\end{theorem}
\begin{corollary}
  Se $\{K, H\} = 0$, $K$ si conserva lungo le linee di flusso. In tal caso, $K$ è detto \dfn{integrale del moto}.
\end{corollary}

La struttura simplettica consente di associare campi vettoriali a funzioni differenziabili. In maniera simile, l'algebra delle funzioni data dalle parentesi di Poisson può essere vista come il risultato, tramite la forma simplettica, di una struttura algebrica di Lie valida per i \emph{campi vettoriali} definiti su una varietà \emph{qualsiasi}, non necessariamente simplettica. Questa struttura è detta \emph{parentesi di Lie} ed è strettamente legata al concetto di derivata di Lie.
\begin{definition}
  Siano $X$ una varietà, $V$ un campo vettoriale su $X$ e $\alpha$ una $k$-forma differenziale su $X$. Si dice \dfn{derivata di Lie} $L_V\alpha$ la $k$-forma differenziale data per ogni $x \in X$ da \begin{equation*}
  (L_V \alpha)_x = \eval{\dv{t}}_{t=0} \big( (\Phi_V^t)^* \alpha_{\Phi_V^t(x)}\big)_x
  \end{equation*} 
  dove $\Phi_V$ è il flusso di $V$ e $\dv{t}$ agisce nello spazio vettoriale delle $k$-forme su $T_x X$.
\end{definition}
\begin{remark}
  La derivata di Lie di $\alpha$ è la derivata temporale della forma che si vede seguendo una linea di flusso, valutata in $t=0$.
\end{remark}
\begin{definition}
  Siano $X$ una varietà e $V,W$ campi vettoriali su $X$. Si dice \dfn{parentesi di Lie} di $V$ e $W$ il campo vettoriale dato per ogni $x \in X$ da \begin{equation*}
  [V,W]_x = - \eval{\dv{t}}_{t=0} \Big(\big(D_{\Phi_V^t(x)} (\Phi_V^t)^{-1}\big)\big(W_{\Phi_V^t(x)}\big) \Big)_x
  \end{equation*}
  dove $\Phi_V$ è il flusso di $V$ e $\dv{t}$ agisce nello spazio vettoriale $T_x X$.
\end{definition}
\begin{remark}
  La parentesi di Lie di due campi vettoriali può anche essere interpretata come la differenza infinitesima fra il punto che risulta seguendo $V$ per un tempo $\epsilon$ e poi $W$ per $\epsilon$ e quello che risulta seguendo $W$ per $\epsilon$ e poi $V$ per $\epsilon$. In alternativa, può anche essere interpretata come commutatore delle derivazioni lungo $V$ e lungo $W$.
\end{remark}
\begin{theorem}
  Sia $(X, \omega)$ una varietà simplettica, siano $\{\cdot , \cdot \}$ le parentesi di Poisson e $[\cdot , \cdot ]$ quelle di Lie su di essa. Allora il gradiente simplettico $H \mapsto X_H$ è un omomorfismo di algebre di Lie da $\big(\mathcal{X}, \{\cdot , \cdot \} \big)$ a $\big(\mathrm{Vec}\, X, [\cdot , \cdot]\big)$, ovvero se $H,K \in \mathcal{X}$ e $X_H, X_K$ sono i loro campi vettoriali hamiltoniani \begin{equation*}
  X_{\{H,K\}} = [X_H, X_K]
  \end{equation*} 
\end{theorem}

\section{Meccanica su varietà}
Nel capitolo 1 si è definito lo spazio delle configurazioni $\mathbb{M}$ di un sistema non vincolato di $N$ particelle come il prodotto di $N$ copie di $\R^{3}$. Questo spazio è identificabile con $\R^{3N}$. Se sulle particelle viene imposto un vincolo, lo spazio delle configurazioni del sistema diventa un qualche insieme $M \subset \R^{3N}$. Come già affermato, questo insieme non avrà in generale una struttura di spazio vettoriale. I suoi punti $x$ potranno però essere individuati da $n$-uple $\vec{q}$, dette coordinate, tramite funzioni biunivoche $\phi^{-1}: \vec{q} \mapsto x$. $M$ avrà quindi la struttura più generale di una \emph{varietà differenziabile}, con atlante dato dalle funzioni $\phi: x \mapsto \vec{q}$ e dai loro domini di biunivocità. Si dice che lo spazio delle configurazioni di un sistema di $N$ particelle con $3N-n$ vincoli è una varietà $n$-dimensionale \dfn{immersa} in $\R^{3N}$, lo spazio $3N$-dimensionale delle configurazioni senza vincoli.

Per individuare univocamente un generico sistema sono necessari due elementi: il suo spazio delle configurazioni $M$ e l'energia potenziale, definita proprio sullo spazio delle configurazioni, $V: x \in M \mapsto V(x) \in \R$.  

% La velocità $\dot{x}$ di un sistema è interpretabile come un vettore tangente allo spazio delle configurazioni nel punto $x(t)$, facendo corrispondere a ogni $\gamma \in T_x M$ il vettore $\dot{x} = \dot{\gamma}(0)$, che appartiene a $\R^{n}$ poiché $\gamma: ]-\epsilon,\epsilon[ \to M \subset \R^{n}$ ed è tangente nel senso ordinario (ortogonale al gradiente dei vincoli) alla varietà immersa $M$. Su ogni spazio tangente è quindi definita l'energia cinetica come forma quadratica $K_x: \dot{x} \mapsto \frac{1}{2} \dot{x}^2$.

La velocità $\dot{x}$ di un sistema è definita come derivata temporale $\dv{\overline{x}}{t}()(t)$ del suo moto $\overline{x}: t \in I \mapsto \overline{x}(t) \in M$, dove $I$ è un intervallo reale. Per definizione, un vettore tangente alla varietà $M$ nel punto $x \in M$ è una classe di equivalenza di cammini infinitesimi che passano da $x$ al tempo $0$. Per un dato $x =\overline{x}(t)$, si può traslare il tempo e ridurne il dominio in modo da ottenere un nuovo cammino $\gamma:]-\epsilon, \epsilon\,[$ tale che $x = \gamma(0)$. Questo cammino sarà inoltre tale che $\dot{x} = \dv{\gamma}{t}()(0)$. Siccome $M \subset \R^{n}$, una scelta valida per $\phi$ è l'identità $\identity$. Quindi, perché un secondo cammino $\gamma'$ sia equivalente a $\gamma$, è necessario e sufficiente che \begin{equation*}
\eval{\dv{t}}_{t=0} \gamma'(t) = \eval{\dv{t}}_{t=0} \gamma(t) = \dot{x}
\end{equation*}
per definizione: la velocità $\dot{x}$ caratterizza la classe $[\gamma]$ dello spazio tangente a $M$ in $x$. Questo potrà quindi essere identificato con lo spazio delle velocità possibili quando il sistema ha configurazione $x$. Si può quindi dire che la velocità di un sistema è un vettore tangente allo spazio delle configurazioni nella configurazione attuale del sistema. L'energia cinetica, funzione della velocità, è quindi definita per ciascun $x \in M$ su $T_x M$ da $K_x: \dot{x} \mapsto \frac{1}{2} \dot{x}^2$, dove la norma è la norma in $\R^{n}$.

La lagrangiana di un sistema dipende sia dall'energia potenziale che dall'energia cinetica, e dunque sia dalla configurazione del sistema che dalla sua velocità. Essa deve quindi essere definita su $TM$, il fibrato tangente di $M$. Le energie cinetica e potenziale dovranno quindi cambiare dominio da rispettivamente $T_x M$ e $M$ a $TM$ per poter formare la lagrangiana. Sia $(x,\dot{x}) \in TM$, con $x \in M$ e $\dot{x} \in T_x M$ ciò si può fare definendo \begin{equation*}
\begin{aligned}
  &V(x,\dot{x})|_{TM} = V(x)|_M \\
  &K(x, \dot{x})|_{TM} = K(\dot{x})|_{T_x M}
\end{aligned}
\end{equation*} 
Si può a questo punto definire la lagrangiana come una funzione differenziabile $\mathcal{L}:TM\to \R$ data per $x \in M$ e $\dot{x} \in T_x M$ da \begin{equation*}
\mathcal{L}(x,\dot{x}) = K(\dot{x}) - V(x)
\end{equation*}

La trasformata di Legendre rispetto alle velocità trasforma la lagrangiana $\mathcal{L}(x,\dot{x})$ nell'hamiltoniana $\mathcal{H}(x,\dot{x}^*)$ dove gli $\dot{x}^*$ sono elementi dello spazio duale a quello degli $\dot{x}$. Ma poiché $\dot{x} \in T_x M$, ciò significa che $\dot{x}^* \in T_x^* M$, e quindi l'hamiltoniana di un sistema è definita sul fibrato cotangente del suo spazio delle configurazioni. Il momento generalizzato $\dot{x}^*$ è infatti un vettore cotangente, dato da \begin{equation*}
  \dot{x}^* = \dd_{\dot{x}} \mathcal{L}
\end{equation*}
Lo spazio delle fasi può quindi essere definito in maniera indipendente dalle coordinate come il fibrato cotangente $T^* M$ dello spazio delle configurazioni. Se una regione di $M$ è coperta dalla carta $(U,\phi)$, un punto $x \in U$ si può individuare con $\vec{q} = (q_1, \ldots, q_n) = (\phi_1(x), \ldots, \phi_n(x))$ e il momento generalizzato cotangente a $M$ in $x$ si può individuare con $\vec{p}=(p_1, \ldots, p_n)$, le coordinate nella base duale data da $\{\dd_x \phi_1, \ldots, \dd_x \phi_n\} $. Si recupera così la definizione basata sulle coordinate.

La \autoref{def:hamField} estende la \autoref{eq:hamFieldUnconstr} al caso di varietà che non costituiscono spazi vettoriali. Si consideri infatti una varietà simplettica $2n$-dimensionale $X$. In una carta $(U,\phi)$ siano i suoi punti $x \in U$ individuati dalle coordinate $(x_1, \ldots, x_{2n})$ e i vettori $\gamma$ tangenti a $X$ in un punto $y \in U$ individuati dalle coordinate $(\gamma_1, \ldots, \gamma_{2n})$. Allora il differenziale $\dd{\mathcal{H}}$ è rappresentato dalla trasposta del gradiente su queste coordinate $(\grad{\mathcal{H}})^T$, così che \begin{equation*}
\dd{\mathcal{H}}(\gamma) = (\grad{\mathcal{H}})^T \begin{pmatrix} \gamma_1\\ \vdots\\ \gamma_{2n} \end{pmatrix}
\end{equation*} 
Allo stesso tampo, la forma simplettica $\omega$ sarà rappresentata da $\mathsf{J} = \left( \begin{smallmatrix}
  0 & -\mathsf{I} \\ \mathsf{I} & 0
\end{smallmatrix}  \right) $ e il campo $X_{\mathcal{H}}$ da un vettore $\mathsf{X_{\mathcal{H}}}$, così che
\begin{equation*}
\dd{\mathcal{H}}(\gamma) = \omega(X_{\mathcal{H}}, \gamma) = \mathsf{X_{\mathcal{H}}}^T\, \mathsf{J}\, \begin{pmatrix} \gamma_1\\ \vdots\\ \gamma_{2n} \end{pmatrix}
\end{equation*} 
Da ciò segue che $\mathsf{X_{\mathcal{H}}}^T \mathsf{J} = (\grad{\mathcal{H}})^T$. Trasponendo e moltiplicando per $\mathsf{J}$, siccome $\mathsf{J}^2 =- \mathsf{I}$, si ha
\begin{equation*}
\mathsf{X}_{\mathcal{H}} = -\mathsf{J} \grad{\mathcal{H}} 
\end{equation*}
ovvero per le coordinate vale \begin{equation*}
\dot{q}_i = \pdv{\mathcal{H}}{p_i} \qquad \dot{p}_i = - \pdv{\mathcal{H}}{q_i}
\end{equation*}
che sono le equazioni di Hamilton. Le linee di flusso del campo vettoriale hamiltoniano minimizzano quindi l'azione associata alla lagrangiana ridotta sullo spazio delle configurazioni vincolato. Esse minimizzano quindi l'azione, e dunque rappresentano le posizioni e le velocità che il sistema ha nel suo moto. Con quest'ultimo tassello è finalmente possibile porre una formulazione della meccanica di sistemi con vincoli olonomi arbitrari puramente geometrica e libera da coordinate.

Un sistema fisico è definito dal suo \emph{spazio delle configurazioni}, una varietà differenziabile $n$-dimensionale $M$ in cui ogni punto $m$ corrisponde a una diversa configurazione delle componenti del sistema, e da una funzione \emph{hamiltoniana} $\mathcal{H}: M\to \R$, che ne governa il movimento. Lo \emph{stato} di un sistema è rappresentato da un punto nello \emph{spazio delle fasi}, il fibrato cotangente $X = T^* M$ dello spazio delle configurazioni. Sullo spazio delle fasi è definita la \emph{forma simplettica canonica} $\omega$. Tramite essa è definito il \emph{campo vettoriale hamiltoniano} $X_{\mathcal{H}}$, ovvero quel campo tale che, per un qualsiasi vettore $\gamma$ tangente allo spazio delle fasi in un qualche punto $x \in X$, prenderne la forma simplettica con il campo hamiltoniano $\omega_{x}(X_{\mathcal{H}}, \gamma)$ sia uguale a prenderne il differenziale dell'hamiltoniano $\dd_{x}{\mathcal{H}}(\gamma)$. Se il sistema parte da uno stato iniziale $(x_0, \dot{x}^*_0)$, il suo stato percorre la linea di flusso di $X_{\mathcal{H}}$ che parte da $(x_0, \dot{x}^*_0)$. L'evoluzione di una qualsiasi funzione di stato $F$ è data dalla parentesi di Poisson $\{F, \mathcal{H}\}$.